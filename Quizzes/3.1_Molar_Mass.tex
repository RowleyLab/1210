\documentclass[11pt, letterpaper]{memoir}
\usepackage{HomeworkStyle}

\begin{document}
	\begin{center}
		{\large	Quiz 3.1 --  Molar Mass}
	\end{center}
{\large Name: \rule[-1mm]{4in}{.1pt}
	
	\subsection*{Question 1}
	Give the molar mass for the following substances:
	\begin{itemize}
		\item \vspace{0.5em} \ch{C2H4O2}
		\item \vspace{0.75em} \ch{Fe(NO3)2}
	\end{itemize}
	
	\vspace{0.5em}
	\subsection*{Question 2}
	How many moles are in $2.50~g$ of each substance:
	\begin{itemize}
		\item \vspace{0.5em} \ch{NiSO4$\cdot$ 2 H2O}
		\item \vspace{0.75em} \ch{C2H5OH}
	\end{itemize}
	
	
	\vspace{0.5em}
	\subsection*{Question 3}
	How many $g$ are in $0.125~mol$ of each substance:
	\begin{itemize}
		\item \vspace{0.5em} \ch{CH2O}
		\item \vspace{0.75em} \ch{AlCl3}
	\end{itemize}

	\vspace{0.5em}
	\subsection*{Question 4}
  Three compounds share the empirical formula of \ch{CH2O}, and the molar masses listed below. Give the molecular formula for each compound
  \begin{itemize}
    \item $M=30.03\nicefrac{g}{mol}$

      \vspace{2em}
    \item $M=60.05\nicefrac{g}{mol}$

      \vspace{2em}
    \item $M=180.16\nicefrac{g}{mol}$

      \vspace{2em}
  \end{itemize}
\newpage
\pagestyle{empty}
\addtocounter{page}{-1}
\newgeometry{vmargin=1in, hmargin=1.25in}
\section*{\emph{The Mortician in San Francisco}}
\paragraph{By Randall Mann}~
\begin{verse}
	This may sound queer,\\
	but in 1985 I held the delicate hands\\
	of Dan White:\\
	I prepared him for burial; by then, Harvey Milk\\
	was made monument—no, myth—by the years\\
	since he was shot.

	I remember when Harvey was shot:\\
	twenty, and I knew I was queer.\\
	Those were the years,\\
	Levi’s and leather jackets holding hands\\
	on Castro Street, cheering for Harvey Milk—\\
	elected on the same day as Dan White.

	I often wonder about Supervisor White,\\
	who fatally shot\\
	Mayor Moscone and Supervisor Milk,\\
	who was one of us, a Castro queer.\\
	May 21, 1979: a jury hands\\
	down the sentence, seven years—

	in truth, five years—\\
	for ex-cop, ex-fireman Dan White,\\
	for the blood on his hands;\\
	when he confessed that he had shot\\
	the mayor and the queer,\\
	a few men in blue cheered. And Harvey Milk?

	Why cry over spilled milk,\\
	some wondered, semi-privately, for years—\\
	it meant “one less queer.”\\
	The jurors turned to White.\\
	If just the mayor had been shot,\\
	Dan might have had trouble on his hands—

	but the twelve who held his life in their hands\\
	maybe didn’t mind the death of Harvey Milk;\\
	maybe, the second murder offered him a shot\\
	at serving only a few years.\\
	In the end, he committed suicide, this Dan White.\\
	And he was made presentable by a queer.
\end{verse}
\end{document}
