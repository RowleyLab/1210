\documentclass[11pt, letterpaper]{memoir}
\usepackage{HomeworkStyle}

\begin{document}
	\begin{center}
		{\large	Quiz 6.2 -- Orbitals and Quantum Numbers}
	\end{center}
{\large Name: \rule[-1mm]{4in}{.1pt}
	
	
	\noindent Below is the hydrogenic atom energy level diagram (which only applies for atoms with 1 electron). Two different possible electron states are shown.
	\begin{center}
		
		
		
		{\large
		\begin{tabular}{ccccccccccc}
			\rule[-1pt]{1.75em}{0.5pt} &~&\rule[-1pt]{1.75em}{0.5pt}&\rule[-1pt]{1.75em}{0.5pt}&\rule[-1pt]{1.75em}{0.5pt} &~& \rule[-1pt]{1.75em}{0.5pt}&\rule[-1pt]{1.75em}{0.5pt}&\rule[-1pt]{1.75em}{0.5pt}&\rule[-1pt]{1.75em}{0.5pt}&\rule[-1pt]{1.75em}{0.5pt}\\
			{$3s$} &&&{$3p$} &&&&&{$3d$}
		\end{tabular}
		
	
		
		\begin{tabular}{ccccc}
			\rule[-1pt]{1.75em}{0.5pt} &~&\rule[-1pt]{1.75em}{0.5pt}&\rule[-1pt]{1.75em}{0.5pt}&\rule[-1pt]{1.75em}{0.5pt}\\
			{$2s$} &&&{$2p$}
		\end{tabular}
		
		
		
		\begin{tabular}{c}
			\rule[-1pt]{1.75em}{0.5pt} \\
			{$1s$}
		\end{tabular}
		}
	\end{center}

	\subsection*{Question 1}
	Give the four quantum numbers for each of the two electrons shown on the diagram above
	
	\vspace{2em}
	\subsection*{Question 2}
	On the diagram above, draw electrons with the following quantum numbers:
	\begin{itemize}
		\item $n=2$, ~$l=1$, ~$m_l=1$, ~$m_s=\frac{1}{2}$
		\item $n=3$, ~$l=2$, ~$m_l=0$, ~$m_s=-\frac{1}{2}$
		\item $n=1$, ~$l=0$, ~$m_l=0$, ~$m_s=-\frac{1}{2}$
	\end{itemize}
	
	\subsection*{Question 3}
	Sketch one of each of the first three orbital types ($s$, $p$, and $d$)

	\vspace{3em}
	\subsection*{Question 4}
	How many electrons can occupy the $3d$ subshell
	
	\vspace{1em}
	\subsection*{Question 5}
	Which of the following subshells does \emph{not} exist (which breaks the rules about orbitals)?
	
	{\large $2s$ \hspace{3em} $3s$ \hspace{3em} $4f$ \hspace{3em} $1p$ \hspace{3em} $4p$ \hspace{3em} $3d$}


\newpage
\pagestyle{empty}
\addtocounter{page}{-1}
\section*{\emph{On Shakespeare. 1630}}
\paragraph{By John Milton}~
\begin{verse}
	What needs my Shakespeare for his honoured bones,\\
	The labor of an age in pilèd stones,\\
	Or that his hallowed relics should be hid\\
	Under a star-ypointing pyramid?\\
	Dear son of Memory, great heir of fame,\\
	What need’st thou such weak witness of thy name?\\
	Thou in our wonder and astonishment\\
	Hast built thyself a live-long monument.\\
	For whilst to th’ shame of slow-endeavouring art,\\
	Thy easy numbers flow, and that each heart\\
	Hath from the leaves of thy unvalued book\\
	Those Delphic lines with deep impression took,\\
	Then thou, our fancy of itself bereaving,\\
	Dost make us marble with too much conceiving;\\
	And so sepúlchred in such pomp dost lie,\\
	That kings for such a tomb would wish to die.
\end{verse}

\vspace{8em}
\hspace{0.3\linewidth}
\begin{minipage}{0.7\linewidth}
\section*{\emph{Odes III: XXX (23 BCE)}}
\paragraph{Horace}~
\begin{verse}
exegi monumentum aere perennius\\
regalique situ pyramidum altius,\\
quod non imber edax, non Aquilo inpotens\\
possit diruere \ldots\\
Non omnis moriar.

I have built a monument more lasting than bronze,\\
higher than the Pyramids’ regal structures,\\
that no consuming rain, nor wild north wind\\
can destroy \ldots\\
I shall not wholly die.
\end{verse}
\end{minipage}
\end{document}
