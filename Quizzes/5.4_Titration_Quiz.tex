\documentclass[11pt, letterpaper]{memoir}
\usepackage{HomeworkStyle}

\begin{document}
	\begin{center}
		{\large	Quiz 5.4 -- Titration}
	\end{center}
{\large Name: \rule[-1mm]{4in}{.1pt}
	
	\subsection*{Question 1 (1 points)}
	$25.00~ml$ of \ch{HCl} with unknown concentration were titrated to the end-point with $37.21~ml$ of $0.150~M$ \ch{NaOH}. Find the initial $\left[\ch{HCl}\right]$
	
	\vspace{9em}
	\subsection*{Question 2 (2 points)}
	$25.00~ml$ of \ch{Ba(OH)2} with unknown concentration were titrated to the end-point with $42.85~ml$ of $0.350~M$ \ch{HNO3}. Find the initial $\left[\ch{Ba(OH)2}\right]$
	
	\vspace{10em}
	\subsection*{Question 3 (2 points)}
	Color-changing indicators are available for redox reactions as well as for acid-base reactions. Consider the following reaction:
	
	\ch{2 Au^{3+}(aq) + 3 Cr^{2+}(aq) -> 2 Au(s) + 3 Cr^{4+}(aq)}
	
	\noindent $50.00~ml$ of \ch{Au(NO3)3} with unknown concentration were titrated to the end-point with $28.63~ml$ of $0.125~M$ \ch{Cr(C2H3O2)2}. Find the initial $\left[\ch{Au(NO3)3}\right]$

\newpage
\pagestyle{empty}
\addtocounter{page}{-1}
\section*{\emph{Sonnet 18: Shall I compare thee to a summer’s day?}}
\paragraph{By William Shakespeare}~
\begin{verse}
	Shall I compare thee to a summer’s day?\\
	Thou art more lovely and more temperate:\\
	Rough winds do shake the darling buds of May,\\
	And summer’s lease hath all too short a date;\\
	Sometime too hot the eye of heaven shines,\\
	And often is his gold complexion dimm'd;\\
	And every fair from fair sometime declines,\\
	By chance or nature’s changing course untrimm'd;\\
	But thy eternal summer shall not fade,\\
	Nor lose possession of that fair thou ow’st;\\
	Nor shall death brag thou wander’st in his shade,\\
	When in eternal lines to time thou grow’st:\\
	\hspace{0.5em} So long as men can breathe or eyes can see,\\
	\hspace{0.5em} So long lives this, and this gives life to thee.
\end{verse}
\end{document}
