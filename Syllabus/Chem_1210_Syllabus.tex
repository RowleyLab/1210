\documentclass[12pt, letterpaper]{article}
\usepackage{SyllabusStyle}

\begin{document}
\begin{center}
{\Large \textsc{Principles of Chemistry I}}

CHEM 1210
\end{center}

\begin{center}
	{\large Fall 2025}
\end{center}
\begin{center}
	\rule{0.99\textwidth}{0.4pt}
	\begin{tabular}{llcll}
		\textbf{Instructor:} & Matthew Rowley           &  & \textbf{Office Hours:} & Daily 11:00 am -- 12:00 am \\
		\textbf{Telephone:}  & (435) 586-7875           &  &                        &  \\
		\textbf{Email:}      & matthewrowley$1$@suu.edu &  & \textbf{Office:}       & SC-220                   \\
		\multicolumn{5}{c}{Please include the course number in the subject line of all correspondence.}
	\end{tabular}
	\rule{0.99\textwidth}{0.4pt}
\end{center}

\section*{Course Description} 
This is an introductory chemistry course designed for students in engineering, physical science, pre-medical, pre-dental, pre-pharmacy, or pre-veterinary medicine. It is for all students who need more than one year of chemistry, and presents the foundational principles on which all other chemistry courses build.

\paragraph{General Education:}
This course is required for beginning Chemistry majors requiring more than one year of chemistry. It can also be taken by other students to fulfill the General Education (GE) requirements for ``Knowledge Area:Physical Science (P).'' The principles of chemistry underlie much of modern society. Students will explore the basic principles of chemistry, allowing them to better understand the world around them at a basic level. This will prepare students to be informed citizens of the modern world in topics such as medicine, materials, energy and pollution.

\paragraph{Prerequisites:}
A proficient understanding of algebra will be required for this course. Either MATH 1050 (College Algebra), or a background of algebra from high school will suffice.

\paragraph{Concurrent requisite:}
CHEM 1215 -- Principles of Chemistry Lab I

\paragraph{Required Course Materials:} ~

Access to \emph{Achieve} online homework and Interactive General Chemistry digital textbook (details below)

\paragraph{Recommended Supplementary Materials:} ~

\emph{Preparing for Your ACS Examination in General Chemistry: The Official Guide} Sold in our department office to fund Chem Club activities

\paragraph{Student Learning Outcomes:}
\begin{description}
  \item[Knowledge of the Physical and Natural World] Students will recall, interpret, compare, explain, and apply chemistry terminology and theory
  \item[Quantitative Literacy] Students will use chemical equations, graphs, and tables to interpret and communicate chemical information.
  \item[Inquiry and Analysis] Students will solve complex chemical problems.
  \item[Critical Thinking] Students will make decisions based on conceptualizing, applying, and analyzing information from different sources.
\end{description}

\section*{Tentative Schedule}
This class will meet on Mondays, Tuesdays, Wednesdays, and Fridays from 3:00 -- 3:50 in room 230 of the Science Center (SC).

\noindent For the best lecture experience, read the indicated textbook chapter \emph{before} each lecture.

~
\noindent
\begin{tabular}{rcccc}
& Date && Topic & Chapter\\
\midrule
Week 1 & W, Aug. 27&& Chemistry in Context & 1.1\\
& F, Aug. 29&& Phases and Classification of Matter & 1.2\\
\midrule
Week 2 & M, Sep. 1& \multicolumn{3}{l}{\textbf{Labor Day - No Class!}}\\
& T, Sep. 2&& Physical and Chemical Properties & 1.3\\
& W, Sep. 3&& Measurements & 1.4\\
& F, Sep. 5&& Measurement Uncertainty, Accuracy, and Precision & 1.5\\
\midrule
Week 3 & M, Sep. 8&& Mathematical Treatment of Measurement Results & 1.6\\
& T, Sep. 9&& Early Ideas in Atomic Theory & 2.1\\
& W, Sep. 10&& Evolution of Atomic Theory & 2.2\\
& F, Sep. 12&& Atomic Structure and Symbolism & 2.3\\
\midrule
Week 4 & M, Sep. 15&& Chemical Formulas & 2.4\\
& T, Sep. 16&& The Periodic Table & 2.5\\
& W, Sep. 17&& Ionic and Molecular Compounds & 2.6\\
& F, Sep. 19&& Chemical Nomenclature & 2.7\\
\end{tabular}

\noindent
\begin{tabular}{rcccc}
& Date && Topic & Chapter\\
\midrule
Week 5 & M, Sep. 22& \multicolumn{3}{l}{\textbf{Catch-up/Review Day for Exam 1: Chapters 1 and 2}}\\
& T, Sep. 23&& Formula Mass and the Mole Concept & 3.1\\
& W, Sep. 24&& Determining Empirical and Molecular Formulas & 3.2\\
& F, Sep. 26&& Molarity & 3.3\\
\midrule
Week 6 & M, Sep. 29&& Other Units for Solution Concentration & 3.4\\
& T, Sep. 30&& Writing and Balancing Chemical Equations & 4.1\\
& W, Oct. 1&& Classifying Chemical Reactions & 4.2\\
& F, Oct. 3&& Reaction Stoichiometry & 4.3\\
\midrule
Week 7 & M, Oct. 6&& Reaction Yields & 4.4\\
& T, Oct. 7&& Quantitative Chemical Analysis & 4.5\\
& W, Oct. 8& \multicolumn{3}{l}{\textbf{Catch-up/Review Day for Exam 2: Chapters 3 and 4}}\\
& F, Oct. 10&& Energy Basics & 5.1\\
\midrule
Week 8 & M, Oct. 13& \multicolumn{3}{l}{\textbf{Fall Break - No Class!}}\\
& T, Oct. 14& \multicolumn{3}{l}{\textbf{Fall Break - No Class!}}\\
& W, Oct. 15&& Calorimetry & 5.2\\
& F, Oct. 17&& Enthalpy & 5.3\\
\midrule
Week 9 & M, Oct. 20&& Electromagnetic Energy & 6.1\\
& T, Oct. 21&& The Bohr Model & 6.2\\
& W, Oct. 22&& Development of Quantum Theory & 6.3\\
& F, Oct. 24&& Electronic Structure of Atoms (Electron Configurations) & 6.4\\
\midrule
Week 10 & M, Oct. 27&& Periodic Variations in Element Properties & 6.5\\
& T, Oct. 28& \multicolumn{3}{l}{\textbf{Catch-up/Review Day for Exam 3: Chapters 5 and 6}}\\
& W, Oct. 29&& Ionic Bonding & 7.1\\
& F, Oct. 31&& Covalent Bonding & 7.2\\
\end{tabular}

\noindent
\begin{tabular}{rcccc}
& Date && Topic & Chapter\\
\midrule
Week 11 & M, Nov. 3&& Lewis Symbols and Structures & 7.3\\
& T, Nov. 4&& Formal Charges and Resonance & 7.4\\
& W, Nov. 5&& Strengths of Ionic and Covalent Bonds & 7.5\\
& F, Nov. 7&& Molecular Structure and Polarity & 7.6\\
\midrule
Week 12 & M, Nov. 10&& Valence Bond Theory & 8.1\\
& T, Nov. 11&& Hybrid Atomic Orbitals & 8.2\\
& W, Nov. 12&& Multiple Bonds & 8.3\\
& F, Nov. 14&& Molecular Orbital Theory & 8.4\\
\midrule
Week 13 & M, Nov. 17&& Gas Pressure & 9.1\\
& T, Nov. 18&& Relating Pressure, etc.: The Ideal Gas Law & 9.2\\
& W, Nov. 19&& Stoichiometry of Gaseous Substances, etc. & 9.3\\
& F, Nov. 21&& Effusion and Diffusion of Gases & 9.4\\
\midrule
Week 14 & M, Nov. 24& \multicolumn{3}{l}{\textbf{Thanksgiving Break - No Class!}}\\
& T, Nov. 25& \multicolumn{3}{l}{\textbf{Thanksgiving Break - No Class!}}\\
& W, Nov. 26& \multicolumn{3}{l}{\textbf{Thanksgiving Break - No Class!}}\\
& F, Nov. 28& \multicolumn{3}{l}{\textbf{Thanksgiving Break - No Class!}}\\
\midrule
Week 15 & M, Dec. 1&& The Kinetic-Molecular Theory & 9.5\\
& T, Dec. 2&& Non-Ideal Gas Behavior & 9.6\\
& W, Dec. 3& \multicolumn{3}{l}{\textbf{Catch-up/Review Day for Exam 4: Chapters 7--9}}\\
& F, Dec. 5&& Intermolecular Forces & 10.1\\
\midrule
\midrule
Finals Week & M, Dec. 8 & \multicolumn{3}{l}{\textbf{Final Exam} ~ 3:00-4:50 ~ Bring a pencil and a scantron sheet}\\
\end{tabular}

\section*{Course Requirements}
Grades will be based on the following items:
\begin{description}
	\item[4 Midterm Exams] 40\%
	\item[Final Exam] 20\%
	\item[Quizzes] 10\%
	\item[Online Homework] 15\%
	\item[Textbook Homework] 15\%
\end{description}
Final Grades will be assigned according to the following grade scale:

\begin{tabular}{cl|c|cl}
	Percentage & Grade &  & Percentage & Grade \\ \midrule
	100--93.0  & A     &  & 77.0--73.0 & C     \\
	93.0--90.0 & A-    &  & 73.0--70.0 & C-    \\
	90.0--87.0 & B+    &  & 70.0--67.0 & D+    \\
	87.0--83.0 & B     &  & 67.0--63.0 & D     \\
	83.0--80.0 & B-    &  & 63.0--60.0 & D-    \\
	80.0--77.0 & C+    &  & < 60.0     & F
\end{tabular}

\paragraph{Midterm Exams:}
There are four midterm exams administered in the testing center through Canvas. Each exam is to be completed in a two-hour session during the indicated week unless prior arrangements have been made. A sparse resource sheet will be provided for each exam, but students should prepare by memorizing appropriate formulas, tables, etc.

\paragraph{Final Exam:}
The final exam is a comprehensive and nationally normalized exam prepared by the American Chemical Society.

\paragraph{Quizzes:}
Quizzes will be handed out at the beginning of most days. The purpose of these quizzes is to provide practice for the exams (since I am the author of both quizzes and exams) and to encourage punctual attendance.

\paragraph{Online Homework:}
Each chapter will have an assignment on Canvas. These assignments cover material from the entire chapter, and so it is recommended to complete bit by bit over several sessions.

\paragraph{Textbook Homework:}
Problems from the textbook will be assigned most days. These short assignments should be completed on paper and turned in the next day.

\paragraph{Attendance Policy:}
Students are expected to attend class. If you must miss class, contact the instructor

\paragraph{Late Work Policy:}
Textbook homework and take-home quizzes will be due on a day when class is regularly scheduled. All work is to be turned in at the \emph{beginning} of the class period, and late work will not be accepted. Online homework has due-dates at 11:55 pm on the Sunday after the relevant material has been covered.

\paragraph{Make-up Work Policy:}
In general, there will be no opportunity to make up missed work, including in-class quizzes. If you must miss class, please do any assigned work in advance, and arrange to turn it in early

\section*{Miscellany}

\paragraph{Scientific Calculator:}
There are many different ways to calculate figures during homework. It is tempting to rely on Online resources such as \href{http://www.wolframalpha.com}{http://www.wolframalpha.com}, or to simply use a calculator application on a smartphone. During exams, however, any devices capable of connecting to the Internet will \emph{not} be allowed. You will instead need a scientific calculator capable of performing exponentiation and logarithms for the exams. Using this calculator exclusively while doing homework will ensure that you are proficient with it for use during exams.

\paragraph{Academic Credit:}
According to the federal definition of a Carnegie credit hour: A credit hour of work is the equivalent of approximately 60 minutes of class time or independent study work. A minimum of 45 hours of work by each student is required for each unit of credit. Credit is earned only when course requirements are met. One (1) credit hour is equivalent to 15 contact hours of lecture, discussion, testing, evaluation, or seminar, as well as 30 hours of student homework. An equivalent amount of work is expected for laboratory work, internships, practica, studio, and other academic work leading to the awarding of credit hours. Credit granted for individual courses, labs, or studio classes range from 0.5 to 15 credit hours per semester.

\paragraph{Academic Freedom:}
SUU is operated for the common good of the greater community it serves. The common good depends upon the free search for truth and its free exposition. Academic Freedom is the right of faculty to study, discuss, investigate, teach, and publish. Academic Freedom is essential to these purposes and applies to both teaching and research. 

\noindent
Academic Freedom in the realm of teaching is fundamental for the protection of the rights of the faculty member and of you, the student, with respect to the free pursuit of learning and discovery. Faculty members possess the right to full freedom in the classroom in discussing their subjects. They may present any controversial material relevant to their courses and their intended learning outcomes, but they shall take care not to introduce into their teaching controversial materials which have no relation to the subject being taught or the intended learning outcomes for the course.

\noindent
As such, students enrolled in any course at SUU may encounter topics, perspectives, and ideas that are unfamiliar or controversial, with the educational intent of providing a meaningful learning environment that fosters your growth and development. These parameters related to Academic Freedom are included in SUU \href{https://www.suu.edu/policies/06/06.html}{Policy 6.6}.

\paragraph{Academic Misconduct:}
Scholastic honesty is expected of all students. Dishonesty will not be tolerated and will be prosecuted to the fullest extent (see \href{https://www.suu.edu/policies/06/33.html}{SUU Policy 6.33}). You are expected to have read and understood the current SUU student conduct code (\href{https://www.suu.edu/policies/11/02.html}{SUU Policy 11.2}) regarding student responsibilities and rights, the intellectual property policy (\href{https://www.suu.edu/policies/05/52.html}{SUU Policy 5.52}), information about procedures, and what constitutes acceptable behavior. 

\noindent
\underline{Please Note}: The use of websites or services that sell essays is a violation of these policies; likewise, the use of websites or services that provide answers to assignments, quizzes, or tests is also a violation of these policies. Regarding the use of Generative Artificial Intelligence (AI), you should check with your individual course instructor.

\paragraph{ADA Statement:}
Students with medical, psychological, learning, or other disabilities desiring academic adjustments, accommodations, or auxiliary aids will need to contact the \href{https://www.suu.edu/disabilityservices/}{Disability Resource Center}, located in Room 206F of the Sharwan Smith Center or by phone at (435) 865-8042. The Disability Resource Center determines eligibility for and authorizes the provision of services.

\noindent
If your instructor requires attendance, you may need to seek an ADA accommodation to request an exception to this attendance policy. Please contact the Disability Resource Center to determine what, if any, ADA accommodations are reasonable and appropriate.

\paragraph{Non-Discrimination Statement:}
SUU is committed to fostering an inclusive community of lifelong learners and believes our university's encompassing of different views, beliefs, and identities makes us stronger, more innovative, and better prepared for the global society. 

\noindent
SUU does not discriminate on the basis of race, religion, color, national origin, citizenship, sex (including sex discrimination and sexual harassment), sexual orientation, gender identity, age, ancestry, disability status, pregnancy, pregnancy-related conditions, genetic information, military status, veteran status, or other bases protected by applicable law in employment, treatment, admission, access to educational programs and activities, or other University benefits or services.

\noindent
SUU strives to cultivate a campus environment that encourages freedom of expression from diverse viewpoints. We encourage all to dialogue within a spirit of respect, civility, and decency. 

\noindent
For additional information on non-discrimination, please see \href{https://www.suu.edu/policies/05/27.html}{Policy 5.27} and/or visit:\newline \href{https://www.suu.edu/nondiscrimination.}{https://www.suu.edu/nondiscrimination.}

\paragraph{Pregnancy:}
Students who are or become pregnant during this course may receive reasonable modifications to facilitate continued access and participation in the course. Pregnancy and related conditions are broadly defined to include pregnancy, childbirth, termination of pregnancy, lactation, related medical conditions, and recovery. To obtain reasonable modifications, please make a request to: \href{mailto:title9@suu.edu}{title9@suu.edu}. To learn more visit: \href{https://www.suu.edu/titleix/pregnancy.html}{https://www.suu.edu/titleix/pregnancy.html}.

\paragraph{Mandatory Reporting:}
University policy (\href{https://www.suu.edu/policies/05/60.html}{SUU Policy 5.60}) requires instructors to report disclosures received from students that indicate they have been subjected to sexual misconduct/harassment. The University defines sexual harassment consistent with Federal Regulations (\href{https://www.ecfr.gov/current/title-34/subtitle-B/chapter-I/part-106/subpart-D}{34 C.F.R. Part 106, Subpart D}) to include quid pro quo, hostile environment harassment, sexual assault, dating violence, domestic violence, and stalking. When students communicate this information to an instructor in-person, by email, or within writing assignments, the instructor will report that to the Title IX Coordinator to ensure students receive support from the Title IX Office. A reporting form is available at \href{https://cm.maxient.com/reportingform.php?SouthernUtahUniv}{https://cm.maxient.com/reportingform.php?SouthernUtahUniv}

\paragraph{Emergency Management Statement:}
In case of emergency, the university's Emergency Notification System (ENS) will be activated. Students are encouraged to maintain updated contact information using the link on the homepage of the \emph{mySUU} portal. In addition, students are encouraged to familiarize themselves with the Emergency Response Protocols posted in each classroom. Detailed information about the university's emergency management plan can be found at: \href{http://www.suu.edu/emergency}{http://www.suu.edu/emergency}

\paragraph{HEOA Compliance Statement:}
For a full set of Higher Education Opportunity Act (HEOA) compliance statements, please visit \href{https://www.suu.edu/heoa}{https://www.suu.edu/heoa}. The sharing of copyrighted material through peer-to-peer (P2P) file sharing, except as provided under U.S. copyright law, is prohibited by law; additional information can be found at \newline\href{https://my.suu.edu/help/article/1096/heoa-compliance-plan}{https://my.suu.edu/help/article/1096/heoa-compliance-plan}.

\noindent
You are also expected to comply with policies regarding intellectual property (\href{https://www.suu.edu/policies/05/52.html}{SUU Policy 5.52}) and copyright (\href{https://www.suu.edu/policies/05/54.html}{SUU Policy 5.54}).

\paragraph{SUUSA Statement:}
As a student at SUU, you have representation from the SUU Student Association (SUUSA) which advocates for student interests and helps work as a liaison between the students and the university administration. You can submit MySUU Voice feedback by going to \href{https://www.suu.edu/suusa/voice}{https://www.suu.edu/suusa/voice}. Likewise, you can learn more about SUUSA’s Executive Council at \href{https://www.suu.edu/suusa/executive-council}{https://www.suu.edu/suusa/executive-council} and about all of SUUSA’s Student Senators at \href{https://www.suu.edu/suusa/senate}{https://www.suu.edu/suusa/senate}. If you have any specific concerns regarding any of your courses, please contact the SUUSA VP of Academics at: \href{suusa_academicsvp@suu.edu}{suusa\_\ignorespaces academicsvp@suu.edu}.

\paragraph{Thriving Thunderbirds:}
Mental health is essential for your academic success. If you are struggling with mental health issues, SUU provides resources, support, and services to help you. Please visit \href{https://www.suu.edu/mentalhealth}{https://www.suu.edu/mentalhealth} for access to these valuable resources.

\noindent
If you need assistance navigating any of the resources, please contact \href{https://www.suu.edu/caps/}{Counseling and Psychological Services}, the \href{https://www.suu.edu/deanofstudents/}{Dean of Students’ Office}, or the \href{https://www.suu.edu/health/}{Health and Wellness Center}.

\paragraph{Writing Center:}
The SUU Undergraduate Writing Center invites all students to the Writing Center in Braithwaite Center 101 where qualified peer tutors are ready to help with any stage of the writing process. Fall hours start September 3: M-Th 8 am–9 pm, F 8 am–5 pm, and Saturday 11 am–3 pm. All appointments are free, and both in-person and Zoom appointments are available. To schedule, visit our website at \href{https://www.suu.edu/hss/writingcenter}{https://www.suu.edu/hss/writingcenter}.

\paragraph{Land Acknowledgement Statement:}
SUU wishes to acknowledge and honor the Indigenous communities of this region as original possessors, stewards, and inhabitants of this Too’veep (land), and recognize that the University is situated on the traditional homelands of the Nung’wu (Southern Paiute People). We recognize that these lands have deeply rooted spiritual, cultural, and historical significance to the Southern Paiutes. We offer gratitude for the land itself, for the collaborative and resilient nature of the Southern Paiute people, and for the continuous opportunity to study, learn, work, and build community. SUU works towards building meaningful relationships with Native Nations and Indigenous communities through academic pursuits, partnerships, historical recognitions, community service, and student success efforts.

\paragraph{Disclaimer:}
Information contained in this syllabus, other than the grading, late assignments, makeup work and attendance policies, may be subject to change with advance notice, as deemed appropriate by the instructor.
\end{document}
