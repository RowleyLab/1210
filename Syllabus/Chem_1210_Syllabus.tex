\documentclass[12pt, letterpaper]{article}
\usepackage{SyllabusStyle}

\begin{document}
\begin{center}
{\Large \textsc{Principles of Chemistry I}}

CHEM 1210
\end{center}
\begin{center}
{\large Fall 2019}
\end{center}
\begin{center}
	\rule{0.97\textwidth}{0.4pt}
	\begin{tabular}{rlcrl}
		\textbf{Instructor:} & Matthew Rowley & & \textbf{Office Hours:} & M,W,R 10:00 am - 11:00 am \\
		\textbf{Telephone:} & (435) 586-7875 & & & T,F 1:00 pm - 2:00 pm\\
		\textbf{Email:} & \multicolumn{2}{l}{matthewrowley$1$@suu.edu} & \textbf{Office:} & SC-220\\
		\multicolumn{5}{c}{Please include the course number in the subject line of all correspondence.} 
	\end{tabular}
	\rule{0.97\textwidth}{0.4pt}
\end{center}

\section*{Course Description} 
This is an introductory chemistry course designed for students in engineering, physical science, pre-medical, pre-dental, pre-pharmacy, or pre-veterinary medicine. It is for all students who need more than one year of chemistry, and presents the foundational principles on which all other chemistry courses build.

\paragraph{General Education:}
This course is required for beginning Chemistry majors requiring more than one year of chemistry. It can also be taken by other students to fulfill the General Education (GE) requirements for ``Knowledge Area:Physical Science (P).'' The principles of chemistry underlie much of modern society. Students will explore the basic principles of chemistry, allowing them to better understand the world around them at a basic level. This will prepare students to be informed citizens of the modern world in topics such as medicine, materials, energy and pollution.

\paragraph{Prerequisites:}
A proficient understanding of algebra will be required for this course. Either MATH 1050 (College Algebra), or a background of algebra from high school will suffice.

\paragraph{Concurrent requisite:}
CHEM 1215 -- Principles of Chemistry Lab I

\paragraph{Required Course Materials:} ~

Access to \emph{Sapling} online homework and Interactive General Chemistry digital textbook (details below)

\paragraph{Recommended Supplementary Materials:} ~

\emph{Preparing for Your ACS Examination in General Chemistry: The Official Guide} by Eubanks and Eubanks (ISBN: 978-0-970-80420-4)

\paragraph{Student Learning Outcomes:}
\begin{description}
  \item[Knowledge of the Physical and Natural World] Students will recall, interpret, compare, explain, and apply chemistry terminology and theory
  \item[Quantitative Literacy] Students will use chemical equations, graphs, and tables to interpret and communicate chemical information.
  \item[Inquiry and Analysis] Students will solve complex chemical problems.
  \item[Critical Thinking] Students will make decisions based on conceptualizing, applying, and analyzing information from different sources.
\end{description}

\section*{Tentative Schedule}
This class will meet on Mondays, Tuesdays, Wednesdays, and Fridays.
\begin{description}
	\item[Section 04 --] 2:00 pm to 2:50 pm in room 214 of the Science Center (SC)
	\item[Section 05 --] 4:00 pm to 4:50 pm in room 214 of the Science Center (SC)
\end{description}

\noindent For the best lecture experience, read the indicated textbook chapter \emph{before} each lecture.

~

\noindent
\begin{tabular}{rcccc}
	& Date && Topic & Chapter\\
	\midrule
	Week 1 & M, Aug. 26&& Classification and Properties of Matter and Energy & 1.1-1.3\\
	& T, Aug. 27&& The Scientific Method & 1.4\\
	& W, Aug. 28&& Units and Significang Digits & 1.5-1.6\\
	& F, Aug. 30&& Dimensional Analysis, Density, and Temperature & 1.7-1.9\\
	\midrule
	Week 2 & M, Sep. 2& \multicolumn{3}{l}{\textbf{Labor Day - No Class!}}\\
	& T, Sep. 3&& Chemical Symbols and Chemical Combination & 2.1-2.2\\
	& W, Sep. 4&& The History of the Atom and Atomic Structure & 2.3-2.4\\
	& F, Sep. 6&& Atomic Masses and the Periodic Table & 2.5-2.6\\
	\midrule
	Week 3 & M, Sep. 9& \multicolumn{3}{l}{\textbf{Midterm Exam 1 (Ch. 1--2)}}\\
	& T, Sep. 10&& Formulas and Names -- Binary Covalent Compounds & 3.1-3.2\\
	& W, Sep. 11&& Formulas and Names -- Ionic Compounds & 3.3-3.4\\
	& F, Sep. 13&& Naming Acids and Nomenclature Review & 3.5-3.6\\
\end{tabular}

\noindent\hspace{-2.5em}
\begin{tabular}{rcccc}
	& Date && Topic & Chapter\\
	\midrule
	Week 4 & M, Sep. 16&& The Mole and Molar Mass & 3.7-3.8\\
	& T, Sep. 17&& Percent Composition and Empirical Formulas & 3.9-3.10\\
	& W, Sep. 18&& Molecular Formulas and Combustion Analysis & 3.11-3.12\\
	& F, Sep. 20&& Chemical Equations and Reactions & 4.1-4.2\\
	\midrule
	Week 5 & M, Sep. 23&& Compounds in Aqueous Solution and Precipitation Reactions & 4.3-4.4\\
	& T, Sep. 24&& Acid-Base Reactions & 4.5\\
	& W, Sep. 25&& Oxidation States and Reox Reactions & 4.6-4.7\\
	& F, Sep. 27&& Calculations for Chemical Reactions & 5.1-5.2\\
	\midrule
	Week 6 & M, Sep. 30&& Limiting Quantities and Yields & 5.3-5.4\\
	& T, Oct. 1&& Definition and Uses of Molarity & 5.5-5.6\\
	& W, Oct. 2&& Calculations Involving Other Quantities & 5.7\\
	& F, Oct. 4&& Calculations with Net Ionic Equations & 5.8\\
	\midrule
	Week 7 & M, Oct. 7&& Titration & 5.9\\
	& T, Oct. 8& \multicolumn{3}{l}{\textbf{Midterm Exam 2 (Ch. 3--5)}}\\
	& W, Oct. 9&& Energy, Heat, and Work & 6.1-6.3\\
	& F, Oct. 11&& Enthalpy and Specific Heat & 6.4-6.5\\
	\midrule
	Week 8 & M, Oct. 14&& Calorimetry: Measuring Energy Changes & 6.6\\
	& T, Oct. 15&& Enthalpy in Chemical Reactions & 6.7\\
	& W, Oct. 16&& Standard Enthalpies of Formation & 6.8\\
	& F, Oct. 18&& Light and the Bohr Model of the Atom & 8.1-8.2\\
	\midrule
	Week 9 & M, Oct. 21& \multicolumn{3}{l}{\textbf{Fall Break - No Class!}}\\
	& T, Oct. 22& \multicolumn{3}{l}{\textbf{Fall Break - No Class!}}\\
	& W, Oct. 23&& Electron Shells, Subshells, and Orbitals & 8.3\\
	& F, Oct. 25&& Energy-Level Diagrams and Electron Configurations & 8.4-8.5\\
\end{tabular}

\noindent\hspace{-1.5em}
\begin{tabular}{rcccc}
	& Date && Topic & Chapter\\
	\midrule
	Week 10 & M, Oct. 28&& Quantum Numbers & 8.6\\
	& T, Oct. 29&& Valence Electrons and Atomic/Ionic Sizes & 9.1-9.2\\
	& W, Oct. 30&& Ionization Energy and Electron Affinity & 9.3\\
	& F, Nov. 1&& Ionic Bonding and Lattice Energy & 9.4-9.5\\
	\midrule
	Week 11 & M, Nov. 4& \multicolumn{3}{l}{\textbf{Midterm Exam 3 (Ch. 6, 8, 9)}}\\
	& T, Nov. 5&& Formation of Covalent Bonds & 10.1\\
	& W, Nov. 6&& Lewis Structures & 10.2\\
	& F, Nov. 8&& Resonance and Formal Charges & 10.3\\
	\midrule
	Week 12 & M, Nov. 11&& Exceptions to the Octet Rule & 10.4\\
	& T, Nov. 12&& Polar Bonds and Bond Enthalpy & 10.5-10.6\\
	& W, Nov. 13&& VSEPR and Molecular Geometry & 11.1\\
	& F, Nov. 15&& Polar and Nonpolar Molecules & 11.2\\
	\midrule
	Week 13 & M, Nov. 18&& Valence Bond Theory & 11.3\\
	& T, Nov. 19&& Using Valence Bond Theory & 11.4\\
	& W, Nov. 20&& Molecular Orbital Theory & 11.5\\
	& F, Nov. 22&& Gas Pressure and Simple Gas Laws & 7.1-7.3\\
	\midrule
	Week 14 & M, Nov. 25&& The Combined Gas Law and the Ideal Gas Law & 7.4-7.6\\
	& T, Nov. 26&& Partial Pressures, Molar Mass, and Density of gases & 7.7-7.8\\
	& W, Nov. 27& \multicolumn{3}{l}{\textbf{Thanksgiving Break - No Class!}}\\
	& F, Nov. 29& \multicolumn{3}{l}{\textbf{Thanksgiving Break - No Class!}}\\
	\midrule
	Week 15 & M, Dec. 2&& Gas Reactions and the Kinetic Molecular Theory & 7.9-7.10\\
	& T, Dec. 3&& Movement of Gas Particles & 7.11\\
	& W, Dec. 4&& Behavior of Real Gases & 7.12\\
	& F, Dec. 6& \multicolumn{3}{l}{\textbf{Midterm Exam 4 (Ch. 10, 11, 7)}}\\
	\midrule
	\midrule
	Finals Week & W, Dec. 11& \multicolumn{3}{l}{\textbf{Section 05 Final Exam 3:00--4:50 pm: Bring a pencil and scantron}}\\
	& F, Dec. 13& \multicolumn{3}{l}{\textbf{Section 04 Final Exam 1:00--2:50 pm: Bring a pencil and scantron}}\\
\end{tabular}

\section*{Course Requirements}
Grades will be based on the following items:
\begin{description}
  \item[4 Midterm Exams] 40\%
  \item[Final Exam] 20\%
  \item[In-Class Quizzes] 10\%
  \item[Homework] 30\%
\end{description}
Final Grades will be assigned according to the following scale:

\begin{tabular}{rl|c|rl}
	Percentage & Grade &  & Percentage & Grade \\ \midrule
	  93.0-100 & A     &  &  73.0-77.0 & C     \\
	 90.0-93.0 & A-    &  &  70.0-73.0 & C-    \\
	 87.0-90.0 & B+    &  &  67.0-70.0 & D+    \\
	 83.0-87.0 & B     &  &  63.0-67.0 & D     \\
	 80.0-83.0 & B-    &  &  60.0-63.0 & D-    \\
	 77.0-80.0 & C+    &  &     < 60.0 & F
\end{tabular}
\paragraph{Midterm Exams:}
There are four midterm exams, to be completed in class on the designated day unless prior arrangements have been made. It is departmental policy that exams not be returned, although students can examine the questions in my office after they have been graded.

\paragraph{Final Exam:}
The final exam is comprehensive. The final is produced by the American Chemical Society, and the instructor will not have access to the exam prior to its administration. Therefore, it is to your advantage to learn as much as possible throughout the semester. The test is multiple choice and a 100-question scantron will be required.

\paragraph{Chapter Homework:}
Homework sets are done through the ``Sapling'' online homework system, though links to all the assignments are found in Canvas. Each assignment is worth 100 points, and longer or more important chapters have been split into two assignments.

\noindent To register for access to the Sapling homework problems and the interactive digital textbook for this course:
\begin{enumerate}
	
  \item Log into our course on Canvas
  \item In the Course Materials module, select the ``Student Registration - Start Here'' item and follow the displayed link to the Sapling website
  \item Enter your first name, last name, and email address to create a Learning Profile (user account)
  \item Indicate whether you agree to the end-user agreement for the Sapling site
  \item You should now be on a checkout page
  \begin{enumerate}
  	\item Check that you have the correct course listed near the top. It should be:
  	
  	Southern Utah University - Chem 1210 - Fall19 (Section 0x) - ROWLEY
  	\item Select a payment option and provide the relevant payment details
  \end{enumerate}
  \item You should now have access to both the online homework, and the digital interactive textbook
\end{enumerate}
 
 \noindent Please note that you may need to adjust your browser settings and/or disable pop-up blockers to access Sapling content
 
\paragraph{Quizzes:}
Quizzes will be given either in class or as a ``take-home'' quiz. The purpose of these quizzes is to gauge the general level of subject mastery in the class as well as provide format practice for the exams

\paragraph{Attendance Policy:}
Students are expected to attend class. If you must miss class, contact the instructor

\paragraph{Late Work Policy:}
Homework and take-home quizzes will be due on a day when class is regularly scheduled. All work is to be turned in at the \emph{beginning} of the class period, and late work will not be accepted. Please note that online homework has due-dates at 11:55 pm, and will not be available after that time

\paragraph{Make-up Work Policy:}
In general, there will be no opportunity to make up missed work, including in-class quizzes. If you must miss class, please do any assigned work in advance, and arrange to turn it in early

\section*{Miscellany}

\paragraph{Scientific Calculator:}
There are many different ways to calculate figures during homework. It is tempting to rely on Online resources such as \href{http://www.wolframalpha.com}{http://www.wolframalpha.com}, or to simply use a calculator application on a smart phone. During exams, however, any devices capable of connecting to the Internet will \emph{not} be allowed. You will instead need a scientific calculator capable of performing exponentiation and logarithms for the exams. Using such a calculator exclusively while doing homework will ensure that you are familiar with it for use during exams.

\paragraph{Academic Integrity:}
Scholastic dishonesty will not be tolerated and will be prosecuted to the fullest extent. You are expected to have read and understood the current issue of the student handbook (published by Student Services) regarding student responsibilities and rights, the intellectual property policy, and for information about procedures and what constitutes acceptable on-campus behavior. In short any time a student presents the words, results, analysis, or conclusions of another without proper attribution will be considered a violation of academic integrity.

\paragraph{ADA Policy:}
Students with medical, psychological, learning, or other disabilities desiring academic adjustments, accommodations, or auxiliary aids will need to contact the Southern Utah University Coordinator of Services for Students with Disabilities (SSD), in Room 206F of the Sharwan Smith Center or phone (435) 865-8022. SSD determines eligibility for and authorizes the provision of services.

\paragraph{Emergency Management Statement:}
In case of emergency, the university's Emergency Notification System (ENS) will be activated. Students are encouraged to maintain updated contact information using the link on the homepage of the \emph{mySUU} portal. In addition, students are encouraged to familiarize themselves with the Emergency Response Protocols posted in each classroom. Detailed information about the university's emergency management plan can be found at: \href{http://www.suu.edu/emergency}{http://www.suu.edu/emergency}

\paragraph{HEOA Compliance Statement:}
The sharing of copyrighted material through peer-to- peer (P2P) file sharing, except as provided under U.S. copyright law, is prohibited by law. Detailed information can be found at: \href{https://help.suu.edu/article/1096/heoa-compliance-plan}{https://help.suu.edu/article/1096/heoa-compliance-plan}

\paragraph{Disclaimer:}
Information contained in this syllabus, other than the grading, late assignments, make up work and attendance policies, may be subject to change with advance notice, as deemed appropriate by the instructor.

\end{document}