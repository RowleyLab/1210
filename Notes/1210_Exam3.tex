\documentclass[12pt, letterpaper]{memoir}
\usepackage{ExamStyle}

\begin{document}
	\mainmatter
	
	\begin{center}
		\vspace{12em}
		{\Huge CHEM 1210}
		{\LARGE-- Fall 2018
		
		Midterm Exam 3}
		
		\vspace{20em}
		
		\begin{tabular}{c|c}
			\multicolumn{2}{c}{Useful Constants} \\ \midrule
			$c$ & $2.998\times10^8~\frac{m}{s}$ \\
			$h$ & $6.626\times10^{-34}~Js$ \\
			$R_H$ & $-2.18\times10^{-18}~J$ \\
			$m_e$ & $9.109\times10^{-31}~kg$
		\end{tabular}
	
		\vspace{15em}
		Feel free to use this page as scratch paper, but work for questions in the free response\\ section must be shown \emph{in that section} to count for credit.

	\end{center}

	\newpage
	
	\includepdf[pages={1},angle=90, width=1.08\textwidth]{Table.pdf}
	
	\newpage
		
	{\Large Name \rule[-1mm]{4in}{.1pt}}
	\section*{Multiple Choice -- 3 points each -- {\large More than one answer may be correct!}}
	
	\subsection*{Problem 1}
	Give the ground state electronic configuration of antimony (\ch{Sb}).
	
	\begin{tabular}{llll}
		A) \ch{[Xe]} $6s^2 ~5d^{10} ~6p^3$ & B) \ch{[Ar]} $4s^2 ~4d^{10} ~4p^3$ & C) \ch{[Kr]} $5s^2 ~5d^{10} ~5p^3$& D) \ch{[Xe]} $5s^2 ~4d^{10} ~5p^3$\\
		E) \ch{[Ar]} $5s^2 ~4d^{13}$& F) \ch{[Kr]} $5s^2 ~4d^{13}$ & G) \ch{[Kr]} $6s^2 ~5d^{10} ~6p^2$ & H) \ch{[Kr]} $5s^2 ~4d^{10} ~5p^3$
	\end{tabular}	
	
	\subsection*{Problem 2}
	Copper has $C_s=0.385~\nicefrac{J}{(g~ ^\circ C)}$, and iron has $C_s=0.450~\nicefrac{J}{(g~ ^\circ C)}$. Two $10.0~g$ blocks of \ch{Cu} and \ch{Fe} start at $20.0^\circ C$ and are each given $200.0~J$ of heat. If the blocks are then put in thermal contact with each other, which direction (if any) will heat flow?
		
	{\large From Copper to Iron} \hspace{2em} {\large From Iron to Copper} \hspace{2em} {\large No Heat is Transferred}
		
	\subsection*{Problem 2}
	For degenerate electron orbitals, electrons should be placed with the same spin in empty orbitals before they are paired up in a filled orbital.
	
	{\large True} \hspace{2em} {\large False}
		
	\subsection*{Problem 3}
	Which of the following are \emph{isoelectronic} with \ch{F^-}?
	
	\begin{tabular}{llll}
		A) \ch{O-} & B) \ch{O^2-} & C) \ch{Ne+} & D) \ch{Cl-}\\
		E) \ch{C^4+} & F) \ch{Mg^2+} & G) \ch{Na-} & H) \ch{Ne}
	\end{tabular}	
	
	\subsection*{Problem 4}	
	Solid aluminum has a specific heat of $0.900~\nicefrac{J}{(g~ ^\circ C)}$, while liquid water has a specific heat of $4.184~\nicefrac{J}{(g~ ^\circ C)}$. If a $12 g$ mass of aluminum is heated to $300 ^\circ C$, and dropped into $50 ml$ of water at $20 ^\circ C$. What will be the final equilibrium temperature?

	\begin{tabular}{llll}
		A) $160. ^\circ C$& B) $68.4 ^\circ C$& C) $74.2 ^\circ C$& D) $25.2 ^\circ C$\\
		E) $132 ^\circ C$& F) $33.7 ^\circ C$& G) $53.7 ^\circ C$& H) $34.5 ^\circ C$
	\end{tabular}
	% 33.7
	
	\subsection*{Problem 4}
	How many angular nodes are present in a 3-d orbital?

	\begin{tabular}{llll}
		A) $0$ & B) $1$ & C) $2$ & D) $3$\\
		E) $4$ & F) $5$ & G) $6$ & H) $7$
	\end{tabular}	

	\subsection*{Problem 5}
	What is the maximum number of electrons that can be placed in the 5-f subshell?
	
	\begin{tabular}{llll}
		A) $2$ & B) $10$ & C) $14$ & D) $16$\\
		E) $20$ & F) $12$ & G) $7$ & H) $5$
	\end{tabular}		
	
	\newpage
	\subsection*{Problem 6}
	Which properties generally \emph{decrease} in magnitude across a row in the periodic table?
	
	\begin{tabular}{llll}
		A) $Z_{eff}$ & B) First Ionization Energy & C) Electron Affinity \\
		D) Atomic Radius & E) Second Ionization Energy & F) Electronegativity 
	\end{tabular}
	
	\subsection*{Problem 7}
	Which element has the greatest ionization energy?
	
	\begin{tabular}{llll}
		A) \ch{Mg} & B) \ch{Na} & C) \ch{K} & D) \ch{Ca}\\
		E) \ch{O} & F) \ch{F} & G) \ch{Br} & H) \ch{S}
	\end{tabular}	
	
	\subsection*{Problem 8}
	Which of the following compounds will exhibit the largest lattice energy?
	
	\begin{tabular}{llll}
		A) \ch{NaI} & B) \ch{KCl} & C) \ch{LiF}& D) \ch{RbI}\\
		E) \ch{KBr}& F) \ch{LiCl}& G) \ch{NaF}& H) \ch{RbBr}
	\end{tabular}
	
	\subsection*{Problem 9}
	Mercury vapor lamps are sometimes used to calibrate spectrometers because they have sharp electronic transition lines. One line corresponds to a transition energy of $4.556\times10^{-19} J$. What is the wavelength of that transition line?

	\begin{tabular}{llll}
		A) $436~m$ & B) $6.876\times 10^{5}~Hz$ & C)  $436~nm$ & D) $9.05~\mu m$\\
		E) $6.87~nm$ & F) $6.876\times10^{14}~Hz$ & G) $9.05 \times10^-6~Hz$& H) $456~nm$
	\end{tabular}	
	
	\subsection*{Problem 10}
	How many valence electrons are there in the phosphate ion (\ch{PO4^{3-}})?
	
	\begin{tabular}{llll}
		A) 35 & B) 28 & C) 34 & D) 32\\
		E) 36 & F) 50 & G) 47 & H) 30
	\end{tabular}	
	
%	{\large True} \hspace{2em} {\large False}

%	\begin{tabular}{llll}
%		A) & B) & C) & D) \\
%		E) & F) & G) & H)
%	\end{tabular}	

%	\begin{tabular}{lllll}
%		A) & B) & C) & D) & E) \\
%		F) & G) & H) & I) & J)
%	\end{tabular}	
	
	\newpage	
	\section*{Free Response -- Work \emph{must} be shown for full credit}
	
	
	\subsection*{Problem 10 -- 20 Points}
	A bomb calorimeter has a calibrated heat capacity of $C_{cal} = 6.724~\nicefrac{kJ}{^\circ C}$. $0.638 g$ of naphthalene (\ch{C10H8}) is combusted in the calorimeter, producing a temperature change of $3.82 ^\circ C$. 
	
	\vspace {1em}
	\noindent What is the balanced chemical reaction for this combustion?
	
	\vspace{8em}
	\noindent What is the molar enthalpy of reaction for the combustion of naphthalene?
	%5155%
	
	\vspace{16em}
	\noindent Given that $\Delta H_f^\circ$ for \ch{H2O (g)} is $-241.8~\nicefrac{kJ}{mol}$, and $\Delta H_f^\circ$ for \ch{CO2 (g)} is $-393.5~\nicefrac{kJ}{mol}$, find $\Delta H_f^\circ$ for naphthalene.
	
	\subsection*{Problem 11 -- 15 Points}
	Consider the following reactions:
	\begin{align*}
		\ch{2 NH3 -> N2 + 3 H2} & \hspace{4em}\Delta H_{rxn} = 92.0~\nicefrac{kJ}{mol} &\hspace{2em}\circled{A}\\
		\ch{2 H2O -> 2 H2 + O2} & \hspace{4em}\Delta H_{rxn} = 572.0~\nicefrac{kJ}{mol} &\hspace{2em}\circled{B}\\
		\ch{N2 + O2 -> 2 NO} & \hspace{4em}\Delta H_{rxn} = -180.0~\nicefrac{kJ}{mol} &\hspace{2em}\circled{C}\\
	\end{align*}
	Referring to chemical reactions \circled{A}, \circled{B}, and \circled{C}, how can you reproduce the reaction:
	\begin{align*}
		\ch{4 NH3 + 5 O2 -> 4 NO + 6 H2O} & \hspace{4em}\hphantom{\Delta H_{rxn} = -25.4 \frac{kJ}{mol}}& \hspace{-1em}\circled{$\star$}
	\end{align*}			
	
	\vspace{18em}
	\noindent What is the $\Delta H_{rxn}$ for reaction \circled{$\star$}?
	
	\subsection*{Problem 11 -- 10 Points}
		
	\vspace{1em}
	\begin{mdframed}
		\centering
		Table of Average Bond Enthalpies
		\hrule
		\begin{tabular}{cc|cc}
			Bond Type & Bond Enthalpy ($\nicefrac{kJ}{mol}$) & Bond Type & Bond Enthalpy ($\nicefrac{kJ}{mol}$) \\ \midrule
			C--H & 413 & O--H & 391\\
			C--C & 348 & O=O & 495 \\
			C=O & 799 & C--O & 358
		\end{tabular}
	\end{mdframed}
	
	\noindent
	Using the table above, estimate the molar enthalpy of reaction ($\Delta H_{rxn}$) for the combustion of ethanol:	
	\begin{equation*}
		\ch{CH3CH2OH + 3 O2 -> 3 H2O + 2 CO2}
	\end{equation*}

	\newpage
		
	\subsection*{Problem 12 -- 20 Points}
	\begin{itemize}
		\item 	\ch{H2SO3}, sulfamic acid, is a weak acid. After it donates both hydrogens, sulfamic acid becomes the sulfite ion, \ch{SO3^{2-}}. Draw the Lewis dot structure for the \ch{SO3^{2-}} ion. Include, if appropriate, all possible resonance structures. Draw the structures large enough to add annotations from later parts of this problem.
		
		\vspace{38em}
		\item Write the formal charges to the upper-right of each atom. For example, a carbon atom with formal charge of 1- would be written as \ch{C^{1-}}. If there are multiple resonance structures, you need only to mark the formal charges on one of them.
		\item Write the oxidation states to the upper-left of each atom. For example, if the same carbon has an oxidation state of +3, then it would be written as \ch{^{+3}C^{1-}}. If there are multiple resonance structures, you need only to mark the oxidation states on one of them.
	\end{itemize}
	
	\newpage
	\subsection*{Problem 13 -- 10 Points}
	
	$\circ$ Calculate the transition energy for the $5\rightarrow2$ energy level transition in a hydrogen atom. Remember that for a hydrogen atom, $R_H = -2.18\times10^{-18}~J$
	
	\vspace{22em}
	\noindent $\circ$ Calculate the transition energy for the $3\rightarrow4$ energy level transition in a hydrogen atom. Remember that for a hydrogen atom, $R_H = -2.18\times10^{-18}~J$
	
	\newpage
	\subsection*{Problem 14 -- 10 Points}
	
	An electron has a rest mass of: $m_e=9.109\times10^{-31}~kg$. The speed of light is: $c=2.998\times10^8~\frac{m}{s}$. Find the characteristic de Broglie wavelength for an electron traveling at $15\%$ of the speed of light. You may need Planck's constant: $h= 6.626\times10^{-34}~Js$ and the equation: $J = \frac{kgm^2}{s^2}$
	
	\vspace{12em}
	\subsection*{Problem 15 -- 20 Points}	
	~
	
	\hspace{-5em}
	\begin{minipage}[t]{0.7\linewidth}
		\begin{center}
			\begin{tabular}{ccccccccccc}
				\rule[-1pt]{1.15em}{0.5pt} &&\rule[-1pt]{1.15em}{0.5pt}&\rule[-1pt]{1.15em}{0.5pt}&\rule[-1pt]{1.15em}{0.5pt} && \rule[-1pt]{1.15em}{0.5pt}&\rule[-1pt]{1.15em}{0.5pt}&\rule[-1pt]{1.15em}{0.5pt}&\rule[-1pt]{1.15em}{0.5pt}&\rule[-1pt]{1.15em}{0.5pt}\\
				{$3s$} &&&{$3p$} &&&&&{$3d$}
			\end{tabular}
		
			~
		
			\begin{tabular}{ccccc}
				\rule[-1pt]{1.15em}{0.5pt} &&\rule[-1pt]{1.15em}{0.5pt}&\rule[-1pt]{1.15em}{0.5pt}&\rule[-1pt]{1.15em}{0.5pt}\\
				{$2s$} &&&{$2p$}
			\end{tabular}
			
			~
			
			\begin{tabular}{c}
				\rule[-1pt]{1.15em}{0.5pt} \\
				{$1s$}
			\end{tabular}
		\end{center}
	\end{minipage}
	\begin{minipage}[t]{0.45\linewidth}
		Give the four quantum numbers for the electron shown on the diagram to the left.
		
		\vspace{5em}
		On the diagram to the left, draw an electron with the following quantum numbers:
		
		$n=3$, ~$l=1$, ~$m_l=0$, ~$m_s=\frac{1}{2}$
	\end{minipage}

	\noindent
	Sketch how the orbitals for both of the electrons discussed above might look. Be sure to include the proper shape and number of nodes.
\end{document}

%A bomb calorimeter has a calibrated heat capacity of $C_{cal} = 2.75~\nicefrac{kJ}{^\circ C}$. $1.52 g$ of oxalic acid (\ch{C2H2O4}) is combusted in the calorimeter, producing a temperature change of $1.23 ^\circ C$. 