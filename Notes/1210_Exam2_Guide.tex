\documentclass[12pt, letterpaper]{memoir}
\usepackage{ExamStyle}

\begin{document}
	\mainmatter
	
	\begin{center}
		{\Huge CHEM 1210}
		{\LARGE-- Fall 2017
		
		Midterm Exam 2 Study Guide (Ch. 3-5)}
	\end{center}
	
	This study guide is meant to provide only the barest direction as you study. Try to find practice problems from the textbook (both in the chapter text and in the end-of-chapter questions) rather than just relying on this guide. Note that tables and equations provided here will {\large\bfseries NOT} necessarily be given on the exam. As the exam gets closer (i.e. after I write it) I will provide you with an example of the tables and equations (if any) which will be provided. 
	
	\subsection*{Previous Chapters}
		{\hspace{-2em}\centering$\vcenter{\hbox{\includegraphics[width=0.55\textwidth]{../../BLMB_Resources/Ch02/02_04_Table}}}\vcenter{\hbox{\includegraphics[width=0.55\textwidth]{../../BLMB_Resources/Ch02/02_05_Table}}}$}
	\subsection*{Chapter 3 -- Stoichiometry}
	\begin{itemize}
		\item Balance Chemical Reactions
		\item Find and Use Molar Masses (a.k.a. Formula Weights): $M_{\textbf{compound}} = \sum M_{\textbf{atoms}}$
		\item Weight \% (i.e. what is the weight \% of copper in \ch{CuSO4}?)
		\item Combustion Analysis Problems
		\item Complete Reaction Problems
		\item Limiting Reactant Problems
		\begin{itemize}
			\item Identify Limiting Reactant
			\item Find the amount of Product
			\item How much excess reactant is left over
		\end{itemize}
		\item \% Yield
	\end{itemize}
	\subsection*{Chapter 4 -- Aqueous Reactions}
	\begin{itemize}
		\item Stong, Weak, and Non-Electrolytes
		\item Solubility Rules and Precipitation Reactions
		\item Balancing Net Ionic Equations	
		\item Acid and Base Definitions
		\item Recognizing Strong Acids and Bases
		\item Recognizing Weak Acids and Bases
		\item Multiprotic Acids and Bases
		\item Acid/Base Reactions
		\item Find Oxidation Numbers
		\item Identify What is Being Reduced/Oxidized
		\item Predict Redox Reactions Based on Electrochemical Series (Table 4.5)
		\item Concentrations
		\begin{itemize}
			\item Preparing Solutions
			\item Dilutions: $M_1V_1 = M_2V_2$
			\item Titrations: $\nu_2M_1V_1 = \nu_1M_2V_2$ \hspace{2em} Where $\nu$ is the stoichiometric coefficient
		\end{itemize}
	\end{itemize}
	{\hspace{-2em}\centering$\vcenter{\hbox{\includegraphics[width=0.55\textwidth]{../../BLMB_Resources/Ch04/04_01_Table}}}\vcenter{\hbox{\includegraphics[width=0.55\textwidth]{../../BLMB_Resources/Ch04/04_02_Table}}}$}
	\subsection*{Chapter 5 -- Thermodynamics}
	\begin{itemize}
		\item System and Surroundings
		\item Open, Closed, and Isolated Systems
		\item Work: $w=fd$
		\item Heat Transfer
		\begin{itemize}
			\item For Heating: $q=C_sm\Delta T$
			\item For (Constant Pressure) Coffee-Cup Calorimetry: $q=C_sm\Delta T = - n_{\text{moles}}\Delta H_{rxn}$
			\item For (Constant Volume) Bomb Calorimetry: $q=C\Delta T=-n_{\text{moles}}\Delta H_{rxn}$
			\item For Phase Changes: $q=n_{moles}\Delta H_{\text{fus or vap}}$
			\item For Heat Transfer from 1 to 2: $q_1=-q_2$	
		\end{itemize}
		\item Thermal Equilibrium
		\item Enthalpy -- Path vs. State Functions
		\item Heats of Reaction: $\Delta H_{rxn} = \dfrac{-q_{\text{surroundings}}}{n_{\text{moles}}}$
		\item Hess's Law
		\item Formation Reactions
		\item $\Delta H_{rxn} = \sum C_i\Delta H_{f-i}^\circ \text{(products)} - \sum C_j\Delta H_{f-j}^\circ \text{(reactants)}$
	\end{itemize}
\end{document}