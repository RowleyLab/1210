\documentclass[12pt, letterpaper]{memoir}
\usepackage{ExamStyle}

\begin{document}
	\mainmatter
	
	\begin{center}
		\vspace{12em}
		{\Huge CHEM 1210}
		{\LARGE-- Fall 2018
		
		Midterm Exam 4}

		\vspace{47em}
		Feel free to use this page as scratch paper, but work for questions in the free response\\ section must be shown \emph{in that section} to count for credit.

	\end{center}

	\newpage
	
	\includepdf[pages={1},angle=90, width=1.08\textwidth]{Table.pdf}
	
	\newpage
		
	{\Large Name \rule[-1mm]{4in}{.1pt}}
	\section*{Multiple Choice -- 3 points each -- {\large More than one answer may be correct!}}
	
	\subsection*{Problem 1}
	Which of the following molecules are capable of participating in hydrogen bonds?
	
	\begin{tabular}{llll}
		A) \ch{He} & B) \ch{NH4^{+}} & C) \ch{CH3OH} & D) \ch{O2}\\
		E) \ch{CH4} & F) \ch{NH3} & G) \ch{C2H6} & H) \ch{Zn^{2+}}
	\end{tabular}
	
	\subsection*{Problem 2}
	Which of the following molecules is a \emph{polar} molecule?
	
	\begin{tabular}{llll}
		A) \ch{CH4} & B) \ch{PCl5} & C) \ch{XeF4} & D) \ch{SF6}\\
		E) \ch{H2O}& F) \ch{CO2} & G) \ch{CH2Cl2} & H) \ch{BF3}
	\end{tabular}
	
		
	\subsection*{Problem 3}
	What is the molecular geometry of a molecule whose central atom bonds with four different atoms and has one lone pair?
	
	\begin{tabular}{llll}
		A) See-Saw & B) Tetrahedral & C) T-Shaped & D) Trigonal Planar\\
		E) Square Pyramidal & F) Trigonal Bipyramidal & G) Bent & H) Square Planar 
	\end{tabular}	
	
	\subsection*{Problem 4}
	What type of hybridization be exhibited by a molecule whose central atom has 3 electron domains surrounding it?

	\begin{tabular}{lll}
		A) $sp$ & B) $sp^2$ & C) $sp^3$ \\
		D) $sp^3d$ & E) $sp^3d^2$ & F) $sp^4$  
	\end{tabular}	

	\subsection*{Problem 5}
	π bonds are created by the overlap of hybridized atomic orbitals.
	
	{\large True} \hspace{2em} {\large False}
	
	\newpage
	\subsection*{Problem 6}
	What will be a the total pressure of an ideal gas mixture which contains $0.7~atm ~\ch{N2}$, $0.25~atm ~\ch{He}$, $0.1~atm ~\ch{NH3}$, and $0.5~atm ~\ch{O2}$?
	
	\begin{tabular}{llll}
		A) $1.25~atm$ & B) $0.75~atm$ & C) $1.55~atm$ & D) $2~atm$\\
		E) $1~atm$ & F) $2.2~atm$ & G) $1.35~atm$& H) $1.4~atm$		
	\end{tabular}	
	
	\subsection*{Problem 7}
	When using the ideal gas law, temperature must be expressed in units of Kelvin, rather than $^\circ$C or $^\circ$F.
		
	{\large True} \hspace{2em} {\large False}
	
	\subsection*{Problem 8}
	Which molecule would exhibit the strongest intermolecular interactions?

	\begin{tabular}{llll}
		A) \ch{CO2} & B) \ch{He} & C) \ch{CH2O} & D) \ch{CCl4}\\
		E) \ch{N2}& F) \ch{N2O} & G) \ch{O3} & H) \ch{NH3}
	\end{tabular}
		
	\subsection*{Problem 9}
	A balloon is filled to a volume of $2.5~L$ with gas at $0.85~atm$ and $17 ^\circ C$. It is then released and ascends into the atmosphere until it reaches a steady altitude. Instruments carried by the balloon communicate with the ground via radio, giving the temperature as $-37 ^\circ C$, and a pressure of $0.42~atm$. Assuming the balloon retains all of its gas, what is the new volume?
	
	\begin{tabular}{llll}
		A) $5.1~L$ & B) $1.5~L$ & C) $11~L$ & D) $4.1~L$\\
		E) $2.0~L$ & F) $5.4~L$ & G) $6.2~L$ & H) $1.0~L$
	\end{tabular}
		
	\subsection*{Problem 10}
	Which of the following gases would have the highest rms velocity, assuming all are at the same temperature?

	\begin{tabular}{llll}
		A) \ch{CO2} & B) \ch{He} & C) \ch{CH2O} & D) \ch{CCl4}\\
		E) \ch{N2}& F) \ch{N2O} & G) \ch{O3} & H) \ch{NH3}
	\end{tabular}	
%	\begin{tabular}{llll}
%		A) & B) & C) & D) \\
%		E) & F) & G) & H)
%	\end{tabular}

%	{\large True} \hspace{2em} {\large False}

%	\begin{tabular}{llll}
%		A) & B) & C) & D) \\
%		E) & F) & G) & H)
%	\end{tabular}	

%	\begin{tabular}{llll}
%		A) \ch{} & B) \ch{} & C) \ch{} & D) \ch{}\\
%		E) \ch{}& F) \ch{} & G) \ch{} & H) \ch{}
%	\end{tabular}

%	\begin{tabular}{lllll}
%		A) & B) & C) & D) & E) \\
%		F) & G) & H) & I) & J)
%	\end{tabular}	
	
	\newpage	
	\section*{Free Response -- Work \emph{must} be shown for full credit}
	\subsection*{Problem 11 -- 20 Points}
	For each molecule or ion, do the following:
	
	$\circ$ Draw the Lewis dot structure.
	
	$\circ$ Give the electron domain geometry.
	
	$\circ$ Give the molecular geometry.
	
	$\circ$ Give the hybridization on the central atom.
	
	$\circ$ State whether the molecule or ion is polar or non-polar.
	
	\noindent\ch{NH4^+}
	
	\vspace{8em}
	
	\noindent\ch{CO3^{2-}}
	
	\vspace{8em}
	
	\noindent\ch{BrF5}
	
	\vspace{8em}
			
	\noindent\ch{SF3^+}
	\newpage
		
	\subsection*{Problem 12 -- 15 Points}
	\begin{wrapfigure}[4]{l}[1em]{0.3\textwidth}
		{\includegraphics[width=0.3\textwidth]{butadiene}}
	\end{wrapfigure}
	
	~
	
	\vspace{2em}
	
	\noindent Consider the molecule 1,3-butadiene (shown at left) from a perspective of VSEPR and hybrid orbitals. 
		
	~
	
	\vspace{2em}
	\noindent$\circ$ Draw the electron orbitals which form the bond(s) between carbons 1 and 2. Do the same for the bond(s) between carbons 2 and 3.
	
	\noindent$\circ$ Label which types of orbitals you drew, and which types of bonds formed for both pairs of carbons.
	
	\noindent$\circ$ Give the predicted bond order for both pairs of carbons based on these orbitals.

	\newpage
	\subsection*{Problem 13 -- 15 Points}
	$\circ$ Draw the molecular orbital diagram for a \ch{F2} molecule. This diagram should show how the valence orbitals on each F atom combine to form molecular orbitals.
	
	\noindent
	$\circ$ Label the molecular orbitals according to their type (σ, σ$^*$, π, or π$^*$).
	
	\noindent
	$\circ$ Fill in the correct number of valence electrons, and predict whether this molecule is diamagnetic or paramagnetic.
	
	\newpage
	\subsection{Problem 14 -- 20 Points}
	\noindent
	$\circ$ Consider a gas mixture that contains both \ch{Cl2} and \ch{N2} gas. If the gases have a temperature of $T=298~K$, then what are the $\mu_{RMS}$ values for both gases?
	
	\vspace{7em}
	\noindent $\circ$ This gas mixture is placed in a balloon with a small hole in it. As the gases leak out of the small hole, what will happen to the composition of the mixture in the balloon?
	
	\vspace{5em}
	\noindent $\circ$ Which of these two gases would you expect to behave \emph{less} like an ideal gas, and why?
	
	\vspace{5em}
	\begin{tabular}{cc|c}
		& Van der Waals a constant $\left(\dfrac{L^2~atm}{mol^2}\right)$ & Van der Waals b constant $\left(\dfrac{L}{mol}\right)$\\ \midrule
		Nitrogen gas & $1.39$ & $0.0391$\\
		Chlorine gas & $6.49$ & $0.0562$\\		
	\end{tabular}

	\noindent $\circ$ Using the table above, prove that your answer above is correct. Assume one mole of each gas is compressed at $T=298~K$ to fill a volume of only $V=1.50~L$. Find the pressures for each gas using the Van der Waals equation, and compare them to the pressure of an ideal gas.
\end{document}