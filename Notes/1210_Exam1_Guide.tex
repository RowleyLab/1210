\documentclass[12pt, letterpaper]{memoir}
\usepackage{ExamStyle}

\begin{document}
	\mainmatter
	
	\begin{center}
		{\Huge CHEM 1210}
		{\LARGE-- Fall 2018
		
		Midterm Exam 1 Study Guide (Ch. 1-2)}
	\end{center}
	
	This study guide is meant to provide only the barest direction as you study. Try to find practice problems from the textbook (both in the chapter text and in the end-of-chapter questions) rather than just relying on this guide. Note that tables and equations will not be provided here. You can find them in your textbook. As the exam gets closer (i.e. after I write it) I will provide you with an example of the tables and equations (if any) which will be provided.	
	

	\subsection*{Chapter 1 -- Introduction: Matter, Energy, and Measurement}
	\begin{itemize}
		\item Elements, compounds, homogeneous mixtures, and heterogeneous mixtures
		\item Physical and chemical properties
		\item Intensive and extensive properties
		\item Physical and chemical changes
		\item Finding derived units from measurements (e.g. density from mass and volume)
		\item Converting between units (be careful for squared and cubed units)
		\item Use of metric prefixes and scientific notation to describe very large or very small numbers
		\item Precision vs. accuracy
		\item Significant figures in a number
		\item Propogating significant figures in mathematical operations ($+-$ rule and $\times \div$ rule)
		\item Solving problems through dimensional analysis
	\end{itemize}
	\subsection*{Chapter 2 -- Atoms, Molecules, and Ions}
	\begin{itemize}
		\item Pivotal developments in atomic theory
		\begin{itemize}
			\item Law of fixed ratios
			\item Thompson's cathode ray experiment and the electron mass/charge ratio
			\item Millikan's oil drop experiment and the charge of an electron
			\item Rutherford's gold foil experiment and the discovery of the atomic nucleus
		\end{itemize}
		\item Modern view of an atom -- nucleus with electron cloud
		\item Neutrons, electrons, and protons
		\begin{itemize}
			\item Neutrons and protons make up most of the mass
			\item Electrons make up most of the volume
			\item Charges of electrons and protons
			\item Protons define the element
		\end{itemize}
		\item Writing atomic symbols from numbers of electrons, neutrons, and protons and vice-versa
		\item Calculating atomic weights from isotope mass and percent abundance
		\item The periodic table
		\begin{itemize}
			\item Names of certain families (alkali, alkaline earth, chalcogens, halogens, and noble gases)
			\item Metals vs. non-metal vs. metalloids
			\item Transition metals vs. main group elements
		\end{itemize}
		\item Molecular vs ionic compounds
		\item Empirical formulas
		\item Representing molecular structure
		\item Predicting the stable charge an ion will take
		\item Important polyatomic ions
		\item Balancing charges in ionic compounds
		\item Naming inorganic compounds
		\begin{itemize}
			\item Non-metal anions end in “--ide”
			\item Include the roman numeral for the charge of transition metal cations
		\end{itemize}
		\item Naming binary molecular compounds
		\begin{itemize}
			\item Second element ends in “--ide”
			\item Greek prefixes (mono, di, tri, tetra, etc.)
		\end{itemize}
		\item Naming organic compounds
		\begin{itemize}
			\item prefixes (meth--, eth--, prop--, etc.)
			\item suffixes --ane and --anol
		\end{itemize}
	\end{itemize}	
\end{document}