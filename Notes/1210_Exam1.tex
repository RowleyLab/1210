\documentclass[12pt, addpoints, letterpaper]{exam}
\usepackage{ExamStyle}

% Openstax textbook
\begin{document}
  \qformat{{\large Problem \thequestion : (\thepoints) \hfill}}
	\begin{center}
		\vspace{12em}
		{\Huge CHEM 1210 (Rowley) }

		{\LARGE Exam 1 (Chapters 1-2) Fall 2025}
	\end{center}

  \vspace{\stretch{1}}
  
  \begin{center}
	{\large Formulas}

  \hrulefill
  \end{center}
	
  \begin{minipage}[c]{0.495\linewidth}
    \begin{equation*}
      T_K=T_{^\circ C}+273.15
    \end{equation*}

    \begin{equation*}
      d=\frac{m}{V}
    \end{equation*}
  \end{minipage}
  \rule[-80pt]{1.5pt}{160pt}
  \begin{minipage}[c]{0.495\linewidth}
    \begin{equation*}
      A.W. =\sum\limits_{i=isotope}mass_i\frac{\%abundance_i}{100\%}
    \end{equation*}
  \end{minipage}

  \vspace{\stretch{1}}
  \begin{center}
	{\large Constants}

  \hrulefill
  \end{center}
	
  \begin{minipage}[c]{0.495\linewidth}
	$R=8.314 \dfrac{J}{mol~K}$
  \end{minipage}	
  \begin{minipage}[c]{0.495\linewidth}
	$R=0.08206 \dfrac{L~atm}{mol~K}$
  \end{minipage}	
	
  \vspace{\stretch{1}}
	\begin{center}
		Feel free to use this page as scratch paper, but final work for questions must be shown \emph{in the question's section} to count for credit.
	\end{center}
	\newpage
  \newgeometry{top=2mm, bottom=2mm, left=2mm, right=2mm}

  \hspace{6em}	\includegraphics[width=1.27\textwidth, angle =90]{UpdatedTable}

  \restoregeometry

	\newpage
  {\Large Name \rule[-1mm]{4in}{.1pt} \hspace{2em} Points: ${\dfrac{~}{~~\numpoints~~}}$}

  \begin{questions}
    \question[4] 
    A runner in a race runs $100.0m$ with a time of $12.54s$. What is their running speed in $mph$? ($1km=0.621371miles$)

    \vspace{4em}

    \vspace{\stretch{1}}
    \question[4]
    For each number below, give the quantity of significant figures, and the position of the least significant figure as a power of $10$

    \vspace{2em}
    $12.004g$ \hspace{\stretch{2}} $0.00240m$ \hspace{\stretch{2}} $3.20\times10^{-6}s$ \hspace{\stretch{2}} $9.2\times10^5m$

    \vspace{2em}

    \vspace{\stretch{1}}
    \question[4]
    For each of the \nth{19} century scientists below, briefly describe what major feature of atomic theory was demonstrated through their work

    \begin{itemize}
      \item Rutherford:

        \vspace{2em}
      \item Thompson:

        \vspace{2em}
      \item Millikan:

        \vspace{2em}
    \end{itemize}

    \vspace{\stretch{1}}
    \question[4]
    For each of the substances below, categorize them as an element, compound, heterogeneous mixture, or homogeneous mixture

    \begin{minipage}{0.49\linewidth}
      \begin{itemize}
        \item Titanium

          \vspace{2em}
        \item Tears (from your eyes)

          \vspace{2em}
        \item Granola

          \vspace{2em}
        \item Carbon monoxide

          \vspace{2em}
      \end{itemize}
    \end{minipage}
    \begin{minipage}{0.49\linewidth}
      \begin{itemize}
        \item Ethanol

          \vspace{2em}
        \item Chicken casserole

          \vspace{2em}
        \item Table salt (\ch{NaCl})

          \vspace{2em}
        \item Silver

          \vspace{2em}
      \end{itemize}
    \end{minipage}

    \vspace{\stretch{1}}
    \question[4]
	State whether each change or property is physical or chemical (just write "P" or "C")

  \noindent
  \begin{minipage}{0.55\linewidth}
    Changes
    \begin{itemize}
      \item A gram of sugar dissolves in hot water
      \item Eaten food is metabolized
      \item Iron is reshaped by a blacksmith
      \item A river carves a canyon through erosion
    \end{itemize}
  \end{minipage}
  \begin{minipage}{0.35\linewidth}
    Properties
    \begin{itemize}
      \item Density
      \item Flammability
      \item Melting Point
      \item Electrical Conductivity
    \end{itemize}
  \end{minipage}
    %
    % For each process below, indicate whether it is a physical change or a chemical change
    %
    % \begin{minipage}{0.49\linewidth}
    %   \begin{itemize}
    %     \item Wood burning
    %
    %       \vspace{2em}
    %     \item Dissolving salt in water
    %
    %       \vspace{2em}
    %     \item Ripping paper
    %
    %       \vspace{2em}
    %     \item Melting iron
    %
    %       \vspace{2em}
    %   \end{itemize}
    % \end{minipage}
    % \begin{minipage}{0.49\linewidth}
    %   \begin{itemize}
    %     \item Spinning yarn
    %
    %       \vspace{2em}
    %     \item Metal rusting
    %
    %       \vspace{2em}
    %     \item Evaporation
    %
    %       \vspace{2em}
    %     \item Digesting food
    %
    %       \vspace{2em}
    %   \end{itemize}
    % \end{minipage}

    \vspace{\stretch{1}}
    \question[4]
    For each compound below, list how many atoms of each element are present.

    \vspace{1em}
    \ch{(NH4)3PO4} \hspace{\stretch{2}} \ch{CH3CH2COCH3} \hspace{\stretch{2}} \ch{MgSO4 $\cdot~7$ H2O} \hspace{\stretch{2}}
  
    \vspace{1em}

    \vspace{\stretch{1}}
    \question[4]
    Use your own words to briefly describe \emph{two} of the four postulates of Dalton's atomic theory

    \vspace{6em}

    \vspace{\stretch{1}}
    \question[4]
    Oxygen will condense from the gas phase into a liquid at $-183^\circ C$. This temperature is called the normal boiling point. Give the normal boiling point for oxygen using the Kelvin temperature scale

    \vspace{3em}

    \vspace{\stretch{1}}
    \question[4]
    Rubidium has 2 stable isotopes with the following properties:
    
    \begin{tabular}{|c|c|c|}\midrule
      Isotope & \ch{^{$85$}Rb} & \ch{^{$87$}Rb}\\ \midrule
      Mass (amu) & $84.911789$ & $86.909183$\\ \midrule 
      \% Abundance & $72.17$ & $27.83$\\ \midrule
      
    \end{tabular}
    
    \noindent Based on these measurements, what atomic mass should we find on the periodic table?

    \vspace{6em}

    \vspace{\stretch{1}}
    \question[4]
    Atoms are composed of electrons, protons, and neutrons. Interestingly, among these particles two of them account for most of the mass but virtually none of the volume, and one of them accounts for most of the volume but virtually none of the mass. Which subatomic particles belong to which categories?

    \vspace{1em}
    Large volume, low mass: \hspace{\stretch{3}} Large mass, low volume: \hspace{\stretch{3}}

    \vspace{1em}

    \vspace{\stretch{1}}
    \question[4]
    Match each prefix to its appropriate power of 10
    
    \begin{tabular}{rlcccl}
      nano-&\rule[-1mm]{2cm}{0.1pt}&~&&& A) $10^{-2}$ \\
      centi-&\rule[-1mm]{2cm}{0.1pt}&&~&& B) $10^{3}$ \\
      kilo-&\rule[-1mm]{2cm}{0.1pt}&&&& C) $10^{-3}$ \\
      micro-&\rule[-1mm]{2cm}{0.1pt}&&&& D) $10^{-6}$ \\
      milli-&\rule[-1mm]{2cm}{0.1pt}&&&& E) $10^{-9}$ \\
    \end{tabular}

    \vspace{\stretch{1}}
    \question[8]
    Fill in the blank portions of the following table 
    
    \begin{tabular}{|c|c|c|c|c|c|}
      \midrule
      Symbol 			& \# of Protons & \# of Neutrons	& \# of Electrons 	& Mass \# 	& Charge\\ \midrule
              & $12$			& $13$				& $12$ 				& 			&\\ \midrule
      \ch{^{85}Rb^{+}}&				&					&					&			&\\ \midrule
              & $50$			& 					& $50$				& $119$		&\\ \midrule
      \ch{^{79}Se^{2-}}& 			& 					& 				& 		&\\ \midrule
    \end{tabular}

    \vspace{\stretch{1}}
    \question[4]
    The following table gives the density of several common metals. 
    
    \begin{minipage}{0.3\linewidth}
      \begin{tabular}{c|c}
        Metal & Density (\nicefrac{g}{cm$^3$})\\ \midrule
        Gold & $19.32$\\
        Rhodium & $12.4$ \\
        Copper & $8.96$ \\
        Niobium & $8.57$ \\
        Iron & $7.87$\\
        Vanadium & $6.11$ \\
        Zirconium & $6.51$
      \end{tabular}
    \end{minipage} ~ 
    \begin{minipage}{0.6\linewidth}
      A cube of metal measures $2.56$ mm on each side, and has a mass of $150.3$ mg.
      
      $\circ$ Find the volume of the cube in $cm^3$.
      
      \vspace{4em}
      $\circ$ Find the density of the metal cube, and identify it based on the above table of densities.
    \end{minipage}

    \vspace{\stretch{1}}
    \question[4]
    Tungsten (\ch{W}) has a density of $19.28 \nicefrac{g}{cm^3}$. Find the volume (in $cm^3$) of $2.68g$ of tungsten

    \vspace{4em}

    \vspace{\stretch{1}}
    \question[4]
    A rock is measured to weigh $5.456~g$, then placed in a graduated cylinder of water. Below are images of the cylinder before and after the rock is added:

    \includegraphics[width=\textwidth]{volume}

    \noindent
    $\circ$ Find the density of the rock in $\dfrac{g}{ml}$ (take care with significant figures!)

    \vspace{12em}\noindent
    $\circ$ Give the density in $\dfrac{kg}{m^3}$ (you will need the relation: $1~ml = 1~cm^3$)

    \vspace{4em}

    \vspace{\stretch{1}}
    \question[8]
    Classify each of the following elements as a \emph{Halogen}, \emph{Alkali Metal}, \emph{Transition Metal}, \emph{Inner Transition Metal}, \emph{Chalcogen}, \emph{Noble Gas}, \emph{Metalloid}, or \emph{Alkaline Earth Metal}
      
      \vspace{1em}
      \begin{minipage}{0.45\linewidth}
      \begin{tabular}{rl}
        Ne & \rule[-1mm]{2in}{1pt}\\ \\
        Li & \rule[-1mm]{2in}{1pt}\\ \\
        Cl & \rule[-1mm]{2in}{1pt}\\ \\
        Au & \rule[-1mm]{2in}{1pt}\\ \\
        Si & \rule[-1mm]{2in}{1pt}\\ \\
      \end{tabular}
      \end{minipage}
      \begin{minipage}{0.45\linewidth}
      \begin{tabular}{rl}
        Ca & \rule[-1mm]{2in}{1pt}\\ \\
        S & \rule[-1mm]{2in}{1pt}\\ \\
        Br & \rule[-1mm]{2in}{1pt}\\ \\
        Zn & \rule[-1mm]{2in}{1pt}\\ \\
        Kr & \rule[-1mm]{2in}{1pt}\\ \\
      \end{tabular}
      \end{minipage}

    \vspace{\stretch{1}}
    \question[4]
    A scientific \emph{hypothesis} only becomes a \emph{theory} after it has been supported by many experiments and accepted as highly reliable by a consensus in the scientific community

    {\Large True \hspace{4em} False}

    \vspace{\stretch{1}}
    \question[4]
    List the name of your \emph{favorite} element, its atomic symbol, and the reason you love it!

    \vspace{\stretch{1}}
    \question[4]
    Give the name or the chemical formula for each compound below:

    \begin{tabular}{ccc}
      Formula && Name \\ \midrule \\
      \ch{W2O} && \rule[-1mm]{2in}{.1pt} \\ \\	
      \ch{Ca(ClO2)2} && \rule[-1mm]{2in}{.1pt} \\ \\
      \rule[-1mm]{1in}{.1pt} && Iron(III) hydrogen sulfate \\ \\
      \rule[-1mm]{1in}{.1pt} && Lithium bicarbonate \\ \\
    \end{tabular}	

    \vspace{\stretch{1}}
    \question[4]
    Give the name or the chemical formula for each compound below:

    \begin{tabular}{ccc}
      Formula && Name \\ \midrule \\
      \ch{XeF4} && \rule[-1mm]{2in}{.1pt} \\ \\	
      \ch{P2Cl5} && \rule[-1mm]{2in}{.1pt} \\ \\
      \rule[-1mm]{1in}{.1pt} && Sulfur difluoride \\ \\
      \rule[-1mm]{1in}{.1pt} && Triphosphorous disulfide \\ \\
    \end{tabular}	

    \vspace{\stretch{1}}
    \question[4]
    Give the name or the chemical formula for each compound below:

    \begin{tabular}{ccc}
      Formula && Name \\ \midrule \\
      \ch{H2SO4} && \rule[-1mm]{2in}{.1pt} \\ \\	
      \ch{CuSO4 $\cdot 5$ H2O} && \rule[-1mm]{2in}{.1pt} \\ \\
      \rule[-1mm]{1in}{.1pt} && Nitrous acid \\ \\
      \rule[-1mm]{1in}{.1pt} && Calcium chloride dihydrate \\ \\
    \end{tabular}	

    \vspace{\stretch{1}}
    \question[4]
    Give the answer to the problems below, with units and the correct number of significant figures:

    $x=\dfrac{1.50g+9.13g}{12.34ml}$ \hspace{\stretch{4}} $3.45cm\times8.64cm-2.93cm\times9.01cm$\hspace{\stretch{4}}

    \vspace{3em}

    \vspace{\stretch{1}}
    \question[4]
    In the space below, draw arrows which represent all possible phase changes and label the arrows, giving the names of the phase changes.

    \begin{center}
      Gas

      \vspace{3em}
      Liquid

      \vspace{3em}
      Solid
    \end{center}

    \vspace{\stretch{1}}
    \end{questions}

\end{document}	
