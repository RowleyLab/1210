\documentclass[12pt, openany, letterpaper]{memoir}
\usepackage{NotesStyle}
%\renewcommand\thesection{\thechapter\Alph{section}}
%\renewcommand\thesubsection{\thesection.\Numeral{subsection}}

\begin{document}
\title{CHEM 1210 Lecture Notes\\ OpenStax Chemistry 2e}
\author{Matthew Rowley}
\mainmatter
\maketitle
\chapter*{Course Administrative Details}
\begin{itemize}
	\item My office hours
	\item Intro to my research
	\item Introductory Quiz
	\item Grading details
	      \begin{itemize}
		      \item Exams - 40, Final - 20, Online Homework - 15, Book Homework - 15, Quizzes - 10
		      \item Online homework
		      \item Frequent quizzes
	      \end{itemize}
	\item Importance of reading and learning on your own
	\item Learning resources
	      \begin{itemize}
		      \item My Office Hours
		      \item Tutoring services - \href{https://www.suu.edu/academicsuccess/tutoring/}{https://www.suu.edu/academicsuccess/tutoring/}
	      \end{itemize}
	\item Show how to access Canvas
	      \begin{itemize}
		      \item Calendar, Grades, Modules, etc.
		      \item Quizzes
		      \item Textbook
	      \end{itemize}
	\item Introduction to chemistry
	      \begin{itemize}
		      \item Ruby fluorescence
		      \item Levomethamphetamine
		      \item Rubber band elasticity
		      \item Structure of the periodic table
		      \item Salt on ice and purifying hydrogen peroxide
	      \end{itemize}
\end{itemize}

\chapter{Essential Ideas}

\section{Chemistry in Context}

\section{Phases and Classification of Matter}

\section{Physical and Chemical Properties}

\section{Measurements}

\section{Measurement Uncertainty, Accuracy, and Precision}

\section{Mathematical Treatment of Measurement Results}

\chapter{Atoms, Molecules, and Ions}

\section{Early Ideas in Atomic Theory}

\section{Evolution of Atomic Theory}

\section{Atomic Structure and Symbolism}

\section{Chemical Formulas}

\section{The Periodic Table}

\section{Ionic and Molecular Compounds}

\section{Chemical Nomenclature}

\chapter{Composition of Substances and Solutions}

\section{Formula Mass and the Mole Concept}

\section{Determining Empirical and Molecular Formulas}

\section{Molarity}

\section{Other Units for Solution Concentration}

\chapter{Stoichiometry of Chemical Reactions}

\section{Writing and Balancing Chemical Equations}

\section{Classifying Chemical Reactions}

\section{Reaction Stoichiometry}

\section{Reaction Yields}

\section{Quantitative Chemical Analysis}

\chapter{Thermochemistry}

\section{Energy Basics}

\section{Calorimetry}

\section{Enthalpy}

\chapter{Electronic Structure and Periodic Properties of Elements}

\section{Electromagnetic Energy}

\section{The Bohr Model}

\section{Development of Quantum Theory}

\section{Electronic Structure of Atoms (Electron Configurations)}

\section{Periodic Variations in Element Properties}

\chapter{Chemical Bonding and Molecular Geometry}

\section{Ionic Bonding}

\section{Covalent Bonding}

\section{Lewis Symbols and Structures}

\section{Formal Charges and Resonance}

\section{Strengths of Ionic and Covalent Bonds}

\section{Molecular Structure and Polarity}

\chapter{Advanced Theories of Covalent Bonding}

\section{Valence Bond Theory}

\section{Hybrid Atomic Orbitals}

\section{Multiple Bonds}

\section{Molecular Orbital Theory}

\chapter{Gases}

\section{Gas Pressure}

\section{Relating Pressure, Volume, Amount, and Temperature: The Ideal Gas Law}

\section{Stoichiometry of Gaseous Substances, Mixtures, and Reactions}

\section{Effusion and Diffusion of Gases}

\section{The Kinetic-Molecular Theory}

\section{Non-Ideal Gas Behavior}

\chapter{Liquids and Solids}

\section{Intermolecular Forces}
\backmatter
\chapter{Errata}
\end{document}
