\documentclass[12pt, openany, letterpaper]{memoir}
\usepackage{NotesStyle}
%\renewcommand\thesection{\thechapter\Alph{section}}
%\renewcommand\thesubsection{\thesection.\Numeral{subsection}}

\begin{document}
\title{CHEM 1210 Lecture Notes\\ OpenStax Chemistry 2e}
\author{Matthew Rowley}
\mainmatter
\maketitle
\chapter*{Course Administrative Details}
\begin{itemize}
	\item My office hours
	\item Intro to my research
	\item Introductory Quiz
	\item Grading details
	      \begin{itemize}
		      \item Exams - 40, Final - 20, Online Homework - 15, Book Homework - 15, Quizzes - 10
		      \item Online homework
		      \item Frequent quizzes
	      \end{itemize}
	\item Importance of reading and learning on your own
	\item Learning resources
	      \begin{itemize}
		      \item My Office Hours
		      \item Tutoring services - \href{https://www.suu.edu/academicsuccess/tutoring/}{https://www.suu.edu/academicsuccess/tutoring/}
	      \end{itemize}
	\item Show how to access Canvas
	      \begin{itemize}
		      \item Calendar, Grades, Modules, etc.
		      \item Quizzes
		      \item Textbook
	      \end{itemize}
	\item Introduction to chemistry
	      \begin{itemize}
		      \item Ruby fluorescence
		      \item Levomethamphetamine
          \item Submerged salt crystals grow due dynamic equilibrium
		      \item Rubber band elasticity
		      \item Structure of the periodic table
		      \item Salt on ice and purifying hydrogen peroxide
	      \end{itemize}
\end{itemize}

\chapter{Essential Ideas}

\section{Chemistry in Context}
\begin{itemize}
  \item Modern chemistry is the end result of thousands of years of humans trying to explain and control the materials around them
  \item Early forays into chemistry (such as alchemy) had deep mystical roots and often relied on serendipity to make good progress
  \item Modern chemistry is a rigorous science, relying on falsifiability and the scientific methd (Figure 1.4)
  \item We sometimes refer to chemistry as ``The Central Science'' (Figure 1.3)
  \item To adequately describe and understand chemical phenomena, we often talk from different perspectives
  \begin{description}
    \item[Macroscopic Domain] This is what we observe with bulk substances. Two chemicals react to produce a new chemical
    \item[Microscopic Domain] We now understand that all microscopic effects are governed by the behavior of \emph{microscopic} actors (molecules, atoms, electrons, etc.)
    \item[Symbolic Domain] Effectively communicating chemical ideas requires new language. Chemical formulas, equations, and mechanisms are all symbolic representations
    \item All three domains are on display in Figure 1.5
  \end{description}
\end{itemize}
\paragraph*{Quiz 1.1 - Scientific Method}
\paragraph*{Homework 1.1}
\begin{itemize}
  \item 1: Thinking in terms of Chemistry
  \item 3: The scientific method
  \item 5: Domains of inquiry
\end{itemize}

\section{Phases and Classification of Matter}
\begin{itemize}
  \item Three primary phases of matter are shown in Figure 1.5 (and 1.6)
  \item Plasmas are like a gas, but with electrically charged particles
  \item Mass vs Weight (for very fine measurements, the difference matters even on Earth due to buoyancy)
  \item Figure 1.8 illustrates the \emph{law of conservation of matter}
  \item Classifying matter (Figure 1.11)
  \begin{itemize}
    \item Pure Substances
    \begin{itemize}
      \item Elements (Anything on the \emph{periodic table of the elements})
      \item Compounds (Combinations of elements -- can have very different properties from their constituent elements)
    \end{itemize}
    \item Mixtures
    \begin{itemize}
      \item Heterogeneous mixtures (variable composition)
      \item Homogeneous mixtures (i.e. solutions, continuous composition)
    \end{itemize}
  \end{itemize}
  \item Table 1.1 shows the abundance of many elements on Earth
  \item Atoms are the smallest particle of an element that has the properties of that element
  \begin{itemize}
    \item Thought-experiment of dividing a sample in half ad-infinitum
    \item Ancient atomic theories and modern Dalton atomic theory (discussed in detail later)
    \item Atoms are \emph{very} small; smaller than we could even detect until recently
  \end{itemize}
  \item Molecules are collections of atoms held together with chemical bonds (more nuanced definition later)
    \begin{itemize}
      \item Many elements occur naturally as molecules, rather than atoms
      \item Figure 1.14 shows many molecular elements and compounds
    \end{itemize}
\end{itemize}

\section{Physical and Chemical Properties}
\begin{itemize}
  \item Physical Properties: Properties which can be observed without changing the chemical identity of the substance
  \item Chemical Properties: Properties which can only be observed through chemical reactions (e.g. flammability, acidity, electrochemical potential, etc.)
  \item Physical Changes: Any change which perserves the \emph{chemical identity} of the substance (including phase changes)
  \item Chemical Changes: Changes which alter the chemical identities of one of more substance
  \item Extensive Properties: Depend on the size of hte system (double the size, double the property measurement, such as mass or volume)
  \item Intensive Properties: Independent of system size (density, temperature, most chemical properties)
  \item The periodic table groups elements according to their properties (Figure 1.22)
  \begin{itemize}
    \item Metals conduct electricity and heat, are maleable and ductile
    \item Non-metals are very diverse, but generally poor conductors
    \item Metalloids exist at the boundary and share properties with both metals and non-metals
    \item There are many other ways to group the elements, which we will learn later
  \end{itemize}
\end{itemize}
\paragraph*{Quiz 1.2 - Matter, Properties, and Change}
\paragraph*{Homework 1.2}
\begin{itemize}
  \item 17: Classifying matter
  \item 27: Classifying changes
\end{itemize}

\section{Measurements}
\begin{itemize}
  \item All measurements are composed of three parts:
  \begin{itemize}
    \item The magnitude of the measurement (the number itself)
    \item The unit of measurement used (g, kg, lbs, etc.)
    \item The degree of uncertainty in the measurement (this is usually implicit, and covered in the next section)
  \end{itemize}
  \item Units are an essential part of any measuement. Develop a habit of \emph{always} including units in your work
  \begin{itemize}
    \item $u_{rms}=\sqrt{\dfrac{3RT}{M}}$ -- example of how units can guide problem solving and ``unit purgatory''
    \item SI units are a collection of fundamental units from which all other units can be derived (Table 1.2)
    \item Metric prefixes make it more convenient to discuss very large or very small numbers (Table 1.3)
    \item Scientific notation is an even more general and robust way of representing numbers
    \begin{itemize}
      \item The quantity is represented by a number with the decimal after the first digit
      \item The magnitude is represented by a power of $10$
    \end{itemize}
    \item Practice converting between normal numbers, metric prefixes, and scientific notation
    \item For temperature, we use both $K$ and $^\circ C$ (But not $^\circ F$)
      \\$T(K) = T(^\circ C) + 273.15$ 
    \item Derived units will combine the fundamental units in some way 
      \\volume: $m^3$, $L$, $ml$
      \\velocity: $\nicefrac{m}{s}$ 
      \\density: $\nicefrac{kg}{m^3}$, $\nicefrac{g}{cm^3}$ (Table 1.4)
      \\energy: $1J\equiv \nicefrac{kgm^2}{s^2}$
  \end{itemize} 
\end{itemize}

\section{Measurement Uncertainty, Accuracy, and Precision}
\begin{itemize}
  \item Countable quantities are considered to be \emph{exact} (no uncertainty)
  \item Measurements (and groups of measurements) always have some degree of undertainty
  \begin{itemize}
    \item Accuracy is how close a measurement is to the \emph{true value} (usually unknown, but approximated by calibration with a well-known standard)
    \item Precision is how finely a measurment is made (What is the margin of error)
    \item Figure 1.27 and Table 1.5 illustrate the differences between precision and accuracy
    \item Accuracy is usually improved through calibration, and moving forward we will usually assume that measurements are as accurate as an instrument allows
    \item Precision is represented in the way we write the number, and can be improved with a better instrument or with repeat measurements
  \end{itemize}
  \item Significant figures are the way that we represent precision in a number
  \begin{itemize}
    \item The number of digits conveys the degree of precision
    \item Example of me saying I'm $6ft~2in$ tall, vs me saying I'm $6ft~1.6241434in$ tall
    \item For graduated measurements, we record one digit beyond the lowest graduation (Figure 1.26)
    \item For digital measurements, we record the number as it is given by the instrument
    \item For any given number, we should track both the \emph{quantity} of significant figures, and the \emph{position} of the least-significant digit
    \item In a written number, digits are considered significant according to the following rules:
    \begin{itemize}
      \item All non-zeros are significant
      \item All \emph{captive} zeros (between two other significant digits) are significant
      \item Trailing zeros are \emph{always} significant
      \item Leading zeros are \emph{never} significant
      \item For scientific notation, only the digits of the quantity (not the magnitude) count
      \item Logarithmic quantities follow different rules which we will revisit in CHEM 1220 (chapter 14)
      \item Note that for some numbers scientific notation is \emph{required} to convey the correct precision ($3.0\times10^3m$)
    \end{itemize}
  \end{itemize}
  \item Errors propogate when multiple measurements are used in a mathematical operation
  \begin{itemize}
    \item For addition and subtraction, the least significant digit of the answer will be in the same position as the least significant digit of hte least precise input
    \item For multiplicationa and division, the quantity of significant digits in the answer will match the quantity of significant digits of the input with fewest significant digits
    \item When rounding an exact $5$ (no further digits beyond the $5$), round up or down to make the last digit even
    \item Compound problems involve multiple types of operations
    \begin{itemize}
      \item Solve the problem in steps, applying the correct rule to each step
      \item Track the significant figures (quantity and position) for each intermediate answer, but do \emph{not} truncate or round any of these answers
      \item Only round after the last step
        \\ $\circ$ Practice $\frac{12.3g+34g}{12.0cm^3+7.7cm^3}=2.4\nicefrac{g}{cm^3}$ (wrong answer with premature rounding)
    \end{itemize}
  \end{itemize}
\end{itemize}
\paragraph*{Quiz 1.3 - Significant Figures}
\paragraph*{Homework 1.3}
\begin{itemize}
  \item 45: Scientific Notation
  \item 49: Counting Significant Figures
  \item 53: Significnat Figures and Calculations
\end{itemize}

\section{Mathematical Treatment of Measurement Results}
\begin{itemize}
  \item Some quantities are calculated based on two or more measurements (such as velocity and density)
  \item These formulas can be used to relate all three quantities together (i.e. $velocity = \frac{distance}{time}$)
  \item The derived quantity can be interpreted as a \emph{comversion factor} between the other two quantitites
  \item Conversion factors and unit conversions 
  \begin{itemize}
    \item Elementary school perspective of $ft$ to $in$ conversions
    \item Conversion factors are a ratio between two identical quantities
    \item Converting units involves multiplying by $1$ in the form of a conversion factor
    \item Units guide the problem solving
  \end{itemize}
  \item Dimensional Analysis is a problem-solving framework based on a series of unit conversions
  \begin{itemize}
    \item Don't dive straight into calculations and equations
    \item Identify the units you expect for the answer
    \item Identify the starting point
    \item Create a plan to convert units from the starting point to the answer
    \item Carry out the calculations
    \item Practice converting $65.0\nicefrac{miles}{hour}$ into $\nicefrac{m}{s}$
    \item The ``railroad ties'' or ``picket fence'' method can help organize your work
  \end{itemize}
  \item Dimensional analysis is not the only way to solve problems, but it is versatile and robust; usually my preferred choice
  \item Practice a more abstract problem:
    \\ Find the $\nicefrac{miles}{gal}$ if a car consumes $8036~g$ of gasoline while driving for $40.0~min$ at $75~\nicefrac{miles}{hour}$
\end{itemize}
\paragraph*{Quiz 1.4 - Dimensional Analysis}
\paragraph*{Homework 1.4}
\begin{itemize}
  \item 65: Simple unit conversion
  \item 87: Density from volume and mass
  \item 89: Mass from volume
  \item 91: Volume from mass
\end{itemize}

\chapter{Atoms, Molecules, and Ions}

\section{Early Ideas in Atomic Theory}
\begin{itemize}
  \item 1807 Dalton's Atomic Theory: (1, 2 and 5 are not strictly true) (Figures 2.2-2.4)
  \begin{enumerate}
    \item Matter is composed of atoms
    \item Atoms of a given element all have identical properties to each other
    \item Atoms of one element differ in properties from elements of a different element
    \item Chemical compounds consist of atoms of different elements combined in a specific ratio
    \item Chemical reactions \emph{rearrange} the atoms which are already there, but cannot create or destroy atoms
  \end{enumerate}
  \item Development of Dalton's theory:
  \begin{itemize}
    \item Dalton relied on prior work by Proust who demonstrated the law of definite proportions (Table 2.1)
    \item This was not at all expected - my analogy with bread, or metal alloys
    \item Dalton further noted that ratios of these proportions followed the law of multiple proportions (Copper (I or II) Chloride example in the book)
  \end{itemize}
\end{itemize}

\section{Evolution of Atomic Theory}
\begin{itemize}
  \item About a century later, scientists discovered that atoms are made of even smaller components
  \item J. J. Thomson discovered the electron, and its charge/mass ratio (Figure 2.6)
  \item Millikan's oil drop experiment found the fundamental charge (and thus mass) of an electron (Figure 2.7)
  \item Figure 2.8 shows some early ideas of how the positive and negative charges were distributed in an atom
  \item Ernest Rutherford discovered the atomic nucleus, consisting of very concetrated positive charge (Figures 2.9 and 2.10)

    ``It was quite the most incredible event that has ever happened to me in my life. It was almost as incredible as if you fired a 15-inch shell at a piece of tissue paper and it came back and hit you.''
  \item Different \emph{isotopes} of atoms were discovered with techniqes that produced isotopically pure samples
  \item Finally, the neutron itself was discovered in 1932, explaining what particle led to different isotopes
\end{itemize}

\section{Atomic Structure and Symbolism}
\begin{itemize}
  \item Atoms are made up of protons, neutrons, and electrons
  \item Figure 2.11 shows the small scale of the atom and nucleus
  \item Atomic units make discussions about atoms convenient
  \begin{itemize}
    \item The Atomic Mass Unit $amu$, $Da$, or $u = 1.6605\times10^{-24}g$
    \item The fundamental charge $e = 1.602\times10^{-19}C$
    \item The Angstrom \AA$=10^{-10}m$
  \end{itemize}
  \item Table 2.2 summarizes the properties of elementary particles in atoms
  \item We track the composition of an atom with three numbers:
  \begin{itemize}
    \item The atomic number is the number of protons $Z=p$ 
    \item The mass number is the number of protons and neutrons $A=p+n$ 
    \item The number of neutrons is therefore $n=A-Z$
    \item The charge is the protons minus the electrons $q=p-e$
    \item The number of electrons is $e=p-q$
  \end{itemize}
  \item A positively charge atom is called a cation, and a negatively charged atom is called an anion
  \item Chemical symbols are a shorthand way of representing everything we need about an atom
  \begin{itemize}
    \item There is a 1 or 2 letter symbol for each element (Table 2.3 shows some make sense, some don't)
    \item $A$ is written as a left superscript
    \item $Z$ is written as a left subscript, but can be left off
    \item $q$ is written as a right superscript with the magnitude first, then the sign. $q$ is left off if $q=0$
    \item Example: $^{13}_6C^{2+}$ (6 protons, 7 neutrons, 4 electrons)
  \end{itemize}
  \item Isotopes are different versions of elements with different mass numbers
  \begin{itemize}
    \item For the most part, different isotopes of an element behave exactly the same in chemisry
    \item Isotope abundance can be found by mass spectrometry, among other methods (Figure 2.15)
    \item Table 2.4 shows the natural abundances of the isotopes of several light elements
    \item Atomic weight (atomic mass) is the weighted average of all the isotopes found on the Earth

      $M = \sum\limits_{i}mass_i\times\dfrac{\% abundance_i}{100\%}$
  \end{itemize}
\end{itemize}

\section{Chemical Formulas}
\begin{itemize}
  \item We can represent the actual structure and makeup of molecules at several levels of abstraction (Figures 2.16 and 2.17)
  \item Molecular formulas
  \begin{itemize}
    \item Each element is listed, with the number of atoms for each element written as a subscript (\ch{H2O})
    \item The order of elements follows certain patterns, with the least electronegative element often written first
  \end{itemize}
  \item Structural Formulas show how atoms are connected with covalent bonds represented as lines
  \item Ball and Stick models show the three-dimensional geometry of a molecule
  \item Space-filling models show the actual volume of space taken up by each atom in a molecule
  \item Figure 2.18 illustrates the difference between subscripts within a formula, and stoichiometric coefficients in front of formulas
  \item Empirical formulas show the mathematically simplified ratios of elements
  \begin{itemize}
    \item Some experimental techniques (especially early ones) could \emph{only} give the empirical formula
    \item Ionic compounds are always reported with the empirical formula
    \item To find the empirical formula, divide all subscripts by their greatest common factor
    \item Some very different compounds share an empirical formula (carbohydrates \ch{CH2O} include formaldehyde, acetic acid, and sugar)
    \item The molecular formula can be calculated from the empirical formula's weight and the molecular weight (Chapter 3)
  \end{itemize}
  \item Arranging the same group of atoms in different ways produces different isomers
    \begin{itemize}
      \item Isomers share the same chemical formula, but can have very different properties
      \item Structural isomers differ in how the atoms are connected to each other (Figure 2.23)
      \item Optical isomers (or spatial isomers) are non-superimposable mirror images (Figure 2.24, glove analogy)
    \end{itemize}
\end{itemize}

\paragraph*{Quiz 2.1 - Atomic Theories}
\paragraph*{Homework 2.1}
\begin{itemize}
  \item 7: Properties of protons and neutrons
  \item 11: Atomic symbols from composition
  \item 19: Composition from atomic symbols
  \item 23: Atomic weight
  \item 29: Molecular and empirical formulas
\end{itemize}

\section{The Periodic Table}
\begin{itemize}
  \item As scientists discovered and studied more and more elements, they started to notice certain natural groupings according to physical and chemical properties
  \item Mendeleev arranged the atoms according to these groups and atomic weight, producing the first periodic table (Figure 2.25)
  \item Mendeleev even predicted the existence of and properties of yet-undiscovered elements based on gaps in his table
  \item Vocabulary around the periodic table:
  \begin{itemize}
    \item Rows are periods or series
    \item Columns are groups or families
  \end{itemize}
  \item Figure 2.26 is a typical periodic table, showing the metals, non-metals and metalloids (contrast with my preferred table)
  \item Figure 2.27 shows many of the names we use for important groups of elements
  \item The structure of the periodic table encodes rich information about the electrons in the elements, as we will learn in chapter 6
\end{itemize}

\paragraph*{Quiz 2.2 - Periodic Table}
\paragraph*{Homework 2.2}
\begin{itemize}
  \item 37: Classifying elements
  \item 41: Using group names
\end{itemize}

\section{Ionic and Molecular Compounds}
\begin{itemize}
  \item An atom which gains or loses electrons (carries a charge) is called an \emph{ion}
  \begin{itemize}
    \item Positively charged ions are called cations, and are smaller than their neutral atom counterparts (Figure 2.28)
    \item Negatively charged ions are called anions, and are larger than their neutral atom counterparts
    \item We can predict which charge different elements will naturally tend to take based on their position in the periodic table (Figure 2.29)
    \item Many other elements can take two or more charges, especially the transition metals
  \end{itemize}
  \item Some ions are composed of more than one atom and are called polyatomic ions
  \begin{itemize}
    \item Table 2.5 gives some common polyatomic ions. Memorize the formula, name, and charge of these and their acids
    \item Notice some trends in the names of oxyanions (per-ate, -ate, -ite, and hypo-ite)
    \item We will learn about the acid names listed here (and more) in section 2.7
  \end{itemize}
  \item Ionic compounds are held together by ionic bonds (coulombic attractions)
  \begin{itemize}
    \item Show dot diagram of how \ch{NaCl} and \ch{CaCl} form from elements
    \item Metal + non-metal is not an adequate definition of ionic compounds (\ch{NH4NO3})
    \item Ionic compounds form an extended lattice of ions (covered more in CHEM 1220)
    \item Ions will combine to form neutral compounds
    \item Practice producing compound formulas from ions (include paranthesis for polyatomic ions where appropriate)
    \item Practice finding ions from formulas of compounds
  \end{itemize}
  \item Molecular compounds are held together by covalent bonds (shared electrons)
  \begin{itemize}
    \item Show a dot diagram of \ch{H2}, \ch{H2O}, \ch{O2}, and \ch{N2}
    \item Bonds between non-metals are covalent bonds
    \item Molecular compounds combine non-metals into discrete units called molecules
    \item Single, double, and triple bonds involve sharing two, four, and six electrons
  \end{itemize}
\end{itemize}

\section{Chemical Nomenclature}
\begin{itemize}
  \item Naming ionic compounds
  \begin{itemize}
    \item Cation names are the name of the element, with the charge in roman numerals in parenthesis \emph{if} the element could take multiple charges
    \item Anion names are the element name with an ``-ide'' ending (some element like \ch{P} remove more than others)
    \item Polyatomic ion names are the same as you learned earlier
    \item For ionic compounds the name is simply cation name + anion name
    \item There is no indication of the quantity of atoms, that is inferred from charge neutrality
    \item Practice getting formulas from names and names from formulas
  \end{itemize}
  \item Naming hydrates
  \begin{itemize}
    \item Some ionic compounds will incorporate water into their ionic lattice
    \item The formulas will have a $\cdot$ then list the number of waters
    \item The names add the degree of hydration using prefixes from table 2.10 and adding ``hydrate''
    \item The waters can be driven of with high temperature, producing the \emph{anhydrous} form
  \end{itemize}
\end{itemize}

\paragraph*{Quiz 2.3 - Naming Ionic Compounds}
\paragraph*{Homework 2.3}
\begin{itemize}
  \item 47: Predicting bond type in compounds
  \item 49: Formulas from ions
  \item 51: Names from formulas
  \item 57: Names from formulas with transition metals
  \item 59: Formulas from names
\end{itemize}

\paragraph*{Resuming section 2.7 Chemical Nomenclature}
\begin{itemize}
  \item Naming molecular compounds
  \begin{itemize}
    \item There are many ways to name molecular compounds, we will focus on just two here
    \item Naming binary molecular compounds
    \begin{itemize}
      \item \# + name + \# + name with ``-ide'' ending
      \item Least electronegative element (leftmost on the periodic table) goes first
      \item \#s come from table 2.10
      \item Omit ``mono-'' for the first element
      \item Practice going from formula to name and vice-versa (Table 2.11)
    \end{itemize}
    \item Naming molecular acids
    \begin{itemize}
      \item Molecular acid names are based on the name of the anion formed when all \ch{H+} are removed
      \item -ide ions form hydro-ic acids (Table 2.12)
      \item -ate ions form -ic acids
      \item -ite ions form -ous acids
      \item preserve the hypo- and per- prefixes
      \item Table 2.13 shows some oxyacid names
    \end{itemize}
  \end{itemize}
\end{itemize}

\paragraph*{Quiz 2.4 - Naming Molecular Compounds}
\paragraph*{No textbook homework to accompany this quiz due to a lack of appropriate questions!}

\chapter{Composition of Substances and Solutions}

\section{Formula Mass and the Mole Concept}
\begin{itemize}
  \item For chemists, the amount of a substance we care about is not grams, but the quantity of atoms
  \item Analogy of hammers and feathers
  \item The mole is defined as the relationship between $AMUs$ and $g$

    $6.022\times10^{23}AMU=1g \longrightarrow 1molAMU=1g \longrightarrow 1AMU=1\dfrac{g}{mol}$
  \item Any time we would use the $AMU$ we could replace it with $\nicefrac{g}{mol}$, and this is called the \emph{molar mass}
  \item Practice converting between grams and moles using atomic weight
  \item For compounds, we can find the molar mass, or formula mass by adding up the atomic weights of all the atoms in the formula
  \item We can count the literal number of atoms, but it is usually best to remain in units of moles
  \item Practice finding formula masses for compounds, and converting between grams and moles for compounds
\end{itemize}

\section{Determining Empirical and Molecular Formulas}
\begin{itemize}
  \item Empirical analysis is a method to determine empirical formulas
  \begin{itemize}
    \item In our text, the masses of C, H, and O are given directly
    \item To get the formula, turn each mass into moles, and divide by the smallest mole number (Figure 3.11)
    \item Actually, this technique involves a careful combustion reaction
    \item The mass of C comes from the amount of \ch{CO2} produced
    \item The mass of H comes from the amount of \ch{H2O} produced (remember the 2!)
    \item The mass of O comes from subtracting those from the original amount
    \item Turn the masses to moles
    \item Practice: $2.443g$ of unknown produce $5.368g$ of \ch{CO2} and $2.930g$ of \ch{H2O} ($M=60.096$, propanol)
    \item Practice: $3.450g$ of unknown produce $5.057g$ of \ch{CO2} and $2.070g$ of \ch{H2O} ($M=180.16$, hexose)
  \end{itemize}
  \item \% Composition gives the \% by mass of each element in a compound
  \begin{itemize}
    \item From masses, we can calculate the \% mass composition easily
    \item If \% mass is given, we can just assume $100.0g$ and turn the \%s into $g$ to do emipirical analysis
    \item From the empirical or molecular formula, we can get \% mass composition: $\%_X=\dfrac{nM_X}{M_{Formula}}$
  \end{itemize}
  \item We can determine the molecular formula from the empirical formula if we know the molar mass
  \begin{itemize}
    \item The molecular formula will be an integer multiple of the empirical formula (could be 1)
    \item That formula is the ratio $n=\dfrac{M_{molar}}{M_{empirical}}$
    \item Find the molecular formulas of the two examples above using their molar masses
  \end{itemize}
\end{itemize}

\paragraph*{Quiz 3.1 - Molar Mass}
\paragraph*{Homework 3.1}
\begin{itemize}
  \item 3: Calculating molar mass
  \item 17: Mass from moles
  \item 18: Moles from mass (sorry, no solution!)
\end{itemize}

\paragraph*{Quiz 3.2 - Empirical Analysis}
\paragraph*{Homework 3.2}
\begin{itemize}
  \item 33: \% composition from formulas
  \item 37: Empirical formulas from \% composition
  \item 39: Molecular formula from \% composition and molar mass
\end{itemize}

\section{Molarity}
\begin{itemize}
  \item Molarity is the most common way of describing the \emph{concentration} of a solute in a solution
  \begin{itemize}
    \item The unit for molarity is $M$, which is unfortunate because that is also used for molar masses
    \item Molarity is the moles of solute divided by the liters of solution $M=\dfrac{mol_{solute}}{L_{solution}}$
    \item Technically, the solution volume could be more or less than the solvent volume before dissolving. When we make a solution, we always describe the process as ``add enough water to meet the mark''
    \item The molarity can now be used as a conversion between moles and volume
    \item Practice: Find molar concentration of solution made by dissolving $13.5g$ \ch{NaCl} in $100.0ml$ of solution
    \item Practice: How many moles of \ch{NaCl} are contained in $35.62ml$ of the solution?
    \item Practice: How many $ml$ would be needed to provide $1.74\times10^{-3}mol$ of \ch{NaCl}?
  \end{itemize}
  \item Solutions can be diluted or concentrated
  \begin{itemize}
    \item The number of moles of solute is the same, just the volume of solution changes (by adding or removing solvent)
    \item $C_1V_1=C_2V_2$
    \item Practice: Find the molar concentration when $36.43ml$ of $1.5M$ \ch{HCl} are diluted to $100.0ml$?
    \item Practice: How many $ml$ of $0.125M$ \ch{NaNO3} should be used to make $50.0ml$ of $2.4\times10^{-3}M$?
  \end{itemize}
\end{itemize}

\section{Other Units for Solution Concentration}
\begin{itemize}
  \item There are other, less common units of concentration (including some not covered until chapter 11)
  \item Mass \%, or \% by mass
  \begin{itemize}
    \item $\%m/m=\dfrac{m_A}{m_{total}}100\%$
  \end{itemize}
  \item Volume \%, or \% by volume
  \begin{itemize}
    \item $\%v/v=\dfrac{V_A}{V_{solution}}100\%$
  \end{itemize}
  \item m/V \%
  \begin{itemize}
    \item $\%m/v=\dfrac{g_A}{V_{solution}}100\%$
  \end{itemize}
  \item ppm and ppb
    \begin{itemize}
      \item $ppm=\dfrac{m_A}{m_{total}}10^6ppm$
      \item $ppb=\dfrac{m_A}{m_{total}}10^9ppb$
    \end{itemize}
\end{itemize}

\paragraph*{Quiz 3.3 - Concentration}
\paragraph*{Homework 3.3}
\begin{itemize}
  \item 49: Mass from molarity and volume
  \item 53: Calculating molarity
  \item 61: Dilution
  \item 73: ppm
  \item 77: \%m/m
\end{itemize}

\chapter{Stoichiometry of Chemical Reactions}

\section{Writing and Balancing Chemical Equations}
\begin{itemize}
  \item A chemical equation relates the amounts of reactants and products in a chemical reaction
  \item Consider the reaction: \ch{CH4 + 2 O2 -> CO2 + 2 H2O} (Figure 4.2)
  \item Reactants on the left, and products are on the right
  \item The large numbers are called \emph{stoichiometric coefficients}
  \item The subscript numbers are part of the chemical identity of the reactants and products
  \item The stoichiometric coefficients tell the right ratio to combine reactants, and the ratio of products produced (Figure 4.3)
  \item Balancing chemical reactions:
    \begin{itemize}
      \item To balance a chemical equation, we will add the proper coefficients to give the same numbers and types of atoms on both sides of the reaction arrow
      \item Start with any repeated atom moieties, such as polyatomic ions, and balance them as a unit rather than counting individual atoms
      \item Next balance elements that appear in only one compound on each side
      \item Finally, balance any remaining elements (\ch{O} is often best left to balance last)
      \item If necessary, multiply all coefficients by an integer
    \end{itemize}
  Practice: Balance the following chemical equation: \ch{Fe2O3(s) + Al(s) -> Al2O3(s) + Fe(s)}

  ~\hphantom{Practice:} \ch{Fe2O3(s) + 2 Al(s) -> Al2O3(s) + 2 Fe(s)}

  Practice: Balance the following chemical equation: \ch{C8H18(l) + O2(g) -> CO2(g) + H2O(g)}

  ~\hphantom{Practice:} \ch{2 C8H18(l) + 25 O2(g) -> 16 CO2(g) + 18 H2O(g)}

  Practice: Balance the following chemical equation: \ch{Ag2SO4(aq) + NaCl(aq) -> AgCl(s) + Na2SO4(aq)}

  ~\hphantom{Practice:} \ch{Ag2SO4(aq) + 2 NaCl(aq) -> 2 AgCl(s) + Na2SO4(aq)}
  \item Chemical reactions will sometimes include additional information
    \begin{itemize}
      \item Phases (s), (l), (g), (aq)
      \item Reaction conditions, such as temperature, or the presence of a catalyst (often written over the arrow)
      \item Enthalpy of reaction or equilibrium constant (covered in later chapters)
    \end{itemize}
  \item Reactions with ionic compounds can often be written differently and simplified
    \begin{itemize}
      \item The normal equation is called a \emph{molecular equation} (despite involving ionic compounds)

        \ch{CaCl2(aq) + 2 AgNO3(aq) -> Ca(NO3)2(aq) + 2 AgCl(s)}
      \item The \emph{complete ionic equation} will break \emph{soluble} ionic compounds into the separate cations and anions
      \item This is a more accurate representation of soluble ionic compounds, which are not actually \emph{together} in any chemically meaningful way

        \ch{Ca^{2+}(aq) + 2 Cl^-(aq) + 2 Ag^+(aq) + 2 NO3^-(aq) -> Ca^{2+}(aq) + 2 NO3^-(aq) + 2 AgCl(s)}
      \item The \emph{net ionic equation} removes \emph{spectator ions}, which didn't actually participate in any chemical change
      \item My silly analogy about party-goers and the social interactions in a party environment

        \ch{Ag^+(aq) + Cl^-(aq) -> AgCl(s)}
    \end{itemize}
\end{itemize}

\paragraph*{Quiz 4.1 - Balance Chemical Equations}
\paragraph*{Homework 4.1}
\begin{itemize}
  \item 5: Balancing equations 
  \item 11: Net ionic equations
\end{itemize}

\section{Classifying Chemical Reactions}
\begin{itemize}
  \item We can categorize chemical reactions in several ways, but for this class we will focus on the types of chemical changes caused by a reaction
  \item Precipitation reactions form one or more solid ionic product from all aqueous ionic reactants
    \begin{itemize}
      \item First, predict the products by switching cations and anions
      \item Next, predict the phase by using solubility rules (Table 4.1)
      \item Identify and eliminate the spectator ions to get a net ionic equation
      \item Some combinations of reactants will have no reaction, and some may have two solid products
      \item More about precipitation reaction in CHEM 1220
    \end{itemize}
\end{itemize}

\paragraph*{Quiz 4.2 - Precipitation Reactions}
\paragraph*{Homework 4.2}
\begin{itemize}
  \item 28: Predicting products of reactions with ionic compounds
\end{itemize}

\paragraph*{Resuming Section 4.2: Classifying Chemical Reactions}
\begin{itemize}
  \item Acid-Base reactions involve the transfer of a \ch{H^+} between reactants
    \begin{itemize}
      \item Acids will react with water to produce hydronium ion
      \item Bases will react with water to produce hydroxide ions
      \item \emph{Strong} acids/bases react with water to completion, while \emph{weak} acids/bases only react partially (equilibrium)
      \item Table 4.2 lists six common strong acids
      \item Alternative definition: Acids are proton donors and bases are proton acceptors
      \item A neutralization reaction produces water and a salt (ionic compound)
      \item More about acid/base reactions in CHEM 1220
    \end{itemize}
  \item Redox reactions involve the transfer of one or more electrons
    \begin{itemize}
      \item Consdier the reaction: \ch{2 Na(s) + Cl2(g) -> 2 NaCl(s)}
      \item OIL RIG (or LEO says GER) to remember which is oxidation, and which is reduction
      \item Oxidizing agents and reducint agents. ``Agent'' describes the effect on the reaction partner
      \item Sometimes the electron transfer is not as obvious: \ch{CH4 + 2 O2 -> CO2 + 2 H2O}
      \item We assign oxidation numbers to keep track of electrons
        \begin{itemize}
          \item Elements have ox. \# of $0$
          \item Monoatomic ions have ox. \# equal to their charge
          \item Within compounds, O has ox. \# $=-2$, and H has ox \# $=+1$
          \item Oxidation \#s add up to the total overall charge (this rule subsumes the first 2)
        \end{itemize}
      \item Balancing redox reactions needs additional steps to account for the transferred electrons
        \begin{enumerate}
          \item Split the reaction into half-reactions (need to find oxidation \#s to do this)
          \item Balance all elements except H and O
          \item Add the electrons, based on the changes to oxidation \#s
          \item Balance charge by adding \ch{H^+} in acid, or \ch{OH^-} in base
          \item Balance H and O together by adding water
          \item Multiply half-reactions to balance their electrons, then add them together
          \item Make any cancellations of water, \ch{H+}, or \ch{OH^-}
        \end{enumerate}
      \item You can balance redox reactions as a whole instead of as half-reactions if you prefer
      \item Electrochemical series and predicting if redox reactions will be spontaneous or not (Not in the book!)
      \item More about Redox reaction in \ldots CHEM 1220!
    \end{itemize}
\end{itemize}

\paragraph*{Quiz 4.3 - Redox Reactions}
\paragraph*{Homework 4.3}
\begin{itemize}
  \item 17: Assigning oxidation states
  \item 19: Classify acid/base and redox reactions
  \item 39: Balancing half-reactions
\end{itemize}

\section{Reaction Stoichiometry}
\begin{itemize}
  \item Measured amounts should always be converted to moles in order to make comparisons to other chemical species in an equation
    \begin{itemize}
      \item We often want to calculate the corresponding amount of different chemicals in a chemical reaction
      \item Making direct comparisons will not work due to different molar masses, etc.
      \item Figure 4.11 shows how to convert to moles from different measurements
      \item My version of this figure (``Chemistry Land'')
    \end{itemize}
  \item Practice: \ch{N2(g) + 3 H2(g) -> 2 NH3(g)} Find other amounts for $0.75g$ of \ch{H2}
\end{itemize}

\paragraph*{Quiz 4.4 - Stoichiometry}
\paragraph*{Homework 4.4}
\begin{itemize}
  \item 49: Stoichiometry with solid reactants
  \item 57: Stoichiometry with aqueous reactions
\end{itemize}

\section{Reaction Yields}
\begin{itemize}
  \item Often one or more reactants are provided in excess, with only one \emph{limiting reactant}
    \begin{itemize}
      \item Pick a product, and calculate how much product would be produced for each reactant
      \item The lowest amount is the theoretical yield. The other amounts can be discarded
      \item The reactant which lead to the lowest amount is the limiting reactant
      \item Calculate the rest of the corresponding amounts starting from the limiting reactant
      \item Double-check your work using the conservation of mass
    \end{itemize}
  \item Practice: \ch{2 C4H10 + 13 O2 -> 8 CO2 + 10 H2O} with $5.00g\ch{C4H10}$ and $9.00g\ch{O2}$
  \item Reactions rarely go perfectly, and the actual amount of product will be different from the theoretical yield
  \item $\%yield = \dfrac{actual~yield}{theoretical~yield}\times 100\%$
  \item Yields below $100\%$ can mean that the reaction didn't go to completion, or some product was lost during a purification stage
  \item Yields above $100\%$ usually indicate that some contaminants (including perhaps excess reactant) are mixed in with the product
\end{itemize}

\paragraph*{Quiz 4.5 - Limiting Reactants}
\paragraph*{Homework 4.5}
\begin{itemize}
  \item 61: Limiting Reactant Problem
  \item 63: Percent Yield
\end{itemize}

\section{Quantitative Chemical Analysis}
\begin{itemize}
  \item Quantitative analysis is the family of techniques which determine the amount of substance in a sample
  \item Titrations find the concentration of an aqueous \emph{analyte}
    \begin{itemize}
      \item React the unknown with a co-reactant of known concentration by adding titrant with a buret
      \item The chemical environment will change ($pH$, redox potential, etc.) once the reaction reaches the equivalence point
      \item Probes or color indicators can show when to stop the titration. This is called the end-point, and is ideally very close to the equivalence point
      \item $\dfrac{C_TV_T}{\nu_T}=\dfrac{C_AV_A}{\nu_A}$
    \end{itemize}
  \item Gravimetric analysis uses a chemical reaction to change the phase of the analyte so it can be separated and weighed
    \begin{itemize}
      \item Your \% copper lab found the amount of copper by reducing it to a solid
      \item Your hydrates lab found the amount of water by weighing it before and after the baking step
    \end{itemize}
  \item Combustion analysis - We actually covered this earlier with \% composition of compounds
\end{itemize}

\paragraph*{Quiz 4.6 - Titrations}
\paragraph*{Homework 4.6}
\begin{itemize}
  \item 79: Acid/base Titration
  \item 81: Precipitation Titration
\end{itemize}

\chapter{Thermochemistry}

\section{Energy Basics}
\begin{itemize}
	\item Thermochemistry is the study of heat and energy changes in chemical reactions
	\item It also includes topics like entropy and spontaneity
	\item Energy can come in two forms, kinetic and potential (Figure 5.3)
    \begin{itemize}
      \item Kinetic energy is the energy of motion: $KE=\dfrac{1}{2}mv^2$
      \item For chemists, the kinetic energy that matters is the movement of individual atoms, molecules, etc., manifested as temperature (Figure 5.4)
      \item Heat will always flow from colder objects to warmer objects (Figure 5.6)
      \item Increases in temperature almost always cause objects to increase in volume (Figure 5.5)
      \item Potential energy is stored energy: Gravitaional, electrostatic, chemical, etc.
      \item For chemists, the potential energy that matters is the energy of chemical bonds and intermolecular forces (Figure 5.2)
    \end{itemize}
	\item Energy has several common units
    \begin{itemize}
      \item The SI unit is the Joule: $1~J=1~kg\dfrac{m^2}{s^2}$
      \item The calorie: $1~cal=4.184~J$
      \item The Calorie (kcalorie): $1~Cal=1000~cal=4184~J$
    \end{itemize}
  \item When we talk about changes and transfers of energy, we need to carefully define our system
    \begin{itemize}
      \item The \emph{system} is the part of the universe where the reaction occurs, such as a beaker or chamber with reactants
      \item The \emph{surroundings} is the rest of the universe
      \item Open systems can exchange both heat and matter with the surroundings (an open beaker)
      \item Closed systems can exchange heat, but not matter with the surroundings (a closed chamber)
      \item Isolated systems cannot exchange either heat or matter with the surroundings (a closed, insulated flask)
    \end{itemize}
	\item Heat from chemical and physical changes is usually associated with temperature changes or phase changes (Figure 5.7)
    \begin{itemize}
      \item These examples all refer to heat from the \emph{system's} perspective
      \item Positive heat is observed by an “upward” phase change or a cold temperature
      \item Holding ice in your hand will melt the ice, and make your hand cold
      \item Negative heat is observed by a “downward” phase change or a hot temperature
      \item Burning wood in a fire feels warm because of the negative system heat
    \end{itemize}
  \item Heat and temperature change are related by several equations (Note that we are always measuring the temperature of the \emph{surroundings})
    \begin{itemize}
      \item When heat is added to a system, it will either undergo a phase change, or heat up
      \item The specific heat gives how much heat is required to warm a given substance
      \item For a complete system, $q=C\Delta T$ where $C$ is the \emph{heat capacity}
      \item For a pure substance, $q=mc_S\Delta T$ where $c_S$ is the \emph{specific heat} of the substance and $m$ is the mass
      \item Table 5.1 shows $c_S$ for many common substances

        Practice: Find how much heat is required to raise the temperature of $15.0g$ of iron by $32^\circ C$

        Practice: Find the temperature change when $42.5J$ of heat are added to $0.374g$ of aluminum
    \end{itemize}
\end{itemize}

\paragraph*{Quiz 5.1 - Energy}
\paragraph*{Homework 5.1}
\begin{itemize}
  \item 5: Heat capacity of samples
  \item 9: $q=mc_S\Delta T$
  \item 11: $q=mc_S\Delta T$, but solve for $c_S$
\end{itemize}

\section{Calorimetry}
\begin{itemize}
  \item Whenever heat is transferred, the total energy of the universe remains constant
	\item For a heat transfer in an isolated system, $q_1=-q_2$
	\item Consider a block of hot metal placed in a beaker of room temperature water. Heat will flow from the block into the water until the two temperature are equal to each other
	\item $q_1=-q_2$ becomes $m_1c_1\left(T_f-T_{i,1}\right)=-m_2c_2\left(T_f-T_{i,2}\right)$

	      Practice: A $10.0~g$ block of iron is heated to $93.5~^\circ C$ and placed in $25~ml$ of $23.0~^\circ C$ water.\\
	      ~\hphantom{Practice:} What is the final temperature? ($25.9~^\circ C$)
	\item We can also measure the heat transfer associated with a chemical reaction, called \emph{calorimetry}
  \item Exothermic processes release heat into the surroundings (warm to the touch)
  \item Endothermic processes absorb heat from the surroundings (cool to the touch)
  \item Figure 5.11 shows how a calorimeter would work for endothermic and exothermic processes
  \item Figures 5.12 and 5.13 show a constant pressure (coffee-cup) calorimeter
    \begin{itemize}
      \item The reaction is carried out in aqueous solution
      \item $q_{rxn}$ is the heat released or absorbed by the reaction
      \item The heat of reaction is exchanged with the solution: $q_{rxn} = -q_{soln} = -mc\Delta T$
      \item $m$ is the solution mass, which will include the water and any solutes
      \item $c$ is the solution specific heat, but this is simplified by assuming $c_{soln}=c_{water}=4.184\frac{J}{g~^\circ C}$
      \item $\Delta H = \dfrac{q_{rxn}}{n_{rxn}} = \dfrac{-mc\Delta T}{n_{rxn}}$ where $n_{rxn}$ is the moles of reaction: $n_{rxn}=\left(\dfrac{n_A}{\nu_A}\right)$
      \item Demo -- \ch{NaOH} enthalpy of solvation $\left(10~g, 100~ml, –44.2~\frac{kJ}{mol}\right)$
    \end{itemize}
  \item Figure 5.17 shows a constant volume (bomb) calorimeter
    \begin{itemize}
      \item The reaction is carried out in a chamber charged with high pressure \ch{O2}
      \item $q_{rxn}$ is exchanged with the whole bomb-calorimeter apparatus
      \item The calorimeter is calibrated to give a \emph{heat capacity} ($C_{cal}$) with units $\frac{J}{^\circ C}$
      \item $q_{rxn}=-q_{cal}=-C_{cal}\Delta T$
      \item For constant volume, we measure $U$ instead of $H$ because $w=0$
      \item $\Delta U = \dfrac{-C_{cal}\Delta T}{n_{rxn}}$
    \end{itemize}
\end{itemize}

\paragraph*{Quiz 5.2 - Calorimetry}
\paragraph*{Homework 5.2}
\begin{itemize}
  \item 19: Reaching thermal equilibrium 
  \item 25: Coffee cup calorimetry
  \item 31: Bomb calorimetry
\end{itemize}

\section{Enthalpy}
\begin{itemize}
  \item The sum of all types of energy in a system is the \emph{internal energy}, $U$
  \item Any change in the internal energy must come from heat or work: $\Delta U=q+w$
  \item Work is defined as $w=f\cdot d$ or $w=-P\cdot\Delta V$
    \begin{itemize}
      \item Positive work is when the system volume decreases
      \item Negative work is when the system volume increases, or a force moves part of the surroundings
    \end{itemize}
	\item Heat is usually associated with temperature changes or phase changes, as discussed previously
	\item First Law of Thermodynamics: The energy of the universe is constant 
\end{itemize}

\chapter{Electronic Structure and Periodic Properties of Elements}

\section{Electromagnetic Energy}
% \begin{itemize}
%   \item 
% \end{itemize}

\section{The Bohr Model}
% \begin{itemize}
%   \item 
% \end{itemize}

\section{Development of Quantum Theory}
% \begin{itemize}
%   \item 
% \end{itemize}

\section{Electronic Structure of Atoms (Electron Configurations)}
% \begin{itemize}
%   \item 
% \end{itemize}

\section{Periodic Variations in Element Properties}
% \begin{itemize}
%   \item 
% \end{itemize}

\chapter{Chemical Bonding and Molecular Geometry}

\section{Ionic Bonding}
% \begin{itemize}
%   \item 
% \end{itemize}

\section{Covalent Bonding}
% \begin{itemize}
%   \item 
% \end{itemize}

\section{Lewis Symbols and Structures}
% \begin{itemize}
%   \item 
% \end{itemize}

\section{Formal Charges and Resonance}
% \begin{itemize}
%   \item 
% \end{itemize}

\section{Strengths of Ionic and Covalent Bonds}
% \begin{itemize}
%   \item 
% \end{itemize}

\section{Molecular Structure and Polarity}
% \begin{itemize}
%   \item 
% \end{itemize}

\chapter{Advanced Theories of Covalent Bonding}

\section{Valence Bond Theory}
% \begin{itemize}
%   \item 
% \end{itemize}

\section{Hybrid Atomic Orbitals}
% \begin{itemize}
%   \item 
% \end{itemize}

\section{Multiple Bonds}
% \begin{itemize}
%   \item 
% \end{itemize}

\section{Molecular Orbital Theory}
% \begin{itemize}
%   \item 
% \end{itemize}

\chapter{Gases}

\section{Gas Pressure}
% \begin{itemize}
%   \item 
% \end{itemize}

\section{Relating Pressure, Volume, Amount, and Temperature: The Ideal Gas Law}
% \begin{itemize}
%   \item 
% \end{itemize}

\section{Stoichiometry of Gaseous Substances, Mixtures, and Reactions}
% \begin{itemize}
%   \item 
% \end{itemize}

\section{Effusion and Diffusion of Gases}
% \begin{itemize}
%   \item 
% \end{itemize}

\section{The Kinetic-Molecular Theory}
% \begin{itemize}
%   \item 
% \end{itemize}

\section{Non-Ideal Gas Behavior}
% \begin{itemize}
%   \item 
% \end{itemize}

\chapter{Liquids and Solids}

\section{Intermolecular Forces}
\begin{itemize}
  \item Many physical properties of solids, liquids, and gases can be explained by the strength of attractive forces between particles (Figure 10.5)
  \item Phase changes happen due to the interplay between kinetic energy and intermolecular forces (Figure 10.2)
  \item Pressure can also play a role in phase changes, as discussed later
  \item These \emph{intermolecular forces} come in different varieties
  \begin{itemize}
    \item Dispersion Forces Non-polar molecules, impacted by polarizability, molecular weight, and surface area
    \begin{itemize}
      \item Dominant in non-polar molecules
      \item Created by induced dipoles (Figure 10.6)
      \item Impacted by polarizability (Table 10.1)
      \item Impacted by molecular weight (hydrocarbons from methane to wax)
      \item Impacted by molecule shape (Figure 10.7 compares the boiling points of pentane isomers)
    \end{itemize}
    \item Dipole-Dipole Forces
    \begin{itemize}
      \item Dominant in polar molecules
      \item Results from attraction between permanent dipoles (Figure 10.9)
    \end{itemize}
  \item Hydrogen Bonding
    \begin{itemize}
      \item Dominant only in molecules capable of hydrogen bonding
      \item Must contain a hydrogen-donor atom (H attached to N, O, or F)
      \item Must contain a hydrogen-acceptor atom (lone pair of electrons attached to N, O, or F)
      \item Hydrogen bonds are more than just particularly strong dipole-dipole forces. They have strong directionality according to VSEPR
      \item Figures 10.10, 10.14, and other figures on the Internet show water, DNA, and proteins all organized by hydrogen bonds
      \item Figures 10.11 and 10.12 illustrate how much hydrogen bonds exceed dipole-dipole forces in strength
    \end{itemize}
  \end{itemize}
\end{itemize}
\backmatter
\chapter{Errata}
\end{document}
