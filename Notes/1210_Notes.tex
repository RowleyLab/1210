\documentclass[12pt, openany, letterpaper]{memoir}
\usepackage{NotesStyle}
%\renewcommand\thesection{\thechapter\Alph{section}}
%\renewcommand\thesubsection{\thesection.\Numeral{subsection}}

\begin{document}
\title{CHEM 1210 Lecture Notes\\ OpenStax Chemistry 2e}
\author{Matthew Rowley}
\mainmatter
\maketitle
\chapter*{Course Administrative Details}
\begin{itemize}
	\item My office hours
	\item Intro to my research
	\item Introductory Quiz
	\item Grading details
	      \begin{itemize}
		      \item Exams - 40, Final - 20, Online Homework - 15, Book Homework - 15, Quizzes - 10
		      \item Online homework
		      \item Frequent quizzes
	      \end{itemize}
	\item Importance of reading and learning on your own
	\item Learning resources
	      \begin{itemize}
		      \item My Office Hours
		      \item Tutoring services - \href{https://www.suu.edu/academicsuccess/tutoring/}{https://www.suu.edu/academicsuccess/tutoring/}
	      \end{itemize}
	\item Show how to access Canvas
	      \begin{itemize}
		      \item Calendar, Grades, Modules, etc.
		      \item Quizzes
		      \item Textbook
	      \end{itemize}
	\item Introduction to chemistry
	      \begin{itemize}
		      \item Ruby fluorescence
		      \item Levomethamphetamine
          \item Submerged salt crystals grow due dynamic equilibrium
		      \item Rubber band elasticity
		      \item Structure of the periodic table
		      \item Salt on ice and purifying hydrogen peroxide
	      \end{itemize}
\end{itemize}

\chapter{Essential Ideas}

\section{Chemistry in Context}
\begin{itemize}
  \item Modern chemistry is the end result of thousands of years of humans trying to explain and control the materials around them
  \item Early forays into chemistry (such as alchemy) had deep mystical roots and often relied on serendipity to make good progress
  \item Modern chemistry is a rigorous science, relying on falsifiability and the scientific methd (Figure 1.4)
  \item We sometimes refer to chemistry as ``The Central Science'' (Figure 1.3)
  \item To adequately describe and understand chemical phenomena, we often talk from different perspectives
  \begin{description}
    \item[Macroscopic Domain] This is what we observe with bulk substances. Two chemicals react to produce a new chemical
    \item[Microscopic Domain] We now understand that all microscopic effects are governed by the behavior of \emph{microscopic} actors (molecules, atoms, electrons, etc.)
    \item[Symbolic Domain] Effectively communicating chemical ideas requires new language. Chemical formulas, equations, and mechanisms are all symbolic representations
    \item All three domains are on display in Figure 1.5
  \end{description}
\end{itemize}
\paragraph*{Quiz 1.1 - Scientific Method}
\paragraph*{Homework 1.1}
\begin{itemize}
  \item 1.1: Thinking in terms of Chemistry
  \item 1.3: The scientific method
  \item 1.5: Domains of inquiry
\end{itemize}

\section{Phases and Classification of Matter}
\begin{itemize}
  \item Three primary phases of matter are shown in Figure 1.5 (and 1.6)
  \item Plasmas are like a gas, but with electrically charged particles
  \item Mass vs Weight (for very fine measurements, the difference matters even on Earth due to buoyancy)
  \item Figure 1.8 illustrates the \emph{law of conservation of matter}
  \item Classifying matter (Figure 1.11)
  \begin{itemize}
    \item Pure Substances
    \begin{itemize}
      \item Elements (Anything on the \emph{periodic table of the elements})
      \item Compounds (Combinations of elements -- can have very different properties from their constituent elements)
    \end{itemize}
    \item Mixtures
    \begin{itemize}
      \item Heterogeneous mixtures (variable composition)
      \item Homogeneous mixtures (i.e. solutions, continuous composition)
    \end{itemize}
  \end{itemize}
  \item Table 1.1 shows the abundance of many elements on Earth
  \item Atoms are the smallest particle of an element that has the properties of that element
  \begin{itemize}
    \item Thought-experiment of dividing a sample in half ad-infinitum
    \item Ancient atomic theories and modern Dalton atomic theory (discussed in detail later)
    \item Atoms are \emph{very} small; smaller than we could even detect until recently
  \end{itemize}
  \item Molecules are collections of atoms held together with chemical bonds (more nuanced definition later)
    \begin{itemize}
      \item Many elements occur naturally as molecules, rather than atoms
      \item Figure 1.14 shows many molecular elements and compounds
    \end{itemize}
\end{itemize}

\section{Physical and Chemical Properties}
\begin{itemize}
  \item Physical Properties: Properties which can be observed without changing the chemical identity of the substance
  \item Chemical Properties: Properties which can only be observed through chemical reactions (e.g. flammability, acidity, electrochemical potential, etc.)
  \item Physical Changes: Any change which perserves the \emph{chemical identity} of the substance (including phase changes)
  \item Chemical Changes: Changes which alter the chemical identities of one of more substance
  \item Extensive Properties: Depend on the size of hte system (double the size, double the property measurement, such as mass or volume)
  \item Intensive Properties: Independent of system size (density, temperature, most chemical properties)
  \item The periodic table groups elements according to their properties (Figure 1.22)
  \begin{itemize}
    \item Metals conduct electricity and heat, are maleable and ductile
    \item Non-metals are very diverse, but generally poor conductors
    \item Metalloids exist at the boundary and share properties with both metals and non-metals
    \item There are many other ways to group the elements, which we will learn later
  \end{itemize}
\end{itemize}
\paragraph*{Quiz 1.2 - Matter, Properties, and Change}
\paragraph*{Homework 1.2}
\begin{itemize}
  \item 1.17: Classifying matter
  \item 1.27: Classifying changes
\end{itemize}

\section{Measurements}
\begin{itemize}
  \item All measurements are composed of three parts:
  \begin{itemize}
    \item The magnitude of the measurement (the number itself)
    \item The unit of measurement used (g, kg, lbs, etc.)
    \item The degree of uncertainty in the measurement (this is usually implicit, and covered in the next section)
  \end{itemize}
  \item Units are an essential part of any measuement. Develop a habit of \emph{always} including units in your work
  \begin{itemize}
    \item $u_{rms}=\sqrt{\dfrac{3RT}{M}}$ -- example of how units can guide problem solving and ``unit purgatory''
    \item SI units are a collection of fundamental units from which all other units can be derived (Table 1.2)
    \item Metric prefixes make it more convenient to discuss very large or very small numbers (Table 1.3)
    \item Scientific notation is an even more general and robust way of representing numbers
    \begin{itemize}
      \item The quantity is represented by a number with the decimal after the first digit
      \item The magnitude is represented by a power of $10$
    \end{itemize}
    \item Practice converting between normal numbers, metric prefixes, and scientific notation
    \item For temperature, we use both $K$ and $^\circ C$ (But not $^\circ F$)
      \\$T(K) = T(^\circ C) + 273.15$ 
    \item Derived units will combine the fundamental units in some way 
      \\volume: $m^3$, $L$, $ml$
      \\velocity: $\nicefrac{m}{s}$ 
      \\density: $\nicefrac{kg}{m^3}$, $\nicefrac{g}{cm^3}$ (Table 1.4)
      \\energy: $1J\equiv \nicefrac{kgm^2}{s^2}$
  \end{itemize} 
\end{itemize}

\section{Measurement Uncertainty, Accuracy, and Precision}
\begin{itemize}
  \item Countable quantities are considered to be \emph{exact} (no uncertainty)
  \item Measurements (and groups of measurements) always have some degree of undertainty
  \begin{itemize}
    \item Accuracy is how close a measurement is to the \emph{true value} (usually unknown, but approximated by calibration with a well-known standard)
    \item Precision is how finely a measurment is made (What is the margin of error)
    \item Figure 1.27 and Table 1.5 illustrate the differences between precision and accuracy
    \item Accuracy is usually improved through calibration, and moving forward we will usually assume that measurements are as accurate as an instrument allows
    \item Precision is represented in the way we write the number, and can be improved with a better instrument or with repeat measurements
  \end{itemize}
  \item Significant figures are the way that we represent precision in a number
  \begin{itemize}
    \item The number of digits conveys the degree of precision
    \item Example of me saying I'm $6ft~2in$ tall, vs me saying I'm $6ft~1.6241434in$ tall
    \item For graduated measurements, we record one digit beyond the lowest graduation (Figure 1.26)
    \item For digital measurements, we record the number as it is given by the instrument
    \item For any given number, we should track both the \emph{quantity} of significant figures, and the \emph{position} of the least-significant digit
    \item In a written number, digits are considered significant according to the following rules:
    \begin{itemize}
      \item All non-zeros are significant
      \item All \emph{captive} zeros (between two other significant digits) are significant
      \item Trailing zeros are \emph{always} significant
      \item Leading zeros are \emph{never} significant
      \item For scientific notation, only the digits of the quantity (not the magnitude) count
      \item Logarithmic quantities follow different rules which we will revisit in CHEM 1220 (chapter 14)
      \item Note that for some numbers scientific notation is \emph{required} to convey the correct precision ($3.0\times10^3m$)
    \end{itemize}
  \end{itemize}
  \item Errors propogate when multiple measurements are used in a mathematical operation
  \begin{itemize}
    \item For addition and subtraction, the least significant digit of the answer will be in the same position as the least significant digit of hte least precise input
    \item For multiplicationa and division, the quantity of significant digits in the answer will match the quantity of significant digits of the input with fewest significant digits
    \item When rounding an exact $5$ (no further digits beyond the $5$), round up or down to make the last digit even
    \item Compound problems involve multiple types of operations
    \begin{itemize}
      \item Solve the problem in steps, applying the correct rule to each step
      \item Track the significant figures (quantity and position) for each intermediate answer, but do \emph{not} truncate or round any of these answers
      \item Only round after the last step
        \\ $\circ$ Practice $\frac{12.3g+34g}{12.0cm^3+7.7cm^3}=2.4\nicefrac{g}{cm^3}$ (wrong answer with premature rounding)
    \end{itemize}
  \end{itemize}
\end{itemize}
\paragraph*{Quiz 1.3 - Significant Figures}

\section{Mathematical Treatment of Measurement Results}
\begin{itemize}
  \item Some quantities are calculated based on two or more measurements (such as velocity and density)
  \item These formulas can be used to relate all three quantities together (i.e. $velocity = \frac{distance}{time}$)
  \item The derived quantity can be interpreted as a \emph{comversion factor} between the other two quantitites
  \item Conversion factors and unit conversions 
  \begin{itemize}
    \item Elementary school perspective of $ft$ to $in$ conversions
    \item Conversion factors are a ratio between two identical quantities
    \item Converting units involves multiplying by $1$ in the form of a conversion factor
    \item Units guide the problem solving
  \end{itemize}
  \item Dimensional Analysis is a problem-solving framework based on a series of unit conversions
  \begin{itemize}
    \item Don't dive straight into calculations and equations
    \item Identify the units you expect for the answer
    \item Identify the starting point
    \item Create a plan to convert units from the starting point to the answer
    \item Carry out the calculations
    \item Practice converting $65.0\nicefrac{miles}{hour}$ into $\nicefrac{m}{s}$
    \item The ``railroad ties'' or ``picket fence'' method can help organize your work
  \end{itemize}
  \item Dimensional analysis is not the only way to solve problems, but it is versatile and robust; usually my preferred choice
  \item Practice a more abstract problem:
    \\ Find the $\nicefrac{miles}{gal}$ if a car consumes $8036~g$ of gasoline while driving for $40.0~min$ at $75~\nicefrac{miles}{hour}$
\end{itemize}
\paragraph*{Quiz 1.4 - Dimensional Analysis}

\chapter{Atoms, Molecules, and Ions}

\section{Early Ideas in Atomic Theory}
% \begin{itemize}
%   \item 
% \end{itemize}

\section{Evolution of Atomic Theory}
% \begin{itemize}
%   \item 
% \end{itemize}

\section{Atomic Structure and Symbolism}
% \begin{itemize}
%   \item 
% \end{itemize}

\section{Chemical Formulas}
% \begin{itemize}
%   \item 
% \end{itemize}

\section{The Periodic Table}
% \begin{itemize}
%   \item 
% \end{itemize}

\section{Ionic and Molecular Compounds}
% \begin{itemize}
%   \item 
% \end{itemize}

\section{Chemical Nomenclature}
% \begin{itemize}
%   \item 
% \end{itemize}

\chapter{Composition of Substances and Solutions}

\section{Formula Mass and the Mole Concept}

\section{Determining Empirical and Molecular Formulas}

\section{Molarity}

\section{Other Units for Solution Concentration}

\chapter{Stoichiometry of Chemical Reactions}

\section{Writing and Balancing Chemical Equations}

\section{Classifying Chemical Reactions}

\section{Reaction Stoichiometry}

\section{Reaction Yields}

\section{Quantitative Chemical Analysis}

\chapter{Thermochemistry}

\section{Energy Basics}

\section{Calorimetry}

\section{Enthalpy}

\chapter{Electronic Structure and Periodic Properties of Elements}

\section{Electromagnetic Energy}

\section{The Bohr Model}

\section{Development of Quantum Theory}

\section{Electronic Structure of Atoms (Electron Configurations)}

\section{Periodic Variations in Element Properties}

\chapter{Chemical Bonding and Molecular Geometry}

\section{Ionic Bonding}

\section{Covalent Bonding}

\section{Lewis Symbols and Structures}

\section{Formal Charges and Resonance}

\section{Strengths of Ionic and Covalent Bonds}

\section{Molecular Structure and Polarity}

\chapter{Advanced Theories of Covalent Bonding}

\section{Valence Bond Theory}

\section{Hybrid Atomic Orbitals}

\section{Multiple Bonds}

\section{Molecular Orbital Theory}

\chapter{Gases}

\section{Gas Pressure}

\section{Relating Pressure, Volume, Amount, and Temperature: The Ideal Gas Law}

\section{Stoichiometry of Gaseous Substances, Mixtures, and Reactions}

\section{Effusion and Diffusion of Gases}

\section{The Kinetic-Molecular Theory}

\section{Non-Ideal Gas Behavior}

\chapter{Liquids and Solids}

\section{Intermolecular Forces}
\begin{itemize}
  \item Many physical properties of solids, liquids, and gases can be explained by the strength of attractive forces between particles (Figure 10.5)
  \item Phase changes happen due to the interplay between kinetic energy and intermolecular forces (Figure 10.2)
  \item Pressure can also play a role in phase changes, as discussed later
  \item These \emph{intermolecular forces} come in different varieties
  \begin{itemize}
    \item Dispersion Forces Non-polar molecules, impacted by polarizability, molecular weight, and surface area
    \begin{itemize}
      \item Dominant in non-polar molecules
      \item Created by induced dipoles (Figure 10.6)
      \item Impacted by polarizability (Table 10.1)
      \item Impacted by molecular weight (hydrocarbons from methane to wax)
      \item Impacted by molecule shape (Figure 10.7 compares the boiling points of pentane isomers)
    \end{itemize}
    \item Dipole-Dipole Forces
    \begin{itemize}
      \item Dominant in polar molecules
      \item Results from attraction between permanent dipoles (Figure 10.9)
    \end{itemize}
  \item Hydrogen Bonding
    \begin{itemize}
      \item Dominant only in molecules capable of hydrogen bonding
      \item Must contain a hydrogen-donor atom (H attached to N, O, or F)
      \item Must contain a hydrogen-acceptor atom (lone pair of electrons attached to N, O, or F)
      \item Hydrogen bonds are more than just particularly strong dipole-dipole forces. They have strong directionality according to VSEPR
      \item Figures 10.10, 10.14, and other figures on the Internet show water, DNA, and proteins all organized by hydrogen bonds
      \item Figures 10.11 and 10.12 illustrate how much hydrogen bonds exceed dipole-dipole forces in strength
    \end{itemize}
  \end{itemize}
\end{itemize}
\backmatter
\chapter{Errata}
\end{document}
