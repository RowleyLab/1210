\documentclass[11pt, openany, letterpaper]{memoir}
\usepackage{NotesStyle}
%\renewcommand\thesection{\thechapter\Alph{section}}
%\renewcommand\thesubsection{\thesection.\Numeral{subsection}}

\begin{document}
\title{CHEM 1210 Lecture Notes}
\author{Matthew Rowley}
\mainmatter
\maketitle
\chapter*{Course Administrative Details}
\begin{itemize}
	\item My office hours
	\item Intro to my research
	\item Introductory Quiz
	\item Grading details
	      \begin{itemize}
		      \item Exams - 40, Final - 20, Homework - 30, Quizzes - 10
		      \item Online homework
		      \item Frequent quizzes
	      \end{itemize}
	\item Importance of reading and learning on your own
	\item Learning resources
	      \begin{itemize}
		      \item My Office Hours
		      \item Tutoring services - \href{https://www.suu.edu/academicsuccess/tutoring/}{https://www.suu.edu/academicsuccess/tutoring/}
	      \end{itemize}
	\item Show how to access Canvas
	      \begin{itemize}
		      \item Calendar, Grades, Modules, etc.
		      \item Achieve Homework
		      \item Textbook
	      \end{itemize}
	\item Algebra Review Assignment 0
	\item Introduction to chemistry
	      \begin{itemize}
		      \item Ruby fluorescence
		      \item Levomethamphetamine
		      \item Rubber band elasticity
		      \item Structure of the periodic table
		      \item Salt on ice and purifying hydrogen peroxide
	      \end{itemize}
\end{itemize}
\chapter{Science and Measurement}
\section{Classification of Matter}
\begin{itemize}
	\item Matter is anything that has mass and occupies space
	\item \emph{Elements} are the basic building blocks of the stable matter around us
	\item Atoms are the smallest, indivisible pieces of matter (demo with a copper penny)
	\item Atoms bond together in chemical bonds (more on them later)
	\item When 2 or more elements combine in a definite ratio, they form \emph{compounds}
	      \begin{itemize}
		      \item Definite proportions and the analogy with bread recipes
		      \item Chemical compounds are different from their elements (drowning in water - 89\% Oxygen)
	      \end{itemize}
	\item Pure substances vs mixtures (Figure 1.1)
	\item Heterogeneous vs homogeneous mixtures (Figure 1.2)
  \item ``Solution'' is just another name for an homogeneous mixture
\end{itemize}
\section{Properties of Matter}
\begin{itemize}
	\item The first categories of properties are \emph{physical} vs \emph{chemical}
	      \begin{itemize}
		      \item Physical properties can be measured or observed without changing the \emph{chemical identity}
		      \item Chemical properties can only be measured or observed by changing the \emph{chemical identity}
		      \item Mass, density, color (absorption spectrum), specific heat, solubility, etc. are physical
		      \item Flammability, oxidization potential, acidity/basicity, corrosiveness, etc. are chemical
	      \end{itemize}
	\item The next categories are \emph{extensive} vs \emph{intensive} properties
	      \begin{itemize}
		      \item Extensive properties depend on the amount of a substance -- mass, volume, enthalpy of combustion, heat capacity, etc.
		      \item Intensive properties are independent of the amount of a substance -- density, temperature, reactivity, specific heat, etc.
	      \end{itemize}
	\item Finally, we categorize \emph{physical changes} and \emph{chemical changes}
	      \begin{itemize}
		      \item Phsyical changes do not change the chemical identity of a substance, and can readily be undone - dissolving, phase changes, deformations, temperature changes, etc.
		      \item Chemical changes cannot be undone because the chemical identity of the substance has changed - oxidation, combustion, corrosion, etc.
          \item Chemical changes are accompanied by either the release or absorption of energy
          \item Starting materials in chemical changes are called \emph{reactants}, and the final materials are called \emph{products}
	      \end{itemize}
  \item Demo with air and hydrogen balloons
\end{itemize}
\section{Matter and Energy}
\begin{itemize}
	\item All chemical phenomena involve changes in \emph{matter} and/or \emph{energy}
	\item Mass is the amount of substance. Contrasted with weight, mass doesn't change due to gravity
	\item Energy is the capacity to do work
	      \begin{itemize}
		      \item Forms of energy: Heat, Chemical, Nuclear, Kinetic, Potential, Electrical, Sound, Light
		      \item Energy can be transferred or change form, but total energy is always conserved
	      \end{itemize}
\end{itemize}
\section{The Scientific Method}
\begin{itemize}
	\item The scientific method is an orderly process for determining information about the natural world (Figure 1.8)
	      \begin{itemize}
		      \item Observation -- Something about the natural wold inspires a question
		      \item Hypothesis -- The scientist suggests a plausible answer or explanation for the observations
		      \item Experiment -- A carefully constructed experiment will test the validity of the hypothesis
		            \begin{itemize}
			            \item Results inconsistent with the hypothesis force a revision, and retesting of the new hypothesis
			            \item Results consistent with the hypothesis suggest the hypothesis has merit
		            \end{itemize}
		      \item Results should be corroborated by other scientists in independent experiments
		      \item A well-supported hypothesis may become a \emph{scientific theory}
	      \end{itemize}
	\item The scientific method is based on the principle of \emph{falsifiability}. It proves bad hypothese false, but can never \emph{prove} a hypothesis to be true
  \item As a hypothesis survives more and more experimentation, it becomes relieable enough to be considered a \emph{theory}. Theories can later be proven wrong, but the new explanation is almost always a marginal extension of the older theory, which better explains certain edge cases
  \item Entirely revolutionary new theories, such as Einstein's general relativity, are exceptionally rare
	\item Figure 1.9 illustrates the differences between scientific laws, hypotheses, and theories
	\item My story about magnets on string and the importance of following the scientific method
	      \begin{itemize}
		      \item Why can't we just make conclusions based on observations (as opposed to experiments)?
		      \item How was my bias displayed in this story?
	      \end{itemize}
\end{itemize}
\section{The International System of Units}
\begin{itemize}
	\item There are often many different units to describe the same quantity (quart, liters, cubic centimeters, etc.)
	\item The International System of Units is a standardized set of units used by scientists globally
	\item Base units are the fundamental units that can describe everything we measure (Table 1.3)
	      \begin{itemize}
		      \item Length - m
		      \item Mass - kg (note that the SI unit is \emph{not} the gram)
		      \item Time - s
		      \item Electrical Current - A
		      \item Temperature - K
		      \item Luminous Intensity - cd
		      \item Countable Amount - mol
	      \end{itemize}
	\item Large or small amounts can be described using metric prefixes (Table 1.4)
	      \begin{itemize}
		      \item These prefixes are mostly based on powers of $10^3$
		      \item You must know the prefixes and their abbreviations in Table 1.4
		      \item Introduction to simple conversions and scientific notation
	      \end{itemize}
	\item Derived units describe quantities that combine multiple base units together
	      \begin{itemize}
		      \item Volume can be $l\times w\times h$, or generalized to any shape
		      \item Density is $\frac{mass}{volume}$
		      \item Some derived units are given a new name: $J=\frac{kgm^2}{s^2}$
	      \end{itemize}
	\item When using the SI units, conversions tend to be easy (even for complex derived units)
	\item How much energy is required to accelerate a $1.75~kg$ object by $0.650~\nicefrac{m}{s^2}$ over a distance of $3.20~m$? ($3.64~J$)
\end{itemize}
\section{Significant Digits}
\begin{itemize}
	\item Qualitative vs quantitative descriptions
	\item For quantitative descriptions, uncertainty comes in the form of \emph{accuracy} and \emph{precision}
	      \begin{itemize}
		      \item Accuracy refers to the average of a set of measurements corresponding to the true value (within a standard deviation)
		      \item If inaccuracies are known, then they can be subtracted by calibration
		      \item Precision refers to the standard deviation of a set of measurements on the same sample
		      \item Figure 1.13 shows accuracy vs. precision on a dart board
		      \item The precision of your tool should match the magnitude of the quantity to measure
		      \item Estimate the position between the smallest graduation on marked rulers, burettes, etc.
	      \end{itemize}
	\item Precision is communicated through \emph{significant figures}
	\item Consider measuring something $1.5~cm$ vs $1.4973~cm$
	\item Now consider $1.5~cm$ vs $1.5000~cm$
	\item Identifying significant figures
	      \begin{itemize}
		      \item Any non-zero digits are significant
		      \item Trailing zeros are significant if they come \emph{after a decimal}
		      \item Zeros between two significant figures are also significant
		      \item Leading zeros are \emph{never} significant
		      \item Practice: $4010$, $0.0034$, $7.100$, $639,000$, $6.390\times10^{5}$
	      \end{itemize}
	\item Scientific Notation not only makes it convenient to write very large or very small numbers, but it also communicates precision without any ambiguity
	\item Propagating sig. figs through calculations
	      \begin{itemize}
		      \item Identify the \# of sig figs and the position of the least significant digit for both numbers
		      \item $+-$: The LSD of the answer will match the position of the LSD for the less precise input
		      \item Practice: $120.7~g + 34~g = 155~g$ \hspace{1em}|\hspace{1em} $212~mm - 210.95~mm = 1~mm$
		      \item $\times/$: The answer will have as many sig. figs as the input with \emph{fewer} sig figs.
		      \item Practice: $56.3~miles / 1.2~h = 47~\nicefrac{miles}{h}$ \hspace{1em}|\hspace{1em} $1.5~cm\times3.62~cm = 5.4~cm^2$
	      \end{itemize}
	\item Multi-step problems have two considerations:
	      \begin{itemize}
		      \item Do not round intermediate answers to avoid compounding rounding errors
		      \item Keep track of both the \# of sig. figs and the LSD for each intermediate answer
		      \item Practice: $(0.0045\times20,000.0)+(2813\times12)=34,000$

		            ~\hphantom{Practice:} $863\times\left[1255-\left(3.45\times108\right)\right]=762,000$
	      \end{itemize}
\end{itemize}
\section{Dimensional Analysis}
\begin{itemize}
	\item Every measurement will have its associated \emph{unit(s)}, which is just as important as the quantity itself
	\item For any calculations, the units can provide a guide and a check on your process
	\item Structuring your calculations around the units is a paradigm called \emph{dimensional analysis}
	      \begin{itemize}
		      \item Treat units like algebraic quantities, which can cancel in calculations
		      \item Conversions factors can be written as ratios. i.e. $1.00~in = 2.54~cm$ becomes $1=\dfrac{2.54~cm}{1.00~in}$
		      \item Consider both the starting and ending units, and how to convert from one to the other
		      \item Solve the problem with the ``picket fence'' method
		      \item Practice: How many $m$ are in $1.00~ft$? ($0.305~m$)

		            ~\hphantom{Practice:} How many \textcent~ does someone make per second if their wage is $\dfrac{\$12.00}{h}$? ($0.3333\nicefrac{c}{s}$)
	      \end{itemize}
	\item Be careful with squared or cubed units! Convert $0.05~m^3$ into $cm^3$ ($50,000~cm^3$)
\end{itemize}
\section{Density}
\begin{itemize}
	\item Density relates how much mass is in a given volume, and is responsible for buoyant forces, etc.
	\item Table 1.7 gives the density of some common substances (add in $\approx 0.001~\nicefrac{g}{ml}$ for gasses)
	\item $density=\dfrac{mass}{volume}$
	\item Volumes can be found by $l\times w\times h$, or by measuring the volume of displaced water
	\item Practice: Find the density of a block with dimensions $1.5~cm\times6.4~cm\times13.2~cm$, and mass of 	~\hphantom{Practice:} $1.43~kg$. Can guess the material? ($11~\nicefrac{g}{ml}$ -- Lead)

	      ~\hphantom{Practice:} A gold nugget displaces $4.52~ml$ of water. What is its mass? ($87.2~g$)
\end{itemize}
\section{Temperature Scales}
\begin{itemize}
	\item There are three common temperature scales, Fahrenheit, Celsius, and Kelvin
	\item Kelvin is essential to use in some formulas, but Celsius may be convenient at other times
	\item $T_{^\circ C}=\dfrac{5}{9}\left(T_F-32\right)$

	      $T_F=\dfrac{9}{5}T_{^\circ C}+32$

	      $T_K=T_{^\circ C}+273.15$

	\item Practice: Convert the current temperature from Fahrenheit to Celsius and Kelvin
\end{itemize}

\chapter{Atoms and the Periodic Table}
\section{Chemical Symbols}
\begin{itemize}
	\item Each element has a Symbol, which starts with a capital letter
	\item \emph{Chemical Formulas} use the symbols and subscripts to show how many of each element are in a compound
	\item Parenthesis show atom groups: How many oxygen atoms are in \ch{Mg(ClO3)2}
\end{itemize}
\section{The Laws of Chemical Combination}
\begin{itemize}
	\item Careful experiments show reactions follow several laws
	\item The Law of Conservation of Mass
	\item The Law of Definite Proportions (For water, $1.00~g$ \ch{O} combines with $0.126~g$ of \ch{H})
	\item The Law of Multiple Proportions (For \ch{H2O2}, $1.00~g$ of \ch{O} combines with $0.063~g$ of \ch{H})

	      Possible ratios of \ch{N}:\ch{O} $1$:$1.14$ (\ch{NO}), $1$:$0.57$ (\ch{N2O}), $1$:$2.29$ \ch{NO2}
\end{itemize}
\section{The History of the Atom}
\begin{itemize}
	\item Based on these laws, Dalton formulated an atomic theory:
	      \begin{itemize}
		      \item Matter is made up of atoms
		      \item Atoms of the same element are identical to each other. Atoms of different elements are different from each other
		      \item Atoms combine to form molecules in whole-number ratios
		      \item Different ratios of the elements give different chemical compounds
	      \end{itemize}
	\item J.J. Thompson discovered the electron in 1897
	      \begin{itemize}
		      \item His cathode ray experiment showed that electrons are negatively charged
		      \item Thompson found the $\dfrac{mass}{charge}$ ratio, and Millikan later found the electron mass itself
		      \item Electrons were a component of atoms, but were first assumed to be distributed evenly through the atom
	      \end{itemize}
	\item Ernest Rutherford discovered the nucleus in 1909
	      \begin{itemize}
		      \item Gold foil experimental details (firing a bullet into jello, vs. into a chainlink fence)
		      \item The mass and positive charge of an atom are densely packed into the nucleus
		      \item Electrons exist diffusely outside of the nucleus (originally in orbits, now in clouds)
		      \item Atoms are mostly empty space (atom is $\approx 1$ \AA ~($100,000~fm$) across, and nucleous is $\approx 1~fm$ across )
	      \end{itemize}
	\item Later, neutrons were also discovered, and are a part of the nucleus
\end{itemize}
\section{Subatomic Particles, Isotopes, and Ions}
\begin{itemize}
	\item The important subatomic particles are: electrons, protons, and neutrons (table 2.2)
	\item For neutral atoms, $e = p$, but $n$ could vary
	\item The identity of an element depends entirely on $p$, so we call it the \emph{atomic number} ($Z$)
	\item The sum $p+n$ is the \emph{mass number} ($A$)
	\item Atoms of the same element with different mass numbers are \emph{isotopes}

	      Practice: What is $n$ for carbon with $A=14$? \hspace{1em}|\hspace{1em} What is $n$ for lead with $A=207$?
	\item The difference $p-e$ is the \emph{charge} ($Q$). Atoms with charge $\neq 0$ are called \emph{ions}
	\item Chemical symbols can show all these values like so: $^A_ZX^Q$
\end{itemize}
\section{Atomic Masses}
\begin{itemize}
	\item The mass of a single atom is too small to conveniently work with, so we use \emph{atomic mass units} (AMUs)
	\item The AMU is defined as $\dfrac{1}{12}M_{^{12}\ch{C}}$, and is $1.00~g=6.022\times10^{23}~AMU$
	\item The mass of a single proton or a single neutron are about $1~AMU$. An electron is about $\dfrac{1}{1800}~AMU$
	\item Individual isotopes have an actual, measured mass similar to their mass number
	\item The masses we see on the periodic table are a weighted average, based on the isotope abundances
	\item $Atomic~Mass=\sum_{isotopes}(Fractional~Abundance)\times(Isotope~Mass)$

	      Practice: Give the atomic mass for Copper with: \begin{tabular}{l|c|c}
		                     & \ch{^{63}Cu} & \ch{^{65}Cu} \\ \midrule
		      Mass (amu)     & $62.929601$  & $64.927794$  \\
		      Abundance (\%) & $69.17\%$    & $30.83\%$
	      \end{tabular}
\end{itemize}
\section{The Periodic Table}
\begin{itemize}
	\item As the number of known elements grew, scientists began to organize them into categories
	\item \ch{F2}, \ch{Cl2}, \ch{Br2}, and \ch{I2} are highly reactive gases. \ch{Li}, \ch{Na}, \ch{K}, and \ch{Rb} are highly reactive metals
	\item Arranging the elements according to reactive trends \emph{and} size essentially gives us the periodic table
	\item Dmitri Mendeleev is considered the father of the periodic table because he used his table to predict the existence and properties of undiscovered elements
	\item Gallium and Germanium were later discovered to have properties very close to what Mendeleev predicted
	\item Now the periodic table is arranged by atomic number, rather than by mass
	\item We have discovered all the elements up to 118, which may be the only stable elements at all
	\item The periodic table is tremendously useful and information-rich in many ways which we will highlight in most chapters of the textbook
	\item Structure of the periodic table:
	      \begin{itemize}
		      \item The periodic table is really long, but we cut out a portion to fit it on a page
		      \item Rows are called \emph{periods} (but I will just call them rows)
		      \item Columns are called \emph{groups} or \emph{families}
		      \item Groups are labeled by two conventions: numbered $1$ -- $18$, or $1A$ -- $8A$
		      \item Metals and non-metals are separated by the metalloids (Figure 2.19)
		      \item Main group, transition metals, and inner-transition metals (Figure 2.18)
		      \item Some families are named: Alkali Metals, Alkaline Earth Metals, Coinage Metals, Halogens, Noble Gases, and others you don't need to know (Figure 2.17)
	      \end{itemize}
\end{itemize}

\chapter{Compounds and the Mole}
\section{Chemical Formulas}
\begin{itemize}
	\item We have already discussed how \emph{Chemical Formulas} give the \# and type of atoms in a compound
	\item Another word for chemical formula is \emph{Formula Unit}, which is especially used for ionic compounds
	\item Molecular compounds are compounds composed of non-metals bonded together
	      \begin{itemize}
		      \item One formula unit is a \emph{molecule}, which is a discrete particle with clear boundaries
		      \item Atoms in a molecular compound are held together by \emph{covalent bonds} -- more on them in chapter 10
	      \end{itemize}
	\item Ionic compounds are compounds composed of two ions bound together
	      \begin{itemize}
		      \item The simplest ionic compounds are a metal bonded to a non-metal
		      \item Ionic compounds form bonds in a large array called a \emph{lattice}
		      \item There is no clear boundary for a single formula unit, the lattice structure just goes on in all directions
		      \item Different compounds will have different lattice structures
		      \item These atoms are held together by \emph{ionic bonds}, which are simply the attraction between negative and positive charges
	      \end{itemize}
	\item Metals bonded to metals for \emph{metallic bonds}, which we will not cover in this class
	\item Some elements for molecules in their natural state
	      \begin{itemize}
		      \item The \emph{diatomic} elements are: \ch{H2}, \ch{N2}, \ch{O2}, \ch{F2}, \ch{Cl2}, \ch{Br2}, and \ch{I2}
		      \item Some solid elements form other molecules: \ch{P4} and \ch{S8}
		      \item Some elements occur in more than one natural state, or \emph{allotrope}

		            Carbon can form coal, graphite, or diamond. Oxygen can form common \ch{O2} or ozone (\ch{O3}). Phosphorous comes in two forms (red, and white)
	      \end{itemize}
\end{itemize}
\section{Naming Binary Covalent Compounds}
\begin{itemize}
	\item Binary molecular compounds are made up two non-metals
	\item Non-metals can combine in different ratios (\ch{NO}, \ch{N2O}, \ch{NO2}, \ch{N2O3}, and more!)
	\item Because of this, binary molecular compounds are named in a way that indicates how many atoms of each element are present
	\item The number of atoms is indicated by Greek prefixes (Table 3.1)
	\item \rule[-1mm]{1in}{.1pt} First Element Name \rule[-1mm]{1in}{.1pt} Second Element Name-``ide''
	\item The element futher left on the periodic table goes first in both the formula and the name
	\item ``Mono-'' may be left off for the \emph{first} element (i.e. Carbon Dioxide, not Monocarbon Dioxide)
	\item Binary compounds which contain \ch{H} are usually called by their common names instead of using this convention (water, hydrochloric acid, and ammonia)

	      Practice: \ch{NO} (nitrogen monoxide)  \hspace{1em} \ch{N2O} (dinitrogen monoxide)  \hspace{1em} \ch{NO2} (nitrogen dioxide)

	      ~\hphantom{Practice:} \ch{N2O3} (dinitrogen trioxide)
\end{itemize}
\section{Formulas for Ionic Compounds}
\begin{itemize}
	\item Positive ions are called \emph{cations}, and negative ions are called \emph{anions}
	\item Ions can be a single atom (monatomic), and the magnitude of the charge can often be inferred from the element's position in the periodic table (Figure 3.7)
	\item Ions can also be made from a group of atoms which carry the charge collectively (polyatomic)  -- Table 3.4
	\item All ionic compounds are electrically neutral, so their positive and negative ions must cancel out
	\item Find the LCM between the positive and negative charges to know how many of each to include

	      Practice: Give the formula unit for compounds made from the following pairs of ions:

	      ~\hphantom{Practice:} \ch{Na^+} and \ch{Cl^-} (\ch{NaCl}) \hspace{1em} \ch{Ag^{+}} and \ch{S^{2-}} (\ch{Ag2S}) \hspace{1em}  \ch{Fe^{3+}} and \ch{O^{2-}} (\ch{Fe2O3})
	\item For compounds with polyatomic ions, just be sure to include parenthesis

	      Practice: Give the formula unit for compounds made from the following pairs of ions:

	      ~\hphantom{Practice:} \ch{Na^+} and \ch{CH3COO^-} (\ch{NaCH3COO}) \hspace{1em} \ch{Ag^{+}} and \ch{SO4^{2-}} (\ch{Ag2SO4}) \hspace{1em}  \ch{Ni^{2+}} and \ch{PO4^{3-}} (\ch{Ni3(PO4)2})
	\item Because of this charge balance, we can calculate the cation's positive charge based on the formula

	      Practice: Give the positive charge of the cation in each ionic compound

	      ~\hphantom{Practice:} \ch{VO2} (\ch{V^{4+}}) \hspace{1em} \ch{AuCl3} (\ch{Au^{3+}}) \hspace{1em} \ch{Al2S3} (\ch{Al^{3+}})
\end{itemize}
\section{Naming Ionic Compounds}
\begin{itemize}
	\item To name an ionic compound, simply give the cation name and the anion name
	\item Monoatomic cations are just the element name
	\item For elements with more than one common charge, the charge is indicated in (Roman Numerals)

	      Practice: Give the name of the following cations: \ch{Ba^{2+}} (Barium) \hspace{1em} \ch{Cr^{2+}} (Chromium(II))
	\item Ammonium and hydronium are the only common polyatomic cations
	\item Monoatomic anions are just the element name with an ``-ide'' ending
	\item Polyatomic anions follow some rules, but are best just memorized
	      \begin{itemize}
		      \item Oxyanions forms series' which follow the pattern shown in Table 3.3
		      \item Oxyanions with one or more hydrogens have ``hydrogen-'' ``dihydrogen-'' etc. added to their names
		      \item Table 3.4 lists common polyatomic ions (you can ignore the right column except chlorates)
	      \end{itemize}

	      Practice: Name the following ionic compounds:  \ch{Na2SO3} (sodium sulfite)

	      ~\hphantom{Practice:} \ch{NH4NO3} (ammonium nitrate) \hspace{1em} \ch{Fe(HCO3)3} (iron(III) hydrogen carbonate)
	\item \emph{Hydrates} are ionic compounds which naturally incorporate water into their structure
	      \begin{itemize}
		      \item An integer number of water molecules are incorporated, called waters of hydration
		      \item If a hydrated salt is heated sufficiently, the waters will evaporate and the salt become anhydrous
		      \item The formula indicates the number of water molecules per formula unit with a dot (\ch{CuSO4$\cdot$ $5$ H2O})
		      \item To name the compound, give the anhydrous compound name followed by ``\rule[-1mm]{1in}{.1pt} hydrate''. Greek prefixes indicate the number of waters
	      \end{itemize}

	      Practice: Give the names of the following hydrated salts

	      \ch{MgSO4 $\cdot$ H2O} (magnesium sulfate monohydrate)  \ch{MgCl2 $\cdot 6$ H2O} (magnesium chloride hexahydrate)
\end{itemize}
\section{Naming Acids}
\begin{itemize}
	\item Acids are covalent compounds which will release \ch{H} ions when dissolved in water

	      (this is a white lie -- the actual mechanism for acids can get quite complex)
	\item For example: \ch{HNO3} will release its \ch{H} in water to produce \ch{H^+(aq)} and \ch{NO3^-(aq)}
	\item Not all \ch{H} atoms will be released. For \ch{CH3COOH}, only the last \ch{H} is ionizable
	\item Often the ionizable hydrogen(s) will be written first. \ch{H2O2} and acetic acid are common exceptions
	\item Acids with 1 ionizable hydrogen are \emph{monoprotic}. Atoms with 2 or more are \emph{polyprotic}
	\item Binary acids are named ``hydro--Element--ic acid'' (hydrochloric acid, hydrofluoric acid)
	\item Oxyacids are acids with \ch{H} attached to an \ch{O}-containing polyatomic anion
	      \begin{itemize}
		      \item The acid name depends on the anion name
		      \item ``--ate'' ions become ``--ic acid'' (sulfate and sulfuric acid)
		      \item ``--ite'' ions become ``--ous acid'' (hypochlorite and hypochlorous acid)
	      \end{itemize}
\end{itemize}
\section{Nomenclature Review}
\begin{itemize}
	\item Figure 3.15 gives a comprehensive flowchart for naming chemical compounds
	\item Conventions for naming organic molecules will be covered in CHEM 1220
\end{itemize}
\section{The Mole}
\begin{itemize}
	\item We have already gone over how $6.022\times10^{23}=1~mol$, and $1~mol~AMU=1~g$
	\item We can talk about moles of anything: formula units, atoms, etc.

	      Practice: How many moles of \ch{O} atoms are in $2.50~mol$ of \ch{Fe2O3}? ($7.50~mol$)

	      ~\hphantom{Practice:} How many \ch{H} atoms are in $1.40~mol$ of \ch{CH4}? ($3.37\times10^{24}~atoms$)
\end{itemize}
\section{Molar Mass}
\begin{itemize}
	\item The \emph{Molar Mass} gives the mass for a mole of a given substance
	\item The atomic weight (on the periodic table) is the mass for a mole of atoms
	\item The molecular weight or formula weight is the mass of a mole of formula units for a compound
	\item I will use \emph{Molar Mass} ($M$) as a general term for both
	\item Many problems will involve converting between mass and moles, moles and mass
	\item ``Measurement Land'' vs ``Chemistry Land'' vs ``Atomic/Counting Land''

	      Practice: How many \ch{O} atoms are in $1.35~g$ of \ch{CO2}? ($3.69\times10^{22}$)

	      ~\hphantom{Practice:} How many \ch{O} atoms are in $1.35~g$ of \ch{CO}? ($2.90\times10^{22}$)

	      ~\hphantom{Practice:} What is the mass of $0.675~mol$ of \ch{C12H22O11}? ($23.1~g$)
\end{itemize}
\section{Percent Composition}
\begin{itemize}
	\item Percent composition tells what percentage of a compound (by mass) is each element
	\item $\%_{Composition}=\dfrac{\nu M_{atomic}}{M_{formula}}100\%$

	      Practice: Find the \% composition of \ch{O} in \ch{H2O}. ($88.81\%$)

	      ~\hphantom{Practice:} Find the \% composition of \ch{O} in \ch{Fe2(SO3)3}. ($40.92\%$)
	\item Percent composition can be used to find the mass of a component in a compound
	\item The \% composition is a conversion factor between $g$ of the element and $100~g$ of the compound

	      Practice: How many $g$ of \ch{N} are in $2.00~g$ of \ch{NH4NO3}?

	      ~\hphantom{Practice:} ($M=80.0434~\nicefrac{g}{mol}$, $\%_N=34.9978\%$, and $0.700~g$)
\end{itemize}
\section{Empirical Formulas}
\begin{itemize}
	\item Some analytical techniques can only give \% composition, not the true molecular formula
	\item The \emph{empirical formula} is the most mathematically reduced form of a chemical formula
	\item Many molecular formulas may share the same empirical formula (i.e. \ch{CH2O}, \ch{C2H4O2} and \ch{C6H12O6})
	\item Finding empirical formulas requires converting masses to moles
	      \begin{itemize}
		      \item Sometimes you will be given masses (See combustion analysis below)
		      \item If you only have \% composition, assume a mass of $100~g$ and the \%s become $g$
		      \item Convert the masses to moles using atomic weights
		      \item Reduce the moles of each atom to a ratio of whole numbers
		      \item Table 3.6 shows the decimal ending for common fractional equivalents
	      \end{itemize}

	      Practice: Give the empirical formula for a $20~g$ sample with $1.34~g$ \ch{H}, $8.00~g$ \ch{C} and $10.7~g$ \ch{O}.

	      ~\hphantom{Practice:} (\ch{CH2O})

	      Practice: Give the empirical formula for a compound which is $69.94\%$ \ch{Fe} and $30.06\%$ \ch{O} by mass.

	      ~\hphantom{Practice:} (\ch{Fe2O3})
\end{itemize}
\section{Molecular Formulas}
\begin{itemize}
	\item The molecular formula gives the actual number of atoms of each type in a single molecule
	\item The molecular formula will be some integer multiple (possibly $1$) of the empirical formula
	\item The compound's molar mass will be the same multiple of the empirical formula's molar mass
\end{itemize}
\section{Combustion Analysis}
\begin{itemize}
	\item Combustion analysis is a laboratory technique for finding the empirical formula of combustible compounds
	\item It involves combusting a carefully weighed mass of compound in excess oxygen, and measuring the amount of water and carbon dioxide produced
	\item The \ch{C} in the carbon dioxide and the \ch{H} in the water came exclusively from the unknown compound
	\item Any unaccounted-for mass is assumed to be oxygen
	\item Process for solving combustion analysis:
	      \begin{itemize}
		      \item Calculate moles of \ch{C} atoms from \ch{CO2} mass
		      \item Calculate moles of \ch{H} atoms from \ch{H2O} mass
		      \item Calculate mass of \ch{C} and \ch{H} atoms
		      \item subtract \ch{C} and \ch{H} masses from sample mass -- This is the \ch{O} mass
		      \item Calculate moles of \ch{O} from the \ch{O} mass
		      \item Reduce all mole ratios to whole numbers (perhaps relying on Table 3.6)
		      \item This is the empirical formula
	      \end{itemize}

	      Practice: $4.24~g$ of an unknown are combusted to yield $6.21~g$ of \ch{CO2} and $2.54~g$ of \ch{H2O}

	      ~\hphantom{Practice:} (Empirical Formula is \ch{CH2O})

	      Practice: $2.50~g$ of an unknown are combusted to yield $5.79~g$ of \ch{CO2} and $1.18~g$ of \ch{H2O}

	      ~\hphantom{Practice:} (Empirical Formula is \ch{C8H8O3})

	      Practice: The molar masses for the two compounds above are: $120.10~\nicefrac{g}{mol}$ and $152.15~\nicefrac{g}{mol}$\\
	      ~\hphantom{Practice: } What are the molecular formulas?

	      ~\hphantom{Practice:} (The formulas are: \ch{C4H8O4} and \ch{C8H8O3})
\end{itemize}

\chapter{Chemical Reactions and Aqueous Solutions}
\section{Chemical Equations}
\begin{itemize}
	\item Chemical equations show how \emph{reactants} (on the left) are converted into \emph{products} (on the right)
	\item When properly balanced, an equation also shows the proportions of reactants and products
	\item Equations also can show the \emph{phase} of each chemical and any reaction conditions
	\item Aqueous solutions are when a substance is dissolved in water
	\item Balancing chemical equations:
	      \begin{itemize}
		      \item To balance a chemical equation, we will add the proper coefficients to give the same numbers and types of atoms on both sides of the reaction arrow
		      \item Start with any polyatomic ions, and balance them as a unit rather than counting individual atoms
		      \item Next balance elements that appear in only one compound on each side
		      \item Finally, balance any remaining elements
	      \end{itemize}

	      Practice: Balance the following chemical equation: \ch{Fe2O3(s) + Al(s) -> Al2O3(s) + Fe(s)}

	      ~\hphantom{Practice:} (\ch{Fe2O3(s) + 2 Al(s) -> Al2O3(s) + 2 Fe(s)})

	      Practice: Balance the following chemical equation: \ch{C8H18(l) + O2(g) -> CO2(g) + H2O(g)}

	      ~\hphantom{Practice:} (\ch{2 C8H18(l) + 25 O2(g) -> 16 CO2(g) + 18 H2O(g)})

	      Practice: Balance the following chemical equation: \ch{Ag2SO4(aq) + NaCl(aq) -> AgCl(s) + Na2SO4(aq)}

	      ~\hphantom{Practice:} (\ch{Ag2SO4(aq) + 2 NaCl(aq) -> 2 AgCl(s) + Na2SO4(aq)})
\end{itemize}
\section{Types of Chemical Reactions}
\begin{itemize}
	\item Reactions can be grouped into broad categories
	\item Synthesis or Combination reactions:
	      \begin{itemize}
		      \item Two or more reactants combine to form a single product
		      \item \ch{2 Na(s) + Cl2{g} -> 2 NaCl(s)}
	      \end{itemize}
	\item Decomposition reactions:
	      \begin{itemize}
		      \item One reactant decomposes into two or more products
		      \item \ch{2 H2O(l) -> 2 H2(g) + O2(g)}
	      \end{itemize}
	\item Single and Double Replacement reactions:
	      \begin{itemize}
		      \item One (or two) element or polyatomic ion is replaced by another in the product
		      \item \ch{Zn(s) + 2 HCl(aq) -> ZnCl2(aq) + H2(g)}
		      \item \ch{2 KI(aq) + Pb(NO3)2(aq)-> PbI2(s) + 2 KNO3(aq)}
	      \end{itemize}
	\item Acid/Base reactions (A type of replacement reaction):
	      \begin{itemize}
		      \item  One or more \ch{H^{+}} are exchanged
		      \item \ch{HCl(aq) + NaOH(aq) -> NaCl(aq) + H2O(l)}
	      \end{itemize}
	\item Precipitation reactions (A type of replacement reaction):
	      \begin{itemize}
		      \item Aqueous reactants produce one or more solid products (called precipitates)
	      \end{itemize}
	\item Combustion reactions:
	      \begin{itemize}
		      \item A substance is reacted with \ch{O2} to produce \ch{CO2} and \ch{H2O}
		      \item \ch{C3H8(g) + 5 O2(g) -> 3 CO2(g) + 4 H2O(g)}
	      \end{itemize}
	\item Redox reactions (synthesis, decomposition, single replacement, and combustion reactions)
	      \begin{itemize}
		      \item We will see later how redox reactions involve the exchange of electrons
	      \end{itemize}
	\item Not all reactions we can write down will actually proceed in nature
	\item Reactions actually happen because of driving forces which make products more thermodynamically stable than the reactants
	      \begin{itemize}
		      \item Precipitation reactions have stable solid products with strong ionic bonds
		      \item Acid/Base reactions have stronger bonds to H after the neutralization
		      \item Redox reactions transfer electrons to make more stable electronic configurations
	      \end{itemize}
\end{itemize}
\section{Compounds in Aqueous Solution}
\begin{itemize}
	\item When ionic compounds dissolve in water, they dissociate into their cations and anions
	\item Figure 4.14 shows how ions are hydrated by the water molecules when they dissolve
	\item Solutions of ionic compounds and acids/bases are called electrolytes because they conduct electricity
	      \begin{itemize}
		      \item Strong electrolytes will dissociate completely and produce a lot of ions
		      \item Weak electrolytes dissociate only partially, and produce few ions
		      \item Soluble ionic compounds and strong acids/bases (Table 4.3) are strong electrolytes
		      \item Weak acids/bases are weak electrolytes
	      \end{itemize}
	\item Solutions of other molecular compounds are non-electrolytes -- They don't dissociate in water
\end{itemize}
\section{Precipitation Reactions}
\begin{itemize}
	\item Not every ionic compound will actually dissolve in water. For example, most rocks and crystals are ionic compounds which don't readily dissolve in water
	\item \emph{Soluble} compounds will dissolve in water, \emph{insoluble} compounds will not
	\item Solubility guidelines help predict if a compound is soluble or not
	      \begin{itemize}
		      \item Ammonium and Group I cations \emph{always} form soluble compounds
		      \item Nitrate, chlorate, perchlorate, and acetate \emph{always} form soluble compounds
		      \item Chlorides, bromides, and iodides are soluble \emph{except} with \ch{Ag^{+}}, \ch{Pb^{2+}}, and \ch{Hg2^{2+}}
		      \item Sulfates are soluble \emph{except} with \ch{Pb^{2+}}, \ch{Hg2^{2+}}, and ``heavy'' group II ions
		      \item Carbonate, sulfite, phosphate, and chromate generally form \emph{insoluble} compounds
		      \item \ch{S^{2-}} and \ch{OH^-} form \emph{insoluble} compounds, except with group I, ammonium, and ``heavy'' group II ions
		      \item \ch{Ag^{+}}, \ch{Pb^{2+}}, and \ch{Hg2^{2+}} generally form \emph{insoluble} compounds
		      \item Table 4.5 gives a comprehensive list, but you will be fine just memorizing the points above
	      \end{itemize}
	\item Precipitation reactions:
	      \begin{itemize}
		      \item When two soluble ionic compounds react, there is the possibility of a precipitate forming
		      \item First, identify the products when cations and anions switch partners
		      \item Then determine if any of the products will be insoluble -- these are the precipitates
		      \item If no precipitate forms, then no reaction really took place at all
	      \end{itemize}
	\item Demos: \ch{Mg(NO3)2} with \ch{NaOH}, and \ch{Ca(NO3)2} with \ch{Na2CO3}
	\item Net ionic equations:
	      \begin{itemize}
		      \item Ionic equations acknowledge that soluble ionic compounds do not exist with anions and cations paired together. Rather, the anions are solvated and the cations are solvated separately by themselves
		      \item To form an ionic equation, write the cations and anions separately for all aqueous (soluble) ionic compounds
		      \item Insoluble compounds are still written with ions paired together because they actually are
		      \item Some ions appear solvated on both sides of the equation -- they are called ``spectators''
		      \item These spectators don't actually do anything at all -- they are as irrelevant to the reaction as the solvent, other trace solutes (\ch{N2} and \ch{O2}), etc.
		      \item To form a \emph{net} ionic equation, simply eliminate the spectators
	      \end{itemize}

	      Practice: Write the net ionic equations for the two demonstrations above

	      ~\hphantom{Practice:} \ch{Mg^{2+}(aq) + 2 OH^{-}(aq) -> Mg(OH)2(s)}

	      ~\hphantom{Practice:} \ch{Ca^{2+}(aq) + CO3^{2-}(aq) -> CaCO3(s)}
\end{itemize}
\section{Acid-Base Reactions}
\begin{itemize}
	\item Recognize acid/base reactions by the exchange of a \ch{H^+} ion
	\item When acids react with \ch{OH^-} (strong bases), water will be produced
	\item When acids react with weak bases, the \ch{H+} switches over to the base
	\item Balance acid/base reactions just like any other reaction
	\item For ionic and net ionic equations, only \emph{strong} acids should be written as dissociated ions
\end{itemize}
\section{Oxidation States and Redox Reactions}
\begin{itemize}
	\item Oxidation states (or numbers) keep track of how many electrons reside on an atom (like a charge)
	\item To determine the oxidation states in a compound, follow these rules:
	      \begin{itemize}
		      \item A neutral elements (not a part of a compound) has an oxidation number of $0$
		      \item Monoatomic ions have oxidation numbers equal to their charge
		      \item The sum of oxidation numbers in any formula is equal to the total charge
		      \item Oxygen tends to have an oxidation number of $-2$ in compounds
		      \item Hydrogen tends to have an oxidation number of $+1$ in compounds
		      \item All other elements can be determined from the rules above
	      \end{itemize}

	      Practice: Identify the oxidation states for all elements in the following compounds\\
	      ~\hphantom{Practice: } \ch{NH3}, \ch{N2O}, \ch{NO3^-}

	      ~\hphantom{Practice:} (N=-3, H=+1), (N=+1, O=-2), (N=+5, O=-2)
	\item Oxidation is the process of \emph{losing} electrons. The oxidation number increases
	\item Reduction is the process of \emph{gaining} electrons. The oxidation number decreases
	\item Two mnemonic devices to keep them straight:
	      \begin{itemize}
		      \item OIL RIG (oxidation is losing, reduction is gaining)
		      \item \emph{Chemical} reduction involves a \emph{mathematical} reduction of the oxidation number
	      \end{itemize}
	\item To identify redox reactions involve the exchange of electrons
	      \begin{itemize}
		      \item To identify a redox reaction, first find the oxidation state of all reactants and products
		      \item If the oxidation state of any elements changes, then this is a redox reaction
		      \item The element whose oxidation number decreases is \emph{reduced}
		      \item The element whose oxidation number increases is \emph{oxidized}
		      \item Any electrons lost by the oxidized element have gone to the reduced element -- they are connected both chemically and mathematically by the electrons
		      \item The compound which contains the reduced element has oxidized its reaction partner. Therefore, it is the ``oxidizing agent''
		      \item The compound which contains the oxidized element has reduced its reaction partner. Therefore, it is the ``reducing agent''
	      \end{itemize}

	      Practice: Identify the reducing agent and the oxidizing agent in each chemical reaction

	      ~\hphantom{Practice:} \ch{2 Al(s) + 3 Cl2(g) -> 2 AlCl3(s)}\\
	      ~\hphantom{Practice: } (O.A. = \ch{Cl2}, R.A. = \ch{Al})

	      ~\hphantom{Practice:} \ch{CH4(g) + 3 O2(g) -> CO2(g) + 2 H2O(g)}\\
	      ~\hphantom{Practice: } (O.A. = \ch{O2}, R.A. = \ch{CH4})
	\item Balancing redox reactions can be quite complicated. You will learn how in CHEM-1220
\end{itemize}
\section{Predicting the Products of Redox Reactions}
\begin{itemize}
	\item Synthesis and decomposition reactions are very often redox reactions
	      \begin{itemize}
		      \item Sythesis of ionic compounds from elements, and decomposition of ionic compounds to elements
		      \item The oxidation states of metals in compounds will match the charges they can take as ions
	      \end{itemize}
	\item Single-replacement reactions are also very often redox reactions
	      \begin{itemize}
		      \item Often, the replaced elements involve an element and an ion swapping places
		      \item One element goes from neutral to positively charged -- it is oxidized -- it is a reducing agent
		      \item The other element goes from positively charged to neutral -- it is reduced -- it is an oxidizing agent
		      \item Some elements are easier to oxidize, and some ions are easier to reduce
		      \item These processes are linked -- an element which is easy to oxidize has an ion which is difficult to reduce
		      \item Table 4.7 shows an activity series for common elements in these reactions
		      \item A reaction will proceed spontaneously if the reducing agent is higher in the table than the oxidizing agent
	      \end{itemize}
	\item Demos: Copper wire in silver solution and Zn piece in copper solution

	      Practice: Predict whether each demonstrated reaction will proceed spontaneously

	      ~\hphantom{Practice:} \ch{Cu(s) + 2 AgNO3(aq) -> Cu(NO3)2(aq) + 2 Ag(s)}\\
	      ~\hphantom{Practice: } (Spontaneous)

	      ~\hphantom{Practice:} \ch{Zn(s) + CuSO4(aq) -> ZnSO4(aq) + Cu(s)}\\
	      ~\hphantom{Practice: } (Spontaneous)
\end{itemize}

\chapter{Stoichiometry}
\section{Mole Calculations for Chemical Reactions}
\begin{itemize}
	\item \emph{Stoichiometry} refers to the ratios between substances in chemical reactions
	\item These ratios are expressed in the coefficients of a balanced chemical equation
	\item Use the coefficients to create a conversion factor between substances
	\item My problem-solving diagram (``Measurement Land'' vs ``Chemistry Land'')

	      Practice: Consider the reaction \ch{2 P + 3 Cl2 -> 2 PCl3}

	      ~\hphantom{Practice:} How many moles of \ch{Cl2} are needed to react with $0.250~mol$ of \ch{P}? \hspace{1em} ($0.375~mol$)

	      ~\hphantom{Practice:} How many moles of \ch{PCl3} would be produced? \hspace{1em} ($0.250~mol$)
\end{itemize}
\section{Mass Calculations for Chemical Reactions}
\begin{itemize}
	\item We will usually be concerned about finding \emph{masses} rather than moles
	\item You cannot compare masses directly, but rather have to go through the molar stoichiometry
	\item Problems are structured as: $mass_A\rightarrow mol_A\rightarrow mol_B\rightarrow mass_B$

	      Practice: Consider the reaction \ch{3 O2 + 2 KCl -> 2 KClO3}

	      ~\hphantom{Practice:} How many $g$ of \ch{O2} are needed to react with $0.850~g$ of \ch{KCl}? \hspace{1em} ($0.547g$)

	      ~\hphantom{Practice:} How many $g$ of \ch{KClO3} would be produced? \hspace{1em} ($1.40g$)

	\item Note that the total mass on the reactant side should equal the total mass on the product side
\end{itemize}
\section{Problems Involving Limiting Quantities}
\begin{itemize}
	\item Reactants are often not mixed in the perfect, proper ratio
	\item One reactant will be consumed and run out first, it is called the \emph{limiting reactant}
	\item The reaction stops once the limiting reactant runs out, so the limiting reactant controls how much product is produced
	\item There are many ways to solve limiting reactant problems. I suggest the following:
	      \begin{itemize}
		      \item Looking ahead, choose a product which you are interested in
		      \item For each reactant, calculate the amount of product it could produce
		      \item Compare the product amounts -- the smallest value is the only value you keep
		      \item Throw out all other values, they are just wishful thinking and will never really happen
		      \item The reactant which gave the smallest amount of product is the limiting reactant
		      \item Start with the limiting reactant for all future calculations
	      \end{itemize}
	\item Sometimes you need to find the amount of excess reactant(s) which remains
	      \begin{itemize}
		      \item Starting from the limiting reactant, find the amount of excess reactant required
		      \item Subtract that mass from the initial value -- this is the amount left over
	      \end{itemize}
	\item Here, the total mass at the end will match the total mass at the beginning

	      Practice: Consider the reaction \ch{2 CH3OH(g) + 3 O2(g) -> 2 CO2(g) + 4 H2O(g)}

	      ~\hphantom{Practice:} If $16.0~g$ of \ch{O2} react with $48.1~g$ of \ch{CH3OH}, which is the limiting reactant? \hspace{1em} (\ch{O2})

	      ~\hphantom{Practice:} How many $g$ of each substance will be present after the reaction completes?\\
	      ~\hphantom{Practice: } ($0~g$ \ch{O2}, $37.4~g$ \ch{CH3OH}, $14.7~g$ \ch{CO2}, and $12.0~g$ \ch{H2O})

	      Practice: Consider the reaction \ch{3 Fe(s) + 4 H2O(l) -> Fe3O4(s) + 4 H2(g)}

	      ~\hphantom{Practice:} If $5.00~g$ of \ch{Fe} react with $5.00~g$ of \ch{H2O}, which is the limiting reactant? \hspace{1em} (\ch{Fe})

	      ~\hphantom{Practice:} How many $g$ of each substance will be present after the reaction completes?\\
	      ~\hphantom{Practice: } ($0~g$ \ch{Fe}, $2.85~g$ \ch{H2O}, $6.91~g$ \ch{Fe3O4}, and $0.241~g$ \ch{H2})
\end{itemize}
\section{Theoretical Yield and Percent Yield}
\begin{itemize}
	\item The amount of product you calculate above is the \emph{theoretical yield}
	\item In the lab, your \emph{actual yield} may be more, or less due to errors and random factors
	\item We can calculate the \% yield based on these two values

	      $Percent~Yield = \left(\dfrac{Actual~Yield}{Theoretical~Yield}\right)\cdot 100\%$

	      Practice: If you run the first reaction above and recover $10.7~g$ of \ch{H2O}, what is the \% yield?\\
	      ~\hphantom{Practice: } ($89.2\%$)

	      ~\hphantom{Practice:} If you run the second reaction above and recover $7.05~g$ of \ch{Fe3O4}, what is the \% yield?\\
	      ~\hphantom{Practice: } ($102\%$)
\end{itemize}
\section{Definition and Uses of Molarity}
\begin{itemize}
	\item The \emph{solvent} is the substance doing the dissolving, the \emph{solute} is the substance being dissolved, and the \emph{solution} is the homogeneous mixture after dissolving
	\item Concentrated solutions have a lot of solute, dilute concentrations have little solute
	\item The most common unit of concentration is \emph{Molarity}: $M=\dfrac{Moles~of~Solute}{L~of~Solution}$ or $\left(M=\dfrac{n}{V}\right)$
	\item Molarity is sometimes notated with square brackets (i.e. $\left[\ch{H2SO4}\right]$)
	\item This can be rearranged to give: $Moles_{Solute} = Molarity \cdot L_{Solution}$ or ($n = M\cdot V$)
	\item For small volumes ($ml$), we can use millimoles ($mmol$) for convenience

	      Practice: What is the molar concentration when $2.50~g$ of \emph{NaCl} are dissolved to make $0.100~L$ of\\ ~\hphantom{Practice: } solution? \hspace{1em} ($0.428~M$)

	      Practice: What mas of \ch{C12H22O11} should be dissolved to make $0.500~L$ of a $0.125~M$ solution?\\
	      ~\hphantom{Practice: } ($21.4~g$)
	\item Dilution is when water is added to a solution to make it less concentrated
	\item Because only solvent is added, the moles of solute is the same at the end as at the beginning
	\item This gives a simple equation for dilution: $M_1V_1=M_2V_2$

	      Practice: What is the final concentration when $25.0~ml$ of $0.832~M$ are diluted to $150.0~ml$?\\
	      ~\hphantom{Practice: } ($0.139~M$)

	      Practice: How much $0.650~M$ solution should be used to make $250.0~ml$ of a $0.100~M$ solution?\\
	      ~\hphantom{Practice: } ($38.5~ml$)
	\item We can now add to the ``Measurement Land'' and ``Chemistry Land'' diagram
	      \begin{itemize}
		      \item Moles can be calculated from $n = MV$
		      \item $ml$ of the co-reactant can be found instead of $g$
		      \item For finding concentrations of the product, consider the \emph{additive} volumes of the reactants
	      \end{itemize}

	      Practice: Consider the following reaction: \ch{2 HCl(aq) + Na2CO3(s) -> 2 NaCl(aq) + H2CO3(aq)}

	      ~\hphantom{Practice:} How many $ml$ of $0.250~M$ should be used to react with $0.125~g$ of \ch{Na2CO3}?\\
	      ~\hphantom{Practice: } ($9.43~ml$)
\end{itemize}
\section{Molarities of Ions}
\begin{itemize}
	\item When ionic compounds dissolve, the ions dissociate
	\item The concentration of the \emph{compound} may be different from the concentration of a given ion
	\item Multiply the formula concentration by the number of ions in the formula

	      Practice: Consider dissolving $1.45~g$ of \ch{MgCl2} to make $125~ml$ of solution

	      ~\hphantom{Practice:} What is the concentration of \ch{MgCl2}?\\
	      ~\hphantom{Practice: } ($1.22~M$)

	      ~\hphantom{Practice:} What is the concentration of \ch{Mg^{2+}} and \ch{Cl^-} ions?\\
	      ~\hphantom{Practice: } ($1.22~M$ and $2.44~M$)
\end{itemize}
\section{Calculations Involving Other Quantities}
\begin{itemize}
	\item Volume is one way to express an amount of a substance
	\item We have so far dealt with volumes of solutions, where $n=MV$
	\item For a pure substance, the volume must be dealt with very differently
	      \begin{itemize}
		      \item Volume must first be converted to mass through density: $m=d\cdot v$
		      \item Then mass is converted to moles through the molar mass
	      \end{itemize}

	      Practice: Sodium metal has a density of $0.97\nicefrac{g}{cm^3}$, and reacts with water according to the\\
	      ~\hphantom{Practice: } following reaction: \ch{2 Na(s) + 2 H2O(l) -> 2 NaOH(aq) + H2(g)}

	      ~\hphantom{Practice:} If $0.750~cm^3$ of \ch{Na} react with $200.0~ml$ of water, what is the final $\left[\ch{NaOH}\right]$? \hspace{1em} ($0.158~M$)
	\item The subscripts in a chemical formula can also relate the number of atoms of a particular element

	      Practice: In the reaction above, how many \ch{H} \emph{atoms} are released as gas? \hspace{1em} ($1.91\times10^{22}~\ch{H}~atoms$)
\end{itemize}
\section{Calculations with Net Ionic Equations}
\begin{itemize}
	\item Stoichiometric calculations with net ionic equations are just like with regular equations
	\item To calculate moles of ions from $g$ of solid salt, you must know the full salt formula
	\item Subscripts indicate when multiple ions come from a single formula unit

	      Practice: How many $g$ of \ch{Ca3(PO4)2} are needed to make $0.250~L$ with $\left[\ch{Ca^{2+}}\right]=0.100~M$? \hspace{1em} ($2.58~g$)
\end{itemize}
\section{Titration}
\begin{itemize}
	\item \emph{Titration} is a technique to determine the concentration of a solution
	\item The unknown is reacted with a solution with precisely known concentration (a \emph{standard solution})
	\item A precisely measured volume of one reactant is placed in an erlenmeyer flask (for easy mixing)
	\item The other reactant (the titrant) is slowly added using a buret, to measure the volume added
	\item The \emph{equivalence point} is the exact point when the unknown is completely consumed by the reaction
	\item The \emph{end point} is when you actually stop the titration
	      \begin{itemize}
		      \item End point is usually identified by a color-changing indicator
		      \item For acid/base reactions, the indicator changes color with pH
		      \item Redox reactions also have appropriate color indicators
		      \item The indicator is chosen so that the end point is as close as possible to the equivalence point
	      \end{itemize}
	\item The stoichiometric coefficients ($\nu$) are included in the titration equation: $\dfrac{M_AV_A}{\nu_A}=\dfrac{M_BV_B}{\nu_B}$

	      Practice: Find $\left[\ch{NaOH}\right]$ when $25.00~ml$ of \ch{NaOH} are titrated with $37.8~ml$ of $0.100~M$ \ch{HCl}\\
	      ~\hphantom{Practice: } ($0.151~M$)

	      Practice: Find $\left[\ch{HNO3}\right]$ when $50.00~ml$ of \ch{HNO3} are titrated with $21.3~ml$ of $0.150~M$ \ch{Ca(OH)2}\\
	      ~\hphantom{Practice: } ($0.128~M$)
\end{itemize}

\chapter{Thermochemistry}
\section{Energy and Energy Units}
\begin{itemize}
	\item Thermochemistry is the study of heat and energy changes in chemical reactions
	\item It also includes topics like entropy and spontaneity
	\item Energy can come in two forms, kinetic and potential
	      \begin{itemize}
		      \item Kinetic energy is the energy of motion: $KE=\dfrac{1}{2}mv^2$
		      \item Potential energy is stored energy: Gravitaional, electrostatic, chemical, etc.
	      \end{itemize}
	\item Energy has several common units
	      \begin{itemize}
		      \item The SI unit is the Joule: $1~J=1~kg\dfrac{m^2}{s^2}$
		      \item The calorie: $1~cal=4.184~J$
		      \item The Calorie (kcalorie): $1~Cal=1000~cal=4184~J$
	      \end{itemize}
\end{itemize}
\section{Energy, Heat, and Work}
\begin{itemize}
	\item When we talk about changes and transfers of energy, we need to carefully define our system
	      \begin{itemize}
		      \item The \emph{system} is the part of the universe where the reaction occurs, such as a beaker or chamber with reactants
		      \item The \emph{surroundings} is the rest of the universe
		      \item Open systems can exchange both heat and matter with the surroundings (an open beaker)
		      \item Closed systems can exchange heat, but not matter with the surroundings (a closed chamber)
		      \item Isolated systems cannot exchange either heat or matter with the surroundings (an closed, insulated flask)
	      \end{itemize}
	\item Work and heat are the ways energy can enter or leave the system
	\item We always talk from the perspective of the system, even though we are a part of the surroundings
	\item Work is defined as $w=f\cdot d$ or $w=-P\cdot\Delta V$
	      \begin{itemize}
		      \item Positive work is when the system volume decreases
		      \item Negative work is when the system volume increases, or a force moves part of the surroundings
	      \end{itemize}
	\item Heat is usually associated with temperature changes or phase changes
	      \begin{itemize}
		      \item Positive heat is observed by an “upward” phase change or a cold temperature
		      \item Holding ice in your hand will melt the ice, and make your hand cold
		      \item Negative heat is observed by a “downward” phase change or a hot temperature
		      \item Burning wood in a fire feels warm because of the negative system heat
	      \end{itemize}
	\item First Law of Thermodynamics: The energy of the universe is constant
	\item Internal energy is the sum of all kinetic and potential energy in a system
	      \begin{itemize}
		      \item Over a process, the sum of work and heat is the change in internal energy
		      \item $\Delta U=q+w$
	      \end{itemize}
	      Practice: A fire piston is a device for starting fires by rapidly plunging an airtight piston with a\\
	      ~\hphantom{Practice: } combustible material on its end.

	      ~\hphantom{Practice:} If you push the plunger with a force of $750~N$ over a distance of $10~cm$, what is $w$ for\\
	      ~\hphantom{Practice: } the fire piston? ($75~J$)

	      ~\hphantom{Practice:} If the fire piston is not perfectly insulated, and loses $5~J$ of heat to the surroundings,\\
	      ~\hphantom{Practice: } what is $\Delta U$ for the system? ($70~J$)
\end{itemize}
\section{Energy as a State Function}
\begin{itemize}
	\item Some quantities are \emph{Path Functions}, while others are \emph{State Functions}
	\item To understand a state function, first consider a path function
	      \begin{itemize}
		      \item Path functions depend on the path taken
		      \item If I push a chair to one side of the room, then pull it back, it will be in the exact same state as it started, but the total $w$ done on the system is not $0$
		      \item Work is a path function, and depends on the path (I might increase the work by taking a different path)
	      \end{itemize}
	\item State functions do not depend on the path, but only on the state
	      \begin{itemize}
		      \item If I lift a chair up, then place it on the ground again, its final elevation will be the same, regardless of the path it took
		      \item Temperature, elevation, and many thermodynamic quantities are state functions
		      \item State functions are useful because you only need to know the current state to know the function's value
	      \end{itemize}
	\item While $q$ and $w$ are path functions, $U$ is a state function
	\item Consider two very different ways to produce ethanol gas:
	      \begin{itemize}
		      \item \ch{C2H6O(s,}$-115~^\circ C$\ch{) -> C2H6O(g,}$79~^\circ C$)
		      \item \ch{2 C(s) + 3 H2(g) + 1/2 O2(g) -> C2H6O(g,}$79~^\circ C$)
		      \item Because the final states are identical, the final $U$s are identical
		      \item Any other process (maybe with many steps) will have the same final $U$ as long as it has the same final state
	      \end{itemize}
\end{itemize}
\section{Energy and Enthalpy}
\begin{itemize}
	\item All chemical reactions involve at least \emph{some} exchange of heat, but many involve work as well
	\item When gasses are produced or consumed, the pressure and volume will change, doing some work
	\item This work is measured by $w=-P\Delta V$, where $1~Latm=101.325~J$
	\item This $PV$ work is actually a bit problematic when trying to keep track of energy
	      \begin{itemize}
		      \item $T$ is easy to measure with a thermometer, but both $P$ and (especially) $V$ are more difficult to measure
		      \item Most of our work as chemists is done at constant pressure (open flask or in a balloon)
		      \item Under constant pressure, we can use \emph{Enthalpy} ($H$) instead of internal energy ($U$)
		      \item While $U=q+w$, $H=q$ under constant pressure conditions
		      \item So, we only need to worry about heat when we deal with $H$
	      \end{itemize}
	\item Processes with $-H$ are called \emph{exothermic}
	\item Examples of exothermic processes are downward phase changes, combustion reactions, etc.
	\item Processes with $+H$ ($+q$) are called \emph{endothermic}
	\item Examples of endothermic processes are upward phase changes, ice-pack chemical reactions, etc.
	\item Demo -- An endothermic reaction: $8.5~g$ \ch{NH4SCN} with $16.0~g$ \ch{Ba(OH)2$\cdot$ 8 H2O}
\end{itemize}
\section{Specific Heat}
\begin{itemize}
	\item When heat is added to a system, it will either undergo a phase change, or heat up
	\item The specific heat gives how much heat is required to warm a given substance
	\item $q=mc\Delta T$ \hspace{2em} where $m$ is the mass, $c$ is the specific heat, and $\Delta T$ is the temperature change
	\item Table 6.3 includes specific heats for a number of common substances

	      Practice: How much heat is required to heat $12.5~g$ of water by $5.75~^\circ C$? ($30.1~J$)

	      ~\hphantom{Practice:} If a $5.23~g$ block of \ch{Al} at $22.0~^\circ C$ is given $75.0~J$ of heat, what will its final $T$ be? ($38.1~^\circ C$)
\end{itemize}
\section{Calorimetry: Measuring Energy Changes}
\begin{itemize}
	\item Whenever heat is transferred, the total energy of the universe remains constant
	\item For a heat transfer in an isolated system, $q_1=-q_2$
	\item Consider a block of hot metal placed in a beaker of room temperature water. Heat will flow from the block into the water until the two temperature are equal to each other
	\item $q_1=-q_2$ becomes $m_1c_1\left(T_f-T_{i,1}\right)=-m_2c_2\left(T_f-T_{i,2}\right)$

	      Practice: A $10.0~g$ block of iron is heated to $93.5~^\circ C$ and placed in $25~ml$ of $23.0~^\circ C$ water.\\
	      ~\hphantom{Practice:} What is the final temperature? ($25.9~^\circ C$)
	\item We can also measure the heat transfer associated with a chemical reaction, called \emph{calorimetry}
	\item Constant-pressure (coffee-cup) calorimetry
	      \begin{itemize}
		      \item The reaction is carried out in aqueous solution
		      \item $q_{rxn}$ is the heat released or absorbed by the reaction
		      \item The heat of reaction is exchanged with the solution: $q_{rxn} = -q_{soln} = -mc\Delta T$
		      \item $m$ is the solution mass, which will include the water and any solutes
		      \item $c$ is the solution specific heat, but this is simplified by assuming $c_{soln}=c_{water}=4.184\frac{J}{g~^\circ C}$
		      \item $\Delta H = \dfrac{q_{rxn}}{n_{rxn}} = \dfrac{-mc\Delta T}{n_{rxn}}$ where $n_{rxn}$ is the moles of reaction: $n_{rxn}=\left(\dfrac{n_A}{\nu_A}\right)$
		      \item Demo -- \ch{NaOH} enthalpy of solvation $\left(10~g, 100~ml, –44.2~\frac{kJ}{mol}\right)$
	      \end{itemize}
	\item Constant-volume (bomb) calorimetry
	      \begin{itemize}
		      \item The reaction is carried out in a chamber charged with high pressure \ch{O2}
		      \item $q_{rxn}$ is exchanged with the whole bomb-calorimeter apparatus
		      \item The calorimeter is calibrated to give a \emph{heat capacity} ($C_{cal}$) with units $\frac{J}{^\circ C}$
		      \item $q_{rxn}=-q_{cal}=-C_{cal}\Delta T$
		      \item For constant volume, we measure $U$ instead of $H$ because $w=0$
		      \item $\Delta U = \dfrac{-C_{cal}\Delta T}{n_{rxn}}$
	      \end{itemize}
\end{itemize}
\section{Enthalpy in Chemical Reactions}
\begin{itemize}
	\item A balanced chemical reaction may also include an enthalpy of reaction $\Delta H$
	\item This tells how much heat is produced or consumed with one mole of reaction
	\item $\Delta H$ can be a conversion factor between heat and amounts of reactants or products

	      Practice: Consider the reaction \ch{N2(g) + 3 H2(g) -> 2 NH3(g)} $\Delta H = -92\frac{kJ}{mol}$

	      ~\hphantom{Practice:} If $2.25~g$ of \ch{H2} are consumed in the above reaction, how much heat is released? ($34.2~kJ$)

	      ~\hphantom{Practice:} If $54.6~kJ$ of heat are released, how many $g$ of \ch{NH3} will be produced? ($20.2~g$)
	\item Because enthalpy is a state function, we can calculate values of $\Delta H$ without measuring them
	\item Hess's Law: Any alternate path with the same starting and ending states will have the same\\overall $\Delta H$
	\item Drawing an energy level diagram can help to illustrate Hess's law
	\item Find $\Delta H$ for this reaction: \ch{C_{diamond} + O2(g)->CO2(g)} \hspace{1em} $\left(\Delta H=-395.4~\nicefrac{kJ}{mol}\right)$

	      \ch{C_{diamond} -> C_{graphite}} \hspace{2em} $\Delta H=-1.9~\nicefrac{kJ}{mol}$

	      \ch{C_{graphite} + O2(g)->CO2(g)} \hspace{2em} $\Delta H=-393.5~\nicefrac{kJ}{mol}$
	\item The reverse of a reaction gives $-\Delta H$
	\item Find $\Delta H$ for this reaction: \ch{C(s) + 1/2 O2(g)->CO(g)} \hspace{1em} $\left(\Delta H=-110~\nicefrac{kJ}{mol}\right)$

	      \ch{C(s) + O2(g) -> CO2(g)} \hspace{2em} $\Delta H=-393~\nicefrac{kJ}{mol}$

	      \ch{CO(g) + 1/2 O2(g) -> CO2(g)} \hspace{2em} $\Delta H=-283~\nicefrac{kJ}{mol}$

	\item Double the reaction gives double the $\Delta H$
	\item Consider trying to find $\Delta H$ for the reaction below:

	      \circled{$\star$} \ch{C2H5OH(l) + 2 O2(g) -> 2 CO(g) + 3 H2O(l)} \hspace{2em} $\Delta H = ?$
	\item Find an alternate path using these reactions with known $\Delta H$:

	      \circled{A} \ch{C2H5OH(l) + 3 O2(g) -> 2 CO2(g) + 3 H2O(l)} \hspace{2em} $\Delta H=-1367~\nicefrac{kJ}{mol}$

	      \circled{B} \ch{CO(g) + 1/2 O2(g) -> CO2(g)} \hspace{2em} $\Delta H=-283~\nicefrac{kJ}{mol}$
	\item The enthalpy of the first reaction can be found from the enthalpies of the other two
	\item $\Delta H_{\star} = \Delta H_A - 2 \Delta H_B = -801\dfrac{kJ}{mol}$ \hspace{1em} (Draw the energy level diagram)

	      Practice: Find the enthalpy of reaction \circled{$\star$} using reactions \circled{A}, \circled{B}, and \circled{C}

	      \circled{$\star$} \ch{CS_2(l) + 3 O2(g) -> CO2(g) + 2 SO2(g)} \hspace{2em} $\left(\Delta H = -1075.0\right)$

	      \circled{A} \ch{C(s) + O2(g) -> CO2(g)} \hspace{2em} $\Delta H=-393.5~\nicefrac{kJ}{mol}$

	      \circled{B} \ch{S(s) + O2(g) -> SO2(g)} \hspace{2em} $\Delta H=-296.8~\nicefrac{kJ}{mol}$

	      \circled{C} \ch{C(s) + 2 S(s) -> CS2(l)} \hspace{2em} $\Delta H = 87.9~\nicefrac{kJ}{mol}$
\end{itemize}
\section{Standard Enthalpies of Formation}
\begin{itemize}
	\item To apply Hess's law to arbitrary reactions, you would need to devise an alternate path from an encyclopedia of known reactions -- this would be \emph{very} inconvenient
	\item Instead of using random reactions from one state to another, it is useful to devise a \emph{standard state} for each element
	\item The standard state is the most stable form of that element e.g. for O, it is \ch{O2(g)}, not \ch{O2(l)} or \ch{O3(g)}
	\item Each compound will have a \emph{standard formation reaction} which forms it from its elements in their standard state

	      For water, that's \ch{H2(g) + 1/2 O2(g) -> H2O(l)} \hspace{1em} (This is one time a \ch{1/2} coefficient is acceptable)
	\item The enthalpy for this reaction is called the compound's \emph{Standard Enthalpy of Formation} $\left(\Delta H^\circ_f\right)$
	\item Elements in their standard state have $\Delta H^\circ_f = 0$
	\item Any reaction can be framed as a combination of standard formation reactions
	      \begin{itemize}
		      \item First, the reactants are broken down into their elements (the \emph{reverse} of formation reactions)
		      \item Then, the elements are reassembled into the products (formation reactions)
		      \item The energy level diagram for any reaction is the same: reactants $\rightarrow$ elements $\rightarrow$ products
		      \item This pathway doesn't need to be \emph{practical}, it is enough to be theoretically \emph{possible}
		      \item $\Delta H_{rxn}=\sum\limits_{products}\nu\cdot\Delta H^\circ_f - \sum\limits_{reactants}\nu\cdot\Delta H^\circ_f$
		      \item This formula is general, for any reaction at all
		      \item Instead of an encyclopedia of thousands of reactions, we only need a table of $\Delta H^\circ_f$ values
		      \item Table 6.4 gives a few values, but appendix A2 is much more comprehensive
	      \end{itemize}
	      Practice: Find $\Delta H_{rxn}$ for~~ \ch{P4O10(s) + 6 H2O(l) -> 4 H3PO4(aq)} \hspace{1em} $\left(\Delta H_{rxn}=-1042.6~\dfrac{kJ}{mol}\right)$

	      ~\hphantom{Practice:} Find $\Delta H_{rxn}$ for~~ \ch{C3H8(g) + 5 O2(g) -> 3 CO2(g) + 4 H2O(g)} \hspace{1em} $\left(\Delta H_{rxn}=-2043.9~\dfrac{kJ}{mol}\right)$
\end{itemize}

\chapter{Gases}
\section{Gas Pressure}
\begin{itemize}
	\item Figure 7.1 shows the different phases from a molecular perspective
	\item The force a gas exerts is called pressure
	\item Pressure can be measured in many ways, including with a barometer (Figure 7.2)
	\item Atmospheric pressure depends on elevation, temperature, and humidity
	\item $1~atm$ is just a standard for pressure under certain conditions
	\item There are two other common units for pressure: $1~atm=760~torr=760~mmHg$
	\item And two less common units for pressure: $1~atm=101.325~kPa=1.01325~bar$
\end{itemize}
\section{Boyle's Law}
\begin{itemize}
	\item In the \nth{17} and \nth{18} century, scientists were studying how gases responded to temperature and pressure
	\item Boyle studied how volume changed with pressure at constant temperature (Figure 7.4)
	\item Higher pressures lead to proportionately lower volumes
	\item $V\propto \dfrac{1}{P}$ \hspace{2em} $PV=k$ \hspace{2em} $P_1V_1=P_2V_2$
\end{itemize}
\section{Charles's Law}
\begin{itemize}
	\item Charles studied how volume changed with temperature at constant pressure (Figure 7.10)
	\item Higher temperatures led to proportionately higher volumes
	\item Figure 7.9 shows how the $V$/$T$ curve can be extrapolated back to absolute $0$ -- Indeed, this method gave rise to the notion of absolute $0$
	\item Consider doubling the temperature at $1~K$ and at $1~^\circ C$. In the Kelvin scale, ``double the temperature'' has real, physical meaning (doubling the thermal energy)
	\item We \emph{must} use the Kelvin temperature scale in these types of problems
	\item $V\propto T$ \hspace{2em} $\dfrac{V}{T} = k$ \hspace{2em} $\dfrac{V_1}{T_1}=\dfrac{V_2}{T_2}$

\end{itemize}
\section{The Combined Gas Law}
\begin{itemize}
	\item Boyle's law and Charles's law can be combined to compare states with different temperatures, volumes, and pressures
	\item $\dfrac{P_1V_1}{T_1}=\dfrac{P_2V_2}{T_2}$

	      Practice: A balloon with $V=1.25~L$ at $P=0.850~atm$ and $T=284~k$ is released into the atmosphere where $P=0.430~atm$ and $T=243~K$. What is the new volume? ($2.11~L$)
\end{itemize}
\section{Avogadro's Law}
\begin{itemize}
	\item For the previous laws, the proportionality constants $k$ were always different for different gases
	\item Once we established molar masses, scientists could investigate the relationship between $n$ (\# of moles) and $P$, $V$, and $T$
	\item Avogadro's law states that $V$ is proportional to $n$ at constant $T$ and $P$
\end{itemize}
\section{Ideal Gas Law}
\begin{itemize}
	\item Avogadro's law was the final piece to get a universal gas constant. $k$ became $R$
	\item $\dfrac{PV}{nT}=R$ is usually rearranged as $PV=nRT$
	\item $R=0.08206~\dfrac{L~atm}{mol~K}$, but there are also other useful units (see Wikipedia page for $R$)
	\item This is called the \emph{ideal gas law}. An \emph{ideal gas} is one that follows this law perfectly
	\item Real gases will deviate from this law a little, but it is reliable at normal temperatures and pressures

	      Practice: Standard temperature and pressure are defined as $0.00~^\circ C$ and $1.00~atm$\\
	      Find the volume of $1.00~mol$ of an ideal gas at STP ($22.4~L$)
\end{itemize}
\section{Dalton's Law of Partial Pressures}
\begin{itemize}
	\item In a mixture of gases, each component will exhibit a pressure the same as if the others weren't there
	\item The total pressure is the sum of all these \emph{partial pressures}
	\item This means that we can find the total pressure of a mixture of gasses by counting the total moles of all gases present
	\item We can find the pressure of one component from the total pressure and the relative composition
	\item $P_A=\chi_A P_{total}$

	      Practice: \ch{O2} accounts for about $21\%$ of the molecules in our atmosphere. Find the partial pressure of \ch{O2} based on today's barometric pressure
\end{itemize}
\section{Molar Mass and Density in Gas Law Calculations}
\begin{itemize}
	\item We can modify the ideal gas law to relate the molar mass of a gas to its density
	\item $PV=nRT \rightarrow \dfrac{n}{V}=\dfrac{P}{RT} \rightarrow \dfrac{nM}{V} = \dfrac{PM}{RT} \rightarrow d\left(\dfrac{g}{L}\right)=\dfrac{PM}{RT}$
	\item Note that this is a density in $\dfrac{g}{L}$. Gases are very diffuse and have very low densities

	      Practice: There are two ways we commonly create buoyancy in the atmosphere: \ch{He}, and hot air. Calculate the density of regular air (assume pure \ch{N2}) and \ch{He} at STP, and the density of regular air at $325~^\circ C$ ($0.625~\nicefrac{g}{L}$, $0.179~\nicefrac{g}{L}$, and $0.285~\nicefrac{g}{L}$)

	      Practice: An unknown gas is found to have density of $0.575~\dfrac{g}{L}$ at $22.4~^\circ C$ and $0.87~atm$. Give a reasonable guess for the chemical identity of the unknown gas (\ch{CH4})

\end{itemize}
\section{Gases in Chemical Reactions}
\begin{itemize}
	\item We can also use the ideal gas law to predict volumes or pressures of a product in a chemical reaction

	      Practice: $0.25~g$ of \ch{Na} are reacted with acid according to the following equation:\\
	      \ch{2 Na(s) + 2 H^+(aq) -> 2 Na^+(aq) + H2(g)}\\
	      If $T=295~K$ and $P=0.875~atm$, what volume will the \ch{H2(g)} product occupy? ($0.150~L$)

	      Practice: Consider the reaction: \ch{2 Hg2O(s) -> 4 Hg(l) + O2(g)}\\
	      $2.50~g$ of \ch{Hg2O} are placed in a sealed $0.500~L$ chamber at $0.82~atm$ and decomposed through heat. Once thermal equilibrium is reached at $T=297.5~K$, what will be the final pressure inside the chamber? ($0.97~atm$)
\end{itemize}
\section{Kinetic Molecular Theory of Gases}
\begin{itemize}
	\item The kinetic molecular theory of gases is a model which can be used to derive the ideal gas law from first principles
	\item There are five postulates which define the theory:
	      \begin{enumerate}
		      \item Gases are composed of small particles that are in constant, random motion (Video 7.20)
		      \item The volume that is taken up by the molecules themselves is vanishingly small compared to the total volume of the gas
		      \item Forces between the particles are negligible
		      \item Molecular collisions are perfectly elastic
		      \item The average kinetic energy of the gas molecules is directly proportional to the temperature
	      \end{enumerate}
	\item These postulates are grounded in mathematical equations which can be analyzed through statistical dynamics to apply to large ensembles of particles and give observable state properties
	      \begin{itemize}
		      \item Pressure results from the billions of collisions between gas particles and the chamber walls
		      \item Adding more gas particles will increase the frequency of collisions, raising the pressure
		      \item The average kinetic energy depends only on temperature, not chemical identity of the gas. This explains Dalton's law of partial pressures
	      \end{itemize}
\end{itemize}
\section{Movement of Gas Particles}
\begin{itemize}
	\item The equation for the \th{5} postulate is: $KE_{avg}=\dfrac{3}{2}k_BT$
	\item $KE_{avg}$ is the average kinetic energy, and $k_B$ is the \emph{Boltzmann} constant
	\item We don't really need to use the Boltzmann constant because $N_Ak_B=R$, and it is useful to talk about molar amounts anyway
	\item So, we get $KE_{avg}=\dfrac{3}{2}RT$ \hspace{2em}  You should actually use $R=8.314~\dfrac{J}{mol~K}$
	\item We also know that $KE=\dfrac{1}{2}Mv^2$ (note that this is also a molar amount), so we can write: $\dfrac{1}{2}Mv^2 = \dfrac{3}{2}RT$
	\item This can be solved to give the rms average velocity for a gas particle: $v_{rms}=\sqrt{\dfrac{3RT}{M}}$
	\item Note the ``unit purgatory'' issue here -- you must use $J\equiv\dfrac{kg~m^2}{s^2}$, and express $M$ in $kg$
	\item \emph{rms velocity} is the square-root of the average of the squared velocities
	\item $v_{rms}$ is used rather than mean velocity because the temperature is proportional to $v^2$, rather than $v$ itself
	\item Figures 10.13 and 10.14 from BLMB show different velocity distributions for different gases and conditions

	      Practice: Find $v_{rms}$ for \ch{N2}, \ch{SF6}, and \ch{He} at $-20.00~^\circ C$ and $150.0~^\circ C$

	      ~\hphantom{Practice: } $-20~^\circ C$($474.8~\nicefrac{m}{s}$, $207.9~\nicefrac{m}{s}$, $1256~\nicefrac{m}{s}$)\hspace{1em} $150~^\circ C$($613.8~\nicefrac{m}{s}$, $269.4~\nicefrac{m}{s}$, $1624~\nicefrac{m}{s}$)
	\item Diffusion is the spread of a gas from areas of high concentration to areas of low concentration
	      \begin{itemize}
		      \item Diffusion happens much more slowly than $v_{rms}$ because the molecules collide so frequently
		      \item Gas molecules take a random walk, deflected in a new direction with each collision
		      \item The \emph{mean free path} is only about $70~nm$ for STP
		      \item Higher pressures decrease the mean free path and slow down diffusion
		      \item In a vacuum, the diffusion rate approaches $v_{rms}$
	      \end{itemize}
	\item Effusion is the slow leak of a gas through a hole
	\item These two processes both depend on $v_{rms}$
	\item Graham's law compares the effusion/diffusion rates of two gases: $\dfrac{r_1}{r_2}=\sqrt{\dfrac{M_2}{M_1}}$

	      Practice: Effusion can be observed by reinforcing a balloon with tape, then poking a hole through\\
	      ~\hphantom{Practice: } the tape

	      ~\hphantom{Practice: } Compare the effusion rate for a \ch{He} and \ch{N2} balloon $\left(r_{\ch{He}}=2.65r_{\ch{N2}}\right)$
\end{itemize}
\section{Behavior of Real Gases}
\begin{itemize}
	\item The ideal gas law is \emph{technically} only valid for an ideal gas with no attractive/repulsive forces
	\item Real gas particles \emph{are} attracted to each other, albeit weakly
	\item These attractions are particularly important at high pressures and temperatures
	\item For one mole of an ideal gas, $\dfrac{PV}{RT}=1$ at all temperatures and pressures
	\item Figure 7.26 shows how $\dfrac{PV}{RT}$ varies with pressure and temperature
	\item Figures 10.19 and 10.20 from BLMB show the compression factor for different gases and conditions
	\item The van der Waals equation is an improvement on the ideal gas law, which works better for real gases
	      \begin{itemize}
		      \item $\left[P+a\left(\dfrac{n}{V}\right)^2\right]\left(V-nb\right)=nRT$ \hspace{1em} --or-- \hspace{1em} $P=\dfrac{nRT}{V-nb}-a\left(\dfrac{n}{V}\right)^2$
		      \item The $a$ term accounts for attractions between gas particles and reduces the pressure of real gases
		      \item The $b$ term accounts for the finite volume of gas particles and increases the pressure of real gases
		      \item Table 7.4 gives van der Waals $a$ and $b$ terms for many common gases
	      \end{itemize}
	      Practice: Use both the ideal gas law and the van der Waals equation to find the pressure for:\\
	      ~\hphantom{Practice: } $1.00~mol$ of \ch{CCl4} at $250~K$ in $10.0~L$ and $1.00~L$

	      ~\hphantom{Practice: } ($2.0515~atm$, $1.881~atm$, $20.515~atm$, and $37.79~atm$)
\end{itemize}

\chapter{The Quantum Model of the Atom}
\section{A Brief Exploration of Light}
\begin{itemize}
	\item Light is an electromagnetic wave, which can be thought of like a wave on a lake (Figure 8.2)
	\item Light has a wavelength, frequency, and speed according to the equation: $\nu\lambda=c$
	\item The speed of light is a constant, $2.998\times10^8~\dfrac{m}{s}$
	\item How far does light travel in $5.00~ms$? ($1.50~km$)
	\item The electromagnetic spectrum is more than just visible light (Figure 8.4)
	      \begin{itemize}
		      \item Higher frequencies (shorter wavelengths) are UV light, X-rays, and gamma rays
		      \item Lower frequencies (longer wavelengths) are infrared light, microwaves, and radio waves
		      \item TV-remotes are flashy lights, and radio towers are flashlights-on-a-stick
		      \item There are important technical differences in how we can use these different kinds of light, but they are fundamentally the same thing (an alternating electromagnetic wave)
		      \item Find the wavelength of your favorite radio station ($M\!H\!z$ is a frequency of $10^6~s^{-1}$)
	      \end{itemize}
	\item Light-matter interactions were central to the discovery of modern physics
	\item The photoelectric effect was an important matter/light interaction
	      \begin{itemize}
		      \item Sometimes light falling on a metal will eject an electron -- this is the photoelectric effect
		      \item The kinetic energy of the ejected electron can be measured
		      \item The energy depended on the wavelength of light -- bluer light ejected electrons at faster velocities
		      \item There was a threshold where electron ejection stopped, and redder light would have no effect
		      \item This was surprising, because light intensity had \emph{no} effect on the photoelectron energy
		      \item Dim blue light would eject fast electrons, bright red light would have no effect
		      \item This was eventually explained by the idea that light carries energy in small discrete packets
		      \item These packets of energy are called \emph{photons} and the energy they carry depends on the wavelength
		      \item The photoelectric effect could be described by the equation: $KE = h\nu - \phi$
		      \item Here, $\nu$ was a new constant, called Planck's constant, and $\phi$ was the metal's work function
		      \item The equation for the energy of light was: $E=h\nu$ where $h$ is Planck's constant
	      \end{itemize}
	\item Light can also be \emph{absorbed} or \emph{emitted} by matter
	      \begin{itemize}
		      \item Light is emitted by gases like \ch{Ne}, \ch{Ar}, or \ch{Na} when high voltage passes through it
		      \item Light can also be absorbed by gases and other materials
		      \item Each substance showed a unique fingerprint of wavelengths of light emitted or absorbed (Figure 8.5)
		      \item \ch{He} was first identified by its absorbance spectrum in sunlight
		      \item The unique spectra arise from the particular energy levels of a substance
		      \item These spectra showed how matter can only store or release energy in certain, constrained amounts (or \emph{quanta})
		      \item i.e. in addition to the energy of light, the energy of matter was \emph{quantized} as well
	      \end{itemize}
	\item Wave/Particle duality (not in the textbook)
	      \begin{itemize}
		      \item These experiments show that light behaves like both a \emph{wave} (interference) and a \emph{particle} (quantization)
		      \item It is wrong to say light is either of those things -- rather, it is a new thing with similarities to both (rhinosceros vs dragon + unicorn)
		      \item On very small scales, matter behaves like both a wave and a particle as well!
		      \item Electrons, in particular, are strongly wave-like, with a characteristic wavelength
		      \item The IBM quantum corral image dramatically showed the real physicality of electron waves
		      \item This wave-like nature of electrons is important for understanding modern models of atomic structure
	      \end{itemize}
\end{itemize}
\section{The Bohr Model of the Atom}
\begin{itemize}
	\item The Rydberg Equation:
	      \begin{itemize}
		      \item School teacher Rydberg recognized a pattern in the wavelengths of light in the \ch{H} spectrum
		      \item His equation can be re-written in terms of energy
		      \item $E=2.179\times10^{-18}~J\left(\dfrac{1}{n_1^2}-\dfrac{1}{n_2^2}\right)$
	      \end{itemize}
	\item The Bohr model of the atom:
	      \begin{itemize}
		      \item There was no real explanation for \emph{why} the absorption and emission spectra of different elements showed different discreet energies
		      \item Niels Bohr proposed that electrons orbit around the nucleus only at fixed distances
		      \item Absorption is when an electron shifts to a higher orbit, using a photon's energy
		      \item Emission is when an electron shifts to a lower orbit, releasing energy as a photon
		      \item The lowest energy state is the \emph{ground state}, all others are \emph{excited states}
		      \item The discrete orbits represent states where the circumference of the orbit is equal to a number of wavelengths for the electron
	      \end{itemize}
	      Practice: What is the wavelength of the $2\leftarrow4$ transition in the \ch{H} spectrum? ($486.1~nm$)
\end{itemize}
\section{Electron Shells, Subshells, and Orbitals}
\begin{itemize}
	\item Quantum mechanics continued to develop
	\item Heissenberg Uncertainty Principle: We cannot simultaneously measure the position and velocity of an electron (or any other quantum mechanical particle)
	\item This means that we generally speak of where an electron is \emph{most probable} to be found, rather than where it \emph{actually is}
	\item The Schr\"odinger wave equation describes matter starting from a wave-like perspective
	      \begin{itemize}
		      \item The Schr\"odinger equation gives mathematical functions which describe the electron probability distribution
		      \item Each solution is called an \emph{orbital}, like the orbits of the Bohr model but 3-dimensional
	      \end{itemize}
	\item Orbitals are organized into \emph{shells} and \emph{subshells}
	      \begin{itemize}
		      \item Subshells are groups of orbitals with similar shapes and the same energy
		      \item Subshells are named $s$, $p$, $d$, and $f$
		      \item An $s$ subshell has only one orbital (2 $e$s), $p$ has 3 (6 $e$s), $d$ has 5 (10 $e$s), and $f$ has 7 (14 $e$s)
		      \item Subshells are grouped into shells, which are indicated by numbers ($1$, $2$, $3$, etc.)
		      \item These numbers are the numbers in the Rydberg equation, and are the principle energy levels
		      \item The first shell only has an $s$ subshell, and each shell beyond that adds one type
	      \end{itemize}
	      Practice: How many electrons can be placed in a $p$ subshell? ($6$)

	      ~\hphantom{Practice:} How many electron can be placed in the \nth{3} shell? ($18$)

	      ~\hphantom{Practice:} Which of the following subshells does \emph{not} exist? $2s$, $3f$, $3p$, $5d$ ($3f$)
	\item Subshells each have orbitals with different \emph{shapes} (Figure 8.14)
	      \begin{itemize}
		      \item Because of the Heissenberg uncertainty principle, we describe region where an electron is likely to be found
		      \item These regions have shapes based on the mathematical functions which form them
		      \item $s$ orbitals are spherical, $p$ orbitals are dumbells, $d$ orbitals are clover-leafed
	      \end{itemize}
\end{itemize}
\section{Energy-Level Diagrams}
\begin{itemize}
	\item The orbitals within each subshell are precisely equal in energy (degenerate)
	\item The subshells themselves differ in energy (their order will be explained shortly)
	\item Draw the subshells with one line for each orbital
	\item Each orbital can hold two electrons, drawn as up- and down-arrows
	\item Find the total number of electrons for the element or ion
	\item Aufbau Principle -- Fill up the orbitals with electrons from the bottom-up
	\item Hund's Rule -- Fill a subshell with one electron in each orbital before pairing them up (like roommates in an apartment)
	\item This is the ground-state configuration of the element
\end{itemize}
\section{Electron Configurations}
\begin{itemize}
	\item We can use the periodic table as a cheat-sheet to the order of the subshells and electron configurations
	      \begin{itemize}
		      \item The P. T. is actually quite long -- Lanthanides and Actinides have been cut and pasted
		      \item Each region of the P. T. represents a different subshell
		      \item The rows represent different shells
		      \item The number of elements in each block is the number of electrons each shell can hold
		      \item The $d$-block and $f$-block trail the row number by 1 and 2 ($d(-1)$ and $f(-2)$)
		      \item The order of the subshells is found by simply following the elements and noting in which block they reside
	      \end{itemize}
	\item The arrangements of the electrons can be written as an electron configuration
	\item Write the subshells, with their number of electrons as a superscript

	      Practice: Write the electronic configurations for \ch{O}, \ch{Zr}, and \ch{Bi}
	\item Especially for large elements like \ch{Pb}, these configurations are very unwieldy
	\item We can shorten them by referencing the configuration of the \emph{preceding} noble gas
	\item The electrons which make up this noble gas configuration are buried inside the atom, and called \emph{core} electrons
	\item For \ch{Bi}, we get $\left[\ch{Xe}\right]6s^24f^{14}5d^{10}6p^3$
	\item The outermost electrons (The ones we write) are called \emph{valence} electrons, and are the ones involved in bonding and ion formation
	\item A few transition metals have anomalous configurations (memorize only \ch{Cr} and \ch{Cu})
	\item The Lanthanum and Actinium boundary also shows some anomalies (don't memorize them)
	\item We can write the configurations of ions as well
	      \begin{itemize}
		      \item For most, simply add or remove electrons according to the normal pattern
		      \item Transition metals will lose the outermost $s$ electrons before they lose any $d$ electrons (This is why so many transition metals have a stable $2+$ ion)
		      \item Ions and atoms with identical configurations are called \emph{isoelectronic} to each other
	      \end{itemize}
	      Practice: List several stable ions which are isoelectronic with \ch{Ar}
\end{itemize}
\section{Quantum Numbers}
\begin{itemize}
	\item Remember that orbitals are actually mathematical functions
	\item Certain parts of those functions depend on integer numbers (like $n$ in the Rydberg equation)
	\item These integer numbers are called \emph{quantum numbers}
	\item Quantum numbers can be thought as an ``address'' for each electron (Street, Buildling, Unit, Name)
	      \begin{itemize}
		      \item $n$ -- Principal quantum number ($1$, $2$, \ldots) gives orbital shell, energy, and size
		      \item $l$ -- Angular momentum quantum number ($0$, $1$, \ldots, $n-1$) gives orbital subshell and shape
		      \item $l$ is why not all shells have all orbital types
		      \item $m_l$ -- Magnetic quantum number ($-l$, \ldots, $l$) gives orbital within a subshell
		      \item $m_l$ is why subshells have different numbers of orbitals
		      \item $m_s$ -- Spin quantum number $\left(\pm \frac{1}{2}\right)$ gives ``spin'' of the electron (up- or down- arrow)
	      \end{itemize}
	\item Each electron in an atom/ion must have a \emph{unique} set of quantum numbers -- Pauli Exclusion Principle
	\item You should be able to point to the right electron given a set of quantum numbers, or give the 4 quantum numbers for an indicated electron in an energy level diagram

	      Practice: Give numbers or indicate electrons on an energy level diagram
\end{itemize}

\chapter{Periodicity and Ionic Bonding}
\section{Valence Electrons}
\begin{itemize}
	\item Technically, filled $d$ and $f$ subshells count as \emph{core}, and only the outermost $s$ and $p$ electrons will always count as \emph{valence}
	\item This means that only the main group elements have a reliable pattern in their number of valence electrons
	\item The number of valence electrons is the same as the ``A'' column names
\end{itemize}
\section{Atomic and Ionic Sizes}
\begin{itemize}
	\item The sizes of atoms and ions is controlled by the attractive and repulsive forces between electrons and protons
	      \begin{itemize}
		      \item Electron and protons attract each other, shrinking the atomic size
		      \item Electrons repel each other, increasing the atomic size
	      \end{itemize}
	\item Effective nuclear charge $\left(Z_{eff}\right)$ attempts to summarize these interactions
	      \begin{itemize}
		      \item The actual nuclear charge ($Z$) is just the number of protons (quite high for larger elements)
		      \item Core electrons will counteract much of the actual nuclear charge (called \emph{shielding})
		      \item $Z_{eff} = Z-S$
		      \item $S$ can be closely approximated by the number of core electrons

		            Practice: Find $Z_{eff}$ for \ch{Mg}, \ch{S}, and \ch{Br} ($2$, $6$, and $7$)
		      \item Slater's rules gives a more sophisticated and accurate value for $S$
		      \item $1$ for deep core electrons, $0.85$ for $V-1$ electrons, and $0.35$ for $V$ all but one $V$ electron
		      \item Note that this is different than what the textbook gives

		            Practice: Find $Z_{eff}$ for the same elements using Slater's rules ($2.85$, $5.45$, and $7.6$)
	      \end{itemize}
	\item Atomic radius increases down a column because you are adding and entire new shell for each row
	\item Atomic radius decreases across a row because of the increasing $Z_{eff}$
	\item This makes \ch{He} the smallest element, and \ch{Fr} the largest (Figure 9.3)
	\item Anions are much larger and cations are much smaller than their neutral counterparts (Figure 9.4)
\end{itemize}
\section{Ionization Energy and Electron Affinity}
\begin{itemize}
	\item Ionization energy (IE) is the energy required to remove an electron
	      \begin{itemize}
		      \item For example, it is the energy for this process: \ch{Li -> Li^+ + e^-}
		      \item The two factors which control IE are radius, and $Z_{eff}$
		      \item It is easier to remove electrons (smaller IE) from larger atoms
		      \item It is easier to remove electrons (smaller IE) from atoms with lower $Z_{eff}$
		      \item Opposite to radius, \ch{He} has the highest IE, and \ch{Fr} has the lowest IE
		      \item There are breaks in this trend at the beginning and middle of the $p$ block (Figure 9.6)
	      \end{itemize}
	\item Second- and third- ionization energy is the energy to remove a second and third electron
	      \begin{itemize}
		      \item Each successive electron is harder to remove
		      \item After the valence electrons are gone, removing a core electron is \emph{much} harder to remove
	      \end{itemize}
	\item Electron affinity (EA) is the energy released when an electron is added
	      \begin{itemize}
		      \item Electron affinity is usually exothermic, so these values are mostly negative
		      \item The \emph{magnitude} of EA follows the same trend as IE
		      \item There are lots of breaks in the trend (Figure 9.9 is a mess), so don't worry too much about EA
	      \end{itemize}
\end{itemize}
\section{Ionic Bonding}
\begin{itemize}
	\item Metal and nonmetal elemental atoms will react to form an ionic compound
	\item The number of electrons gained/lost will result in noble gas configurations for both elements
	\item The cation and anion are now attracted to each other, and bind together in a lattice structure
	\item A \emph{formula unit} is the smallest unit which builds the extended lattice (Figure 9.11)
\end{itemize}
\section{Lattice Energy}
\begin{itemize}
	\item Consider the reaction: \ch{K(s) + 1/2 Cl2(g) -> KCl(s)} \hspace{1em}$\Delta H^\circ_f = -437~\dfrac{kJ}{mol}$
	\item The energy of formation is more than just the IE of the metal and EA of the non-metal
	\item The Born-Haber Cycle is an alternate route for formation reactions of ionic compounds:
	      \begin{itemize}
		      \item The steps of the Born-Haber cycle are shown in Figure 9.13
		      \item Sublimating the metal, breaking the diatomic bond, IE, EA, and lattice energy
		      \item The lattice energy is directly related to the attractive force between the ions
	      \end{itemize}
	\item Lattice energy is controlled by two factors:
	      \begin{itemize}
		      \item Ions with larger charges have higher lattice energies (Table 9.2)
		      \item Smaller ions have higher lattice energies (Figure 9.14)
	      \end{itemize}
\end{itemize}

\chapter{Covalent Bonding}
\section{Formation of Covalent Bonds}
\begin{itemize}
	\item In both ionic and covalent compounds, bonds will form to complete the atoms' \emph{octets} (\emph{duets} in the case of \ch{H})
	\item Instead of transferring electrons to form ions, covalent compounds will share electrons
	\item The electrons in a covalent bond will count toward the octets of both bonding partners
	\item Drawing dot structures, we can see how many bonds an atom might need to form to fill its octet
	\item We'll usually represent two shared electrons in a bond by a dash -- this is the beginning of Lewis structures
	\item Electron pairs not involved in a bond are called \emph{lone pairs}
	\item Single, Double and Triple bonds share 2, 4, and 6 electrons, respectively
	\item Double and triple bonds are shorter and stronger than single bonds

	      Practice: Draw dot diagrams and Lewis diagrams for the diatomics \ch{F2}, \ch{O2}, and \ch{N2}\\
	      ~\hphantom{Practice: } Identify the bonds by their type, as well as any lone pairs
\end{itemize}
\section{Lewis Structures}
\begin{itemize}
	\item For more complex molecules, dot diagrams are not robust enough to predict the proper structure
	\item Follow these 8 steps:
	      \begin{itemize}
		      \item Find the total number of valence electrons, considering any overall charge
		      \item Write the peripheral atoms around the central atom
		            \begin{itemize}
			            \item Central atom will be the least electronegative (except \ch{H})
			            \item This is often the first atom written in a formula (except \ch{H})
			            \item \ch{H} will never be the central atom because it can only form 1 bond
		            \end{itemize}
		      \item Connect each peripheral atom to the central atom with single bonds
		      \item Place the remaining valence electrons around the peripheral atoms until their octets are complete
		      \item If any electrons remain, place them on the central atom
		      \item Convert peripheral lone pairs into double or triple bonds according to these two rules
		            \begin{itemize}
			            \item Make multiple bonds until the central atom has a complete octet
			            \item Make multiple bonds in a way that minimizes \emph{formal charges} (more on this later)
		            \end{itemize}
		      \item Verify that the octet and duet rule are followed for all atoms in the structure
		      \item If there is a charge, enclose the structure in square brackets and write the charge
	      \end{itemize}

	      Practice: \ch{CH4}, \ch{H2O}, \ch{NH3}, \ch{HCN}, \ch{CH2O}, \ch{NO3^-}, \ch{CO3^{2-}}, and \ch{NH4^+}
\end{itemize}
\section{Resonance and Formal Charges}
\begin{itemize}
	\item Often times we will have an arbitrary choice about which peripheral atom to form a double bond with
	\item Consider \ch{CO3^{2-}}, you could form the double bond with any of the three oxygens
	\item In these instances, the molecules will exhibit a phenomenon called \emph{resonance}
	      \begin{itemize}
		      \item Resonance is when electrons in a double (or triple) bond are shared between two or more bonding locations
		      \item \ch{CO3^{2-}} forms a bond with \emph{all three} oxygen atoms
		      \item It is not nearly as strong as a normal double bond, because the electrons are spread between three bonding centers
		      \item This is called \emph{delocalization}, and sometimes the trio of bonding locations are collectively called a delocalized bond
		      \item The bond lengths are all the same, somewhere between a single and double bond length
	      \end{itemize}
	\item We represent resonance two ways: First, resonance structures
	      \begin{itemize}
		      \item Draw a different structure for each bonding location
		      \item Enclose each structure in square brackets (even for neutral structures)
		      \item Draw double-headed arrows between the structures
	      \end{itemize}
	\item The other method is hybrid structures
	      \begin{itemize}
		      \item Draw a single structure, with dotted lines for each bonding location
		      \item These structures are closer to an accurate picture of reality
		      \item Counting electrons in these structures is impossible (bonds and lone-pairs)
	      \end{itemize}

	      Practice: Draw the resonance structures and hybrid structure for \ch{NO3^-} and \ch{O3}
	\item Electrons are not actually shared equally between bonding partners
	      \begin{itemize}
		      \item \emph{Electronegativity} is the measure of how strongly an atom pulls on shared electrons
		      \item Electronegativity increases across a row and up a column (Figure 10.7)
		      \item Less electronegative atoms will be central atoms (except \ch{H})
	      \end{itemize}
	\item Formal charge compares how many electrons an atom has within an molecule to its number of valence electrons
	      \begin{itemize}
		      \item First draw a Lewis structure
		      \item Count the electrons around an atom, dividing the bonding electrons between bonding partners
		      \item Subtract this number from the number of valence atoms
	      \end{itemize}
	\item Formal charge can be used to determine which structures are best when there are options
	      \begin{itemize}
		      \item Minimize the total formal charge
		      \item Favor structures with negative formal charges on the more electronegative atoms
	      \end{itemize}

	      Practice: Find the best structures for \ch{OCN} and \ch{N2O}
\end{itemize}
\section{Exceptions to the Octet Rule}
\begin{itemize}
	\item A few elements will have \emph{fewer} than 8 electrons
	      \begin{itemize}
		      \item \ch{Be} has only 2 electrons, so it can form only 2 bonds
		      \item \ch{B} has only 3 electrons, so it can form only 3 bonds
	      \end{itemize}
	\item Radicals are compounds with an unpaired electron
	      \begin{itemize}
		      \item Radicals tend to be very reactive
		      \item Many radicals contain \ch{N} (\ch{NO} and \ch{NO2})
	      \end{itemize}
	\item Some elements can have \emph{more} than 8 electrons
	      \begin{itemize}
		      \item Only elements in the 3rd row or below can exceed the octet rule
		      \item This is because the extra electrons go into the $d$ subshell
		      \item This can occur in order to minimize formal charges

		            Practice: Find proper Lewis structures for \ch{SO4^{2-}}, \ch{PO4^{3-}}, \ch{ClO3^{-}}, and \ch{ClO4^{-}}
		      \item It can also occur because there are simply too many outer atoms

		            Practice: Find proper Lewis structures for \ch{SF6}, \ch{ClF5}, and \ch{PCl5}
	      \end{itemize}
\end{itemize}
\section{Polar Bonds and the Bonding Continuum}
\begin{itemize}
	\item Figure 10.14 shows electrons being shared equally and unequally between bonding partners
	\item The electrons will favor the element with greater electronegativity
	\item The uneven distribution of charge is called a \emph{dipole} and the bond is called \emph{polar}
	\item Polar covalent bonds exist along a continuum from purely covalent to purely ionic
	\item Greater electronegativity differences give greater ionic character to the bond
	\item Similar (or identical) electronegativities create non-polar bonds
	\item The dipole moment is $\mu=qr$, and can be measured
	\item Real dipole moments can be compared to the value for a complete transfer of electrons to give $\%~ionic~character$
	\item $\%~ionic~character=\dfrac{\mu_{measured}}{\mu_{ionic}}\times100\%$
\end{itemize}
\section{Bond Enthalpy}
\begin{itemize}
	\item Bond enthalpy is the energy required to break a bond
	\item Really, not every \ch{C-H} bond is the same. The surrounding atoms affect bond enthalpy
	\item We can take the average enthalpy of a given bond type over many molecules (Table 10.3)
	\item This gives us a new way to calculate reaction enthalpies
	      \begin{itemize}
		      \item Draw a Lewis structure for each reactant and product
		      \item identify the numbers and types of bonds
		      \item Consider first breaking all the bonds of the reactants to produce individual atoms
		      \item Then form new bonds between the atoms to make products
		      \item $\Delta H_{rxn}=\sum\limits_{Bonds~Broken}\Delta n\cdot H_{Bond} - \sum\limits_{Bonds~Formed}n\cdot \Delta H_{Bond}$
		      \item This will give an \emph{approximate} reaction enthalpy
	      \end{itemize}
	\item Table 10.4 gives average values for bond lengths as well

	      Practice: Using Table 10.3, estimate the reaction enthalpy for the combustion of methane $\left(-810~\dfrac{kJ}{mol}\right)$
\end{itemize}

\chapter{Molecular Shape and Bonding Theories}
\section{VSEPR and Molecular Geometry}
\begin{itemize}
	\item The three-dimensional shape of a molecule is important to its properties
	\item \emph{Isomers} are molecules with the same chemical formula, but a different 3-dimensional arrangement of its atoms
	\item Isomers might have quite different properties despite their identical composition
	\item Valence Shell Electron Pair Repulsion (VSEPR) model:
	      \begin{itemize}
		      \item VSEPR is based on the idea that electron pairs will arrange themselves to be as far apart from each other as possible.
		      \item This model gives accurate geometries for covalent molecules
		      \item First, draw a good Lewis structure
		      \item \emph{Electron Domains} are the regions around the central atom where electrons group -- A single bond, a double bond, a triple bond, and a lone pair are all electron domains
		      \item The number of electron domains will give the \emph{electron geometry} -- This electron geometry is the template on which molecular geometry is based
		      \item Demo: Balloons naturally adopt the electron geometries
		      \item Next, count how many domains are bonding vs lone pairs
		      \item The number of bonds within the electron geometry determines the molecular geometry
		      \item Table 11.1 sums it all up nicely -- Just memorize this table
		      \item Trigonal Bipyramidal electron geometry
		            \begin{itemize}
			            \item The 2 axial and 3 equatorial positions are different
			            \item Lone pairs will occupy the equatorial positions first
			            \item Linear molecules are symmetrical (this will matter later)
		            \end{itemize}
		      \item Octahedral electron geometry
		            \begin{itemize}
			            \item All the positions are equivalent
			            \item The second lone pair will be opposite the first one
			            \item Square planar molecules are symmetrical (this will matter later)
		            \end{itemize}
	      \end{itemize}
	\item For larger molecules, you can apply VSEPR to each bonding center (consider \ch{CH3CO2H})
\end{itemize}
\section{Polar and Non-polar Molecules}
\begin{itemize}
	\item We already discussed electronegativity and its role in making polar \emph{bonds}
	\item For molecules with many polar bonds, those dipoles might cancel each other out or work together to make a polar molecule
	\item The \emph{molecular dipole} is the vector sum of all the bond dipoles
	\item Factors that make non-polar molecules:
	      \begin{itemize}
		      \item No polar bonds like diatomic elements and \ch{O3}
		      \item Symmetry in the polar bonds (no lone pairs, or linear and square planar molecules)
	      \end{itemize}
	\item Factors that make polar-molecules:
	      \begin{itemize}
		      \item Lone pairs which break symmetry
		      \item Bonds with different atoms (\ch{CH2Cl2})
	      \end{itemize}
\end{itemize}
\section{Valence Bond Theory: Hybrid Orbitals and Bonding}
\begin{itemize}
	\item Covalent bonding basics:
	      \begin{itemize}
		      \item Why should sharing electrons lead to a bond?
		      \item Draw the attractive and repulsive forces in \ch{H2}
		      \item When the orbitals from both bonding partner overlap, these forces are optimized to form a bond
		      \item Figure 11.17 shows how the energy changes with internuclear distance in \ch{H2}
	      \end{itemize}
	\item Orbital overlap cannot explain the actual geometries we see in molecules. No atomic orbitals have a tetrahedral or trigonal planar geometry, for example
	\item Hybridization is the linear combination (mixing) of atomic orbitals
	      \begin{itemize}
		      \item Remember that orbitals are simply mathematical functions
		      \item Consider the functions $f_1(x) = x^2$, $f_2(x) = x^3$, and $g(x) = f_1(x) + f_2(x)$
		      \item Atomic orbitals can similarly be combined to form new orbitals, called \emph{hybrid} orbitals
		      \item Show my hybridization figures
		      \item The number of hybrid orbitals is equal to the number of atomic orbitals mixed together
		      \item We name the orbitals and the type of hybridization by the atomic orbitals used ($sp$, $sp^2$,and $sp^3$)
		      \item These hybridization types correspond to the electron geometries
	      \end{itemize}
	\item Hybrid orbitals and bonding
	      \begin{itemize}
		      \item Hybrid orbitals are used for 2 purposes: forming $\sigma$ bonds, and housing lone pairs
		      \item $\sigma$ bonds have electron density aligned along the bond axis
		      \item Single bonds are $\sigma$ bonds, and multiple bonds each contain a $\sigma$ bond
		      \item For $sp$ and $sp^2$ hybridized atoms, unhybridized $p$ orbitals remain (Figures 11.24 and 11.26)
		      \item These unhybridized $p$ orbitals are used to form $\pi$ bonds
		      \item $\pi$ bonds have electron density along either side of the bond axis
		      \item Double bonds contain one $\pi$ bond, and triple bonds contain 2 $\pi$ bonds
		      \item A double bond is like a hot dog in a bun -- a triple bond is like a hot-dog with 2 buns
		      \item Use BLMB Figures 9.23-9.28
		      \item Multiple $p$ orbitals overlap in molecules which exhibit resonance (like benzene and nitrate)
	      \end{itemize}
\end{itemize}
\section{Using Valence Bond Theory}
\begin{itemize}
	\item Molecules can twist around their bonds without changing bond lengths or angles
	\item These rotations create different \emph{conformations} (show with molecular models)
	\item The differences between $\sigma$ and $\pi$ bonds has significance to how molecules can move
	\item $\sigma$ bonds have the same overlap no matter how you rotate them, so rotation does not break the bond
	\item This makes rotating around $\sigma$ bonds energetically free and easy
	\item $\pi$ bonds, however, would break if you twist them by $90^\circ$ because they lose their overlap (Figure 11.3)
	\item $\pi$ bonds are constrained, and cannot rotate without a substantial energy input
\end{itemize}
\section{Molecular Orbital Theory}
\begin{itemize}
	\item In hybrid orbital theory, we combined atomic orbitals from the same atom to create new hybrid orbitals
	\item These hybrid orbitals had new shapes, which could explain bonding geometries
	\item We can apply the same trick to molecules - combine atomic orbitals from \emph{different} atoms across a molecule to create new orbitals
	\item These new orbitals span the whole molecule, and are called \emph{molecular orbitals}
	\item Molecular orbitals are the closest approximations to the true Schr\"odinger equation solutions
	\item Molecular orbitals can combine in two ways (Figure 11.32):
	      \begin{itemize}
		      \item Adding orbitals together creates constructive interference between the orbitals
		      \item Orbitals which combine constructively increase the electron density between the atoms
		      \item This leads to stronger bonds, so these MOs are called \emph{bonding} orbitals
		      \item Subtracting orbitals creates destructive interference between the orbitals
		      \item Orbitals which combine destructively decrease the electron density between the atoms
		      \item This leads to weaker bonds, so these MOs are called \emph{antibonding} orbitals (and get a $^*$ label)
		      \item Both of the orbitals formed from $s$ orbitals have electron density along the bond axis, so they are called the $\sigma_{ns}$ and $\sigma_{ns}^*$ orbitals
	      \end{itemize}
	\item Molecular orbital diagrams
	      \begin{itemize}
		      \item When we draw molecular orbital diagrams, it is helpful to show the atomic orbitals which gave rise to the MOs
		      \item Draw the atomic energy level diagram for each atom on either side of the MO energy levels
		      \item Draw dotted lines showing how the atomic orbitals combine to produce new MOs
		      \item Fill the MO energy levels from the bottom-up as usual
	      \end{itemize}
	\item $p$ orbitals can combine as well
	      \begin{itemize}
		      \item Head-on combination creates two more $\sigma$ orbitals (Figure 11.35)
		      \item Sideways combination creates orbitals with density along either side of the bond axis, so they are $\pi_{2p}$ and $\pi^*_{2p}$ MOs
		      \item Because there are two pairs of $p$ orbitals that can combine this way, there are two degenerate $\pi_{2p}$ and two degenerate $\pi^*_{2p}$ orbitals
		      \item The ordering of the $\pi_{2p}$ and $\sigma_{2p}$ orbitals is different for different elements (Figure 11.37)
	      \end{itemize}
	\item Figure 11.38 shows the configurations for all period 2 homonuclear diatomics (note the different orbital ordering)
	\item We can tell the \emph{bond order} (single, double, or triple bond) of a diatomic from its MO diagram
	      \begin{itemize}
		      \item Electron pairs which occupy a bonding orbital increase the bond order by 1
		      \item Electron pairs which occupy a non-bonding orbital decrease the bond order by 1
		      \item Molecules with a bond order of 0 (\ch{Ne2}) don't form a bond at all, and so don't occur naturally
	      \end{itemize}
	\item Magnetic properties are also indicated by the MO diagram
	      \begin{itemize}
		      \item \emph{All} materials are either attracted to or repelled by a magnetic field
		      \item \emph{Diamagnetic} materials are repelled by a magnetic field, while \emph{paramagnetic} materials are attracted by a magnetic field. Ferromagnetism (regular old magnetism) is like a cooperative paramagnetic effect in some metals
		      \item Diamagnetism comes from having all electrons paired. \ch{C2}, \ch{N2}, and \ch{F2} are diamagnetic
		      \item Paramagnetism comes from at least one unpaired electron. \ch{B2} and \ch{O2} are paramagnetic
	      \end{itemize}
	\item Molecules that exhibit resonance will have $\pi$ bonding MOs which span all of the bonding locations
	\item Some molecules without traditional resonance will still show delocalization (1,3-butadiene)
	\item Show some MOs in Avogadro on my laptop
\end{itemize}


\backmatter
\chapter{Errata}
\begin{itemize}
	\item Figure 1.13 identifies (c) as neither precise nor accurate, but it \emph{is} accurate within the apparent precision
	\item In Section 4.7, the decomposition of mercury(II) oxide should be: \ch{2 HgO(s) -> 2 Hg(l) + O2(g)}
	\item The formation of water given in Section 6.8 should produce only 1 mole of water as a product
	\item Slater's Rules in section 9.2 should treat deep core and V-1 electrons differently
	\item Lattice energy is often referenced as positive values, instead of the negative values used here. This convention makes explaining the Born-Haber cycle a little easier because you only need to reverse one value $\left(\Delta H^\circ_f\right)$
\end{itemize}
\end{document}
