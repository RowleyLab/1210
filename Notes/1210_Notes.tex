\documentclass[12pt, openany, letterpaper]{memoir}
\usepackage{NotesStyle}
%\renewcommand\thesection{\thechapter\Alph{section}}
%\renewcommand\thesubsection{\thesection.\Numeral{subsection}}

\begin{document}
\title{CHEM 1210 Lecture Notes\\ OpenStax Chemistry 2e}
\author{Matthew Rowley}
\mainmatter
\maketitle
\chapter*{Course Administrative Details}
\begin{itemize}
	\item My office hours
	\item Intro to my research
	\item Introductory Quiz
	\item Grading details
	      \begin{itemize}
		      \item Exams - 40, Final - 20, Online Homework - 15, Book Homework - 15, Quizzes - 10
		      \item Online homework
		      \item Frequent quizzes
	      \end{itemize}
	\item Importance of reading and learning on your own
	\item Learning resources
	      \begin{itemize}
		      \item My Office Hours
		      \item Tutoring services - \href{https://www.suu.edu/academicsuccess/tutoring/}{https://www.suu.edu/academicsuccess/tutoring/}
	      \end{itemize}
	\item Show how to access Canvas
	      \begin{itemize}
		      \item Calendar, Grades, Modules, etc.
		      \item Quizzes
		      \item Textbook
	      \end{itemize}
	\item Introduction to chemistry
	      \begin{itemize}
		      \item Ruby fluorescence
		      \item Levomethamphetamine
          \item Submerged salt crystals grow due dynamic equilibrium
		      \item Rubber band elasticity
		      \item Structure of the periodic table
		      \item Salt on ice and purifying hydrogen peroxide
	      \end{itemize}
\end{itemize}

\chapter{Essential Ideas}

\section{Chemistry in Context}
\begin{itemize}
  \item Modern chemistry is the end result of thousands of years of humans trying to explain and control the materials around them
  \item Early forays into chemistry (such as alchemy) had deep mystical roots and often relied on serendipity to make good progress
  \item Modern chemistry is a rigorous science, relying on falsifiability and the scientific methd (Figure 1.4)
  \item We sometimes refer to chemistry as ``The Central Science'' (Figure 1.3)
  \item To adequately describe and understand chemical phenomena, we often talk from different perspectives
  \begin{description}
    \item[Macroscopic Domain] This is what we observe with bulk substances. Two chemicals react to produce a new chemical
    \item[Microscopic Domain] We now understand that all microscopic effects are governed by the behavior of \emph{microscopic} actors (molecules, atoms, electrons, etc.)
    \item[Symbolic Domain] Effectively communicating chemical ideas requires new language. Chemical formulas, equations, and mechanisms are all symbolic representations
    \item All three domains are on display in Figure 1.5
  \end{description}
\end{itemize}
\paragraph*{Quiz 1.1 - Scientific Method}
\paragraph*{Homework 1.1}
\begin{itemize}
  \item 1: Thinking in terms of Chemistry
  \item 3: The scientific method
  \item 5: Domains of inquiry
\end{itemize}

\section{Phases and Classification of Matter}
\begin{itemize}
  \item Three primary phases of matter are shown in Figure 1.5 (and 1.6)
  \item Plasmas are like a gas, but with electrically charged particles
  \item Mass vs Weight (for very fine measurements, the difference matters even on Earth due to buoyancy)
  \item Figure 1.8 illustrates the \emph{law of conservation of matter}
  \item Classifying matter (Figure 1.11)
  \begin{itemize}
    \item Pure Substances
    \begin{itemize}
      \item Elements (Anything on the \emph{periodic table of the elements})
      \item Compounds (Combinations of elements -- can have very different properties from their constituent elements)
    \end{itemize}
    \item Mixtures
    \begin{itemize}
      \item Heterogeneous mixtures (variable composition)
      \item Homogeneous mixtures (i.e. solutions, continuous composition)
    \end{itemize}
  \end{itemize}
  \item Table 1.1 shows the abundance of many elements on Earth
  \item Atoms are the smallest particle of an element that has the properties of that element
  \begin{itemize}
    \item Thought-experiment of dividing a sample in half ad-infinitum
    \item Ancient atomic theories and modern Dalton atomic theory (discussed in detail later)
    \item Atoms are \emph{very} small; smaller than we could even detect until recently
  \end{itemize}
  \item Molecules are collections of atoms held together with chemical bonds (more nuanced definition later)
    \begin{itemize}
      \item Many elements occur naturally as molecules, rather than atoms
      \item Figure 1.14 shows many molecular elements and compounds
    \end{itemize}
\end{itemize}

\section{Physical and Chemical Properties}
\begin{itemize}
  \item Physical Properties: Properties which can be observed without changing the chemical identity of the substance
  \item Chemical Properties: Properties which can only be observed through chemical reactions (e.g. flammability, acidity, electrochemical potential, etc.)
  \item Physical Changes: Any change which perserves the \emph{chemical identity} of the substance (including phase changes)
  \item Chemical Changes: Changes which alter the chemical identities of one of more substance
  \item Extensive Properties: Depend on the size of hte system (double the size, double the property measurement, such as mass or volume)
  \item Intensive Properties: Independent of system size (density, temperature, most chemical properties)
  \item The periodic table groups elements according to their properties (Figure 1.22)
  \begin{itemize}
    \item Metals conduct electricity and heat, are maleable and ductile
    \item Non-metals are very diverse, but generally poor conductors
    \item Metalloids exist at the boundary and share properties with both metals and non-metals
    \item There are many other ways to group the elements, which we will learn later
  \end{itemize}
\end{itemize}
\paragraph*{Quiz 1.2 - Matter, Properties, and Change}
\paragraph*{Homework 1.2}
\begin{itemize}
  \item 17: Classifying matter
  \item 27: Classifying changes
\end{itemize}

\section{Measurements}
\begin{itemize}
  \item All measurements are composed of three parts:
  \begin{itemize}
    \item The magnitude of the measurement (the number itself)
    \item The unit of measurement used (g, kg, lbs, etc.)
    \item The degree of uncertainty in the measurement (this is usually implicit, and covered in the next section)
  \end{itemize}
  \item Units are an essential part of any measuement. Develop a habit of \emph{always} including units in your work
  \begin{itemize}
    \item $u_{rms}=\sqrt{\dfrac{3RT}{M}}$ -- example of how units can guide problem solving and ``unit purgatory''
    \item SI units are a collection of fundamental units from which all other units can be derived (Table 1.2)
    \item Metric prefixes make it more convenient to discuss very large or very small numbers (Table 1.3)
    \item Scientific notation is an even more general and robust way of representing numbers
    \begin{itemize}
      \item The quantity is represented by a number with the decimal after the first digit
      \item The magnitude is represented by a power of $10$
    \end{itemize}
    \item Practice converting between normal numbers, metric prefixes, and scientific notation
    \item For temperature, we use both $K$ and $^\circ C$ (But not $^\circ F$)
      \\$T(K) = T(^\circ C) + 273.15$ 
    \item Derived units will combine the fundamental units in some way 
      \\volume: $m^3$, $L$, $ml$
      \\velocity: $\nicefrac{m}{s}$ 
      \\density: $\nicefrac{kg}{m^3}$, $\nicefrac{g}{cm^3}$ (Table 1.4)
      \\energy: $1J\equiv \nicefrac{kgm^2}{s^2}$
  \end{itemize} 
\end{itemize}

\section{Measurement Uncertainty, Accuracy, and Precision}
\begin{itemize}
  \item Countable quantities are considered to be \emph{exact} (no uncertainty)
  \item Measurements (and groups of measurements) always have some degree of undertainty
  \begin{itemize}
    \item Accuracy is how close a measurement is to the \emph{true value} (usually unknown, but approximated by calibration with a well-known standard)
    \item Precision is how finely a measurment is made (What is the margin of error)
    \item Figure 1.27 and Table 1.5 illustrate the differences between precision and accuracy
    \item Accuracy is usually improved through calibration, and moving forward we will usually assume that measurements are as accurate as an instrument allows
    \item Precision is represented in the way we write the number, and can be improved with a better instrument or with repeat measurements
  \end{itemize}
  \item Significant figures are the way that we represent precision in a number
  \begin{itemize}
    \item The number of digits conveys the degree of precision
    \item Example of me saying I'm $6ft~2in$ tall, vs me saying I'm $6ft~1.6241434in$ tall
    \item For graduated measurements, we record one digit beyond the lowest graduation (Figure 1.26)
    \item For digital measurements, we record the number as it is given by the instrument
    \item For any given number, we should track both the \emph{quantity} of significant figures, and the \emph{position} of the least-significant digit
    \item In a written number, digits are considered significant according to the following rules:
    \begin{itemize}
      \item All non-zeros are significant
      \item All \emph{captive} zeros (between two other significant digits) are significant
      \item Trailing zeros are \emph{always} significant
      \item Leading zeros are \emph{never} significant
      \item For scientific notation, only the digits of the quantity (not the magnitude) count
      \item Logarithmic quantities follow different rules which we will revisit in CHEM 1220 (chapter 14)
      \item Note that for some numbers scientific notation is \emph{required} to convey the correct precision ($3.0\times10^3m$)
    \end{itemize}
  \end{itemize}
  \item Errors propogate when multiple measurements are used in a mathematical operation
  \begin{itemize}
    \item For addition and subtraction, the least significant digit of the answer will be in the same position as the least significant digit of hte least precise input
    \item For multiplicationa and division, the quantity of significant digits in the answer will match the quantity of significant digits of the input with fewest significant digits
    \item When rounding an exact $5$ (no further digits beyond the $5$), round up or down to make the last digit even
    \item Compound problems involve multiple types of operations
    \begin{itemize}
      \item Solve the problem in steps, applying the correct rule to each step
      \item Track the significant figures (quantity and position) for each intermediate answer, but do \emph{not} truncate or round any of these answers
      \item Only round after the last step
        \\ $\circ$ Practice $\frac{12.3g+34g}{12.0cm^3+7.7cm^3}=2.4\nicefrac{g}{cm^3}$ (wrong answer with premature rounding)
    \end{itemize}
  \end{itemize}
\end{itemize}
\paragraph*{Quiz 1.3 - Significant Figures}
\paragraph*{Homework 1.3}
\begin{itemize}
  \item 45: Scientific Notation
  \item 49: Counting Significant Figures
  \item 53: Significnat Figures and Calculations
\end{itemize}

\section{Mathematical Treatment of Measurement Results}
\begin{itemize}
  \item Some quantities are calculated based on two or more measurements (such as velocity and density)
  \item These formulas can be used to relate all three quantities together (i.e. $velocity = \frac{distance}{time}$)
  \item The derived quantity can be interpreted as a \emph{comversion factor} between the other two quantitites
  \item Conversion factors and unit conversions 
  \begin{itemize}
    \item Elementary school perspective of $ft$ to $in$ conversions
    \item Conversion factors are a ratio between two identical quantities
    \item Converting units involves multiplying by $1$ in the form of a conversion factor
    \item Units guide the problem solving
  \end{itemize}
  \item Dimensional Analysis is a problem-solving framework based on a series of unit conversions
  \begin{itemize}
    \item Don't dive straight into calculations and equations
    \item Identify the units you expect for the answer
    \item Identify the starting point
    \item Create a plan to convert units from the starting point to the answer
    \item Carry out the calculations
    \item Practice converting $65.0\nicefrac{miles}{hour}$ into $\nicefrac{m}{s}$
    \item The ``railroad ties'' or ``picket fence'' method can help organize your work
  \end{itemize}
  \item Dimensional analysis is not the only way to solve problems, but it is versatile and robust; usually my preferred choice
  \item Practice a more abstract problem:
    \\ Find the $\nicefrac{miles}{gal}$ if a car consumes $8036~g$ of gasoline while driving for $40.0~min$ at $75~\nicefrac{miles}{hour}$
\end{itemize}
\paragraph*{Quiz 1.4 - Dimensional Analysis}
\paragraph*{Homework 1.4}
\begin{itemize}
  \item 65: Simple unit conversion
  \item 87: Density from volume and mass
  \item 89: Mass from volume
  \item 91: Volume from mass
\end{itemize}

\chapter{Atoms, Molecules, and Ions}

\section{Early Ideas in Atomic Theory}
\begin{itemize}
  \item 1807 Dalton's Atomic Theory: (1, 2 and 5 are not strictly true) (Figures 2.2-2.4)
  \begin{enumerate}
    \item Matter is composed of atoms
    \item Atoms of a given element all have identical properties to each other
    \item Atoms of one element differ in properties from elements of a different element
    \item Chemical compounds consist of atoms of different elements combined in a specific ratio
    \item Chemical reactions \emph{rearrange} the atoms which are already there, but cannot create or destroy atoms
  \end{enumerate}
  \item Development of Dalton's theory:
  \begin{itemize}
    \item Dalton relied on prior work by Proust who demonstrated the law of definite proportions (Table 2.1)
    \item This was not at all expected - my analogy with bread, or metal alloys
    \item Dalton further noted that ratios of these proportions followed the law of multiple proportions (Copper (I or II) Chloride example in the book)
  \end{itemize}
\end{itemize}

\section{Evolution of Atomic Theory}
\begin{itemize}
  \item About a century later, scientists discovered that atoms are made of even smaller components
  \item J. J. Thomson discovered the electron, and its charge/mass ratio (Figure 2.6)
  \item Millikan's oil drop experiment found the fundamental charge (and thus mass) of an electron (Figure 2.7)
  \item Figure 2.8 shows some early ideas of how the positive and negative charges were distributed in an atom
  \item Ernest Rutherford discovered the atomic nucleus, consisting of very concetrated positive charge (Figures 2.9 and 2.10)

    ``It was quite the most incredible event that has ever happened to me in my life. It was almost as incredible as if you fired a 15-inch shell at a piece of tissue paper and it came back and hit you.''
  \item Different \emph{isotopes} of atoms were discovered with techniqes that produced isotopically pure samples
  \item Finally, the neutron itself was discovered in 1932, explaining what particle led to different isotopes
\end{itemize}

\section{Atomic Structure and Symbolism}
\begin{itemize}
  \item Atoms are made up of protons, neutrons, and electrons
  \item Figure 2.11 shows the small scale of the atom and nucleus
  \item Atomic units make discussions about atoms convenient
  \begin{itemize}
    \item The Atomic Mass Unit $amu$, $Da$, or $u = 1.6605\times10^{-24}g$
    \item The fundamental charge $e = 1.602\times10^{-19}C$
    \item The Angstrom \AA$=10^{-10}m$
  \end{itemize}
  \item Table 2.2 summarizes the properties of elementary particles in atoms
  \item We track the composition of an atom with three numbers:
  \begin{itemize}
    \item The atomic number is the number of protons $Z=p$ 
    \item The mass number is the number of protons and neutrons $A=p+n$ 
    \item The number of neutrons is therefore $n=A-Z$
    \item The charge is the protons minus the electrons $q=p-e$
    \item The number of electrons is $e=p-q$
  \end{itemize}
  \item A positively charge atom is called a cation, and a negatively charged atom is called an anion
  \item Chemical symbols are a shorthand way of representing everything we need about an atom
  \begin{itemize}
    \item There is a 1 or 2 letter symbol for each element (Table 2.3 shows some make sense, some don't)
    \item $A$ is written as a left superscript
    \item $Z$ is written as a left subscript, but can be left off
    \item $q$ is written as a right superscript with the magnitude first, then the sign. $q$ is left off if $q=0$
    \item Example: $^{13}_6C^{2+}$ (6 protons, 7 neutrons, 4 electrons)
  \end{itemize}
  \item Isotopes are different versions of elements with different mass numbers
  \begin{itemize}
    \item For the most part, different isotopes of an element behave exactly the same in chemisry
    \item Isotope abundance can be found by mass spectrometry, among other methods (Figure 2.15)
    \item Table 2.4 shows the natural abundances of the isotopes of several light elements
    \item Atomic weight (atomic mass) is the weighted average of all the isotopes found on the Earth

      $M = \sum\limits_{i}mass_i\times\dfrac{\% abundance_i}{100\%}$
  \end{itemize}
\end{itemize}

\section{Chemical Formulas}
\begin{itemize}
  \item We can represent the actual structure and makeup of molecules at several levels of abstraction (Figures 2.16 and 2.17)
  \item Molecular formulas
  \begin{itemize}
    \item Each element is listed, with the number of atoms for each element written as a subscript (\ch{H2O})
    \item The order of elements follows certain patterns, with the least electronegative element often written first
  \end{itemize}
  \item Structural Formulas show how atoms are connected with covalent bonds represented as lines
  \item Ball and Stick models show the three-dimensional geometry of a molecule
  \item Space-filling models show the actual volume of space taken up by each atom in a molecule
  \item Figure 2.18 illustrates the difference between subscripts within a formula, and stoichiometric coefficients in front of formulas
  \item Empirical formulas show the mathematically simplified ratios of elements
  \begin{itemize}
    \item Some experimental techniques (especially early ones) could \emph{only} give the empirical formula
    \item Ionic compounds are always reported with the empirical formula
    \item To find the empirical formula, divide all subscripts by their greatest common factor
    \item Some very different compounds share an empirical formula (carbohydrates \ch{CH2O} include formaldehyde, acetic acid, and sugar)
    \item The molecular formula can be calculated from the empirical formula's weight and the molecular weight (Chapter 3)
  \end{itemize}
  \item Arranging the same group of atoms in different ways produces different isomers
    \begin{itemize}
      \item Isomers share the same chemical formula, but can have very different properties
      \item Structural isomers differ in how the atoms are connected to each other (Figure 2.23)
      \item Optical isomers (or spatial isomers) are non-superimposable mirror images (Figure 2.24, glove analogy)
    \end{itemize}
\end{itemize}

\paragraph*{Quiz 2.1 - Atomic Theories}
\paragraph*{Homework 2.1}
\begin{itemize}
  \item 7: Properties of protons and neutrons
  \item 11: Atomic symbols from composition
  \item 19: Composition from atomic symbols
  \item 23: Atomic weight
  \item 29: Molecular and empirical formulas
\end{itemize}

\section{The Periodic Table}
\begin{itemize}
  \item As scientists discovered and studied more and more elements, they started to notice certain natural groupings according to physical and chemical properties
  \item Mendeleev arranged the atoms according to these groups and atomic weight, producing the first periodic table (Figure 2.25)
  \item Mendeleev even predicted the existence of and properties of yet-undiscovered elements based on gaps in his table
  \item Vocabulary around the periodic table:
  \begin{itemize}
    \item Rows are periods or series
    \item Columns are groups or families
  \end{itemize}
  \item Figure 2.26 is a typical periodic table, showing the metals, non-metals and metalloids (contrast with my preferred table)
  \item Figure 2.27 shows many of the names we use for important groups of elements
  \item The structure of the periodic table encodes rich information about the electrons in the elements, as we will learn in chapter 6
\end{itemize}

\paragraph*{Quiz 2.2 - Periodic Table}
\paragraph*{Homework 2.2}
\begin{itemize}
  \item 37: Classifying elements
  \item 41: Using group names
\end{itemize}

\section{Ionic and Molecular Compounds}
\begin{itemize}
  \item An atom which gains or loses electrons (carries a charge) is called an \emph{ion}
  \begin{itemize}
    \item Positively charged ions are called cations, and are smaller than their neutral atom counterparts (Figure 2.28)
    \item Negatively charged ions are called anions, and are larger than their neutral atom counterparts
    \item We can predict which charge different elements will naturally tend to take based on their position in the periodic table (Figure 2.29)
    \item Many other elements can take two or more charges, especially the transition metals
  \end{itemize}
  \item Some ions are composed of more than one atom and are called polyatomic ions
  \begin{itemize}
    \item Table 2.5 gives some common polyatomic ions. Memorize the formula, name, and charge of these and their acids
    \item Notice some trends in the names of oxyanions (per-ate, -ate, -ite, and hypo-ite)
    \item We will learn about the acid names listed here (and more) in section 2.7
  \end{itemize}
  \item Ionic compounds are held together by ionic bonds (coulombic attractions)
  \begin{itemize}
    \item Show dot diagram of how \ch{NaCl} and \ch{CaCl} form from elements
    \item Metal + non-metal is not an adequate definition of ionic compounds (\ch{NH4NO3})
    \item Ionic compounds form an extended lattice of ions (covered more in CHEM 1220)
    \item Ions will combine to form neutral compounds
    \item Practice producing compound formulas from ions (include paranthesis for polyatomic ions where appropriate)
    \item Practice finding ions from formulas of compounds
  \end{itemize}
  \item Molecular compounds are held together by covalent bonds (shared electrons)
  \begin{itemize}
    \item Show a dot diagram of \ch{H2}, \ch{H2O}, \ch{O2}, and \ch{N2}
    \item Bonds between non-metals are covalent bonds
    \item Molecular compounds combine non-metals into discrete units called molecules
    \item Single, double, and triple bonds involve sharing two, four, and six electrons
  \end{itemize}
\end{itemize}

\section{Chemical Nomenclature}
\begin{itemize}
  \item Naming ionic compounds
  \begin{itemize}
    \item Cation names are the name of the element, with the charge in roman numerals in parenthesis \emph{if} the element could take multiple charges
    \item Anion names are the element name with an ``-ide'' ending (some element like \ch{P} remove more than others)
    \item Polyatomic ion names are the same as you learned earlier
    \item For ionic compounds the name is simply cation name + anion name
    \item There is no indication of the quantity of atoms, that is inferred from charge neutrality
    \item Practice getting formulas from names and names from formulas
  \end{itemize}
  \item Naming hydrates
  \begin{itemize}
    \item Some ionic compounds will incorporate water into their ionic lattice
    \item The formulas will have a $\cdot$ then list the number of waters
    \item The names add the degree of hydration using prefixes from table 2.10 and adding ``hydrate''
    \item The waters can be driven of with high temperature, producing the \emph{anhydrous} form
  \end{itemize}
\end{itemize}

\paragraph*{Quiz 2.3 - Naming Ionic Compounds}
\paragraph*{Homework 2.3}
\begin{itemize}
  \item 47: Predicting bond type in compounds
  \item 49: Formulas from ions
  \item 51: Names from formulas
  \item 57: Names from formulas with transition metals
  \item 59: Formulas from names
\end{itemize}

\paragraph*{Resuming section 2.7 Chemical Nomenclature}
\begin{itemize}
  \item Naming molecular compounds
  \begin{itemize}
    \item There are many ways to name molecular compounds, we will focus on just two here
    \item Naming binary molecular compounds
    \begin{itemize}
      \item \# + name + \# + name with ``-ide'' ending
      \item Least electronegative element (leftmost on the periodic table) goes first
      \item \#s come from table 2.10
      \item Omit ``mono-'' for the first element
      \item Practice going from formula to name and vice-versa (Table 2.11)
    \end{itemize}
    \item Naming molecular acids
    \begin{itemize}
      \item Molecular acid names are based on the name of the anion formed when all \ch{H+} are removed
      \item -ide ions form hydro-ic acids (Table 2.12)
      \item -ate ions form -ic acids
      \item -ite ions form -ous acids
      \item preserve the hypo- and per- prefixes
      \item Table 2.13 shows some oxyacid names
    \end{itemize}
  \end{itemize}
\end{itemize}

\paragraph*{Quiz 2.4 - Naming Molecular Compounds}
\paragraph*{No textbook homework to accompany this quiz due to a lack of appropriate questions!}

\chapter{Composition of Substances and Solutions}

\section{Formula Mass and the Mole Concept}
\begin{itemize}
  \item For chemists, the amount of a substance we care about is not grams, but the quantity of atoms
  \item Analogy of hammers and feathers
  \item The mole is defined as the relationship between $AMUs$ and $g$

    $6.022\times10^{23}AMU=1g \longrightarrow 1molAMU=1g \longrightarrow 1AMU=1\dfrac{g}{mol}$
  \item Any time we would use the $AMU$ we could replace it with $\nicefrac{g}{mol}$, and this is called the \emph{molar mass}
  \item Practice converting between grams and moles using atomic weight
  \item For compounds, we can find the molar mass, or formula mass by adding up the atomic weights of all the atoms in the formula
  \item We can count the literal number of atoms, but it is usually best to remain in units of moles
  \item Practice finding formula masses for compounds, and converting between grams and moles for compounds
\end{itemize}

\section{Determining Empirical and Molecular Formulas}
\begin{itemize}
  \item Empirical analysis is a method to determine empirical formulas
  \begin{itemize}
    \item In our text, the masses of C, H, and O are given directly
    \item To get the formula, turn each mass into moles, and divide by the smallest mole number (Figure 3.11)
    \item Actually, this technique involves a careful combustion reaction
    \item The mass of C comes from the amount of \ch{CO2} produced
    \item The mass of H comes from the amount of \ch{H2O} produced (remember the 2!)
    \item The mass of O comes from subtracting those from the original amount
    \item Turn the masses to moles
    \item Practice: $2.443g$ of unknown produce $5.368g$ of \ch{CO2} and $2.930g$ of \ch{H2O} ($M=60.096$, propanol)
    \item Practice: $3.450g$ of unknown produce $5.057g$ of \ch{CO2} and $2.070g$ of \ch{H2O} ($M=180.16$, hexose)
  \end{itemize}
  \item \% Composition gives the \% by mass of each element in a compound
  \begin{itemize}
    \item From masses, we can calculate the \% mass composition easily
    \item If \% mass is given, we can just assume $100.0g$ and turn the \%s into $g$ to do emipirical analysis
    \item From the empirical or molecular formula, we can get \% mass composition: $\%_X=\dfrac{nM_X}{M_{Formula}}$
  \end{itemize}
  \item We can determine the molecular formula from the empirical formula if we know the molar mass
  \begin{itemize}
    \item The molecular formula will be an integer multiple of the empirical formula (could be 1)
    \item That formula is the ratio $n=\dfrac{M_{molar}}{M_{empirical}}$
    \item Find the molecular formulas of the two examples above using their molar masses
  \end{itemize}
\end{itemize}

\paragraph*{Quiz 3.1 - Molar Mass}
\paragraph*{Homework 3.1}
\begin{itemize}
  \item 3: Calculating molar mass
  \item 17: Mass from moles
  \item 18: Moles from mass (sorry, no solution!)
\end{itemize}

\paragraph*{Quiz 3.2 - Empirical Analysis}
\paragraph*{Homework 3.2}
\begin{itemize}
  \item 33: \% composition from formulas
  \item 37: Empirical formulas from \% composition
  \item 39: Molecular formula from \% composition and molar mass
\end{itemize}

\section{Molarity}
\begin{itemize}
  \item Molarity is the most common way of describing the \emph{concentration} of a solute in a solution
  \begin{itemize}
    \item The unit for molarity is $M$, which is unfortunate because that is also used for molar masses
    \item Molarity is the moles of solute divided by the liters of solution $M=\dfrac{mol_{solute}}{L_{solution}}$
    \item Technically, the solution volume could be more or less than the solvent volume before dissolving. When we make a solution, we always describe the process as ``add enough water to meet the mark''
    \item The molarity can now be used as a conversion between moles and volume
    \item Practice: Find molar concentration of solution made by dissolving $13.5g$ \ch{NaCl} in $100.0ml$ of solution
    \item Practice: How many moles of \ch{NaCl} are contained in $35.62ml$ of the solution?
    \item Practice: How many $ml$ would be needed to provide $1.74\times10^{-3}mol$ of \ch{NaCl}?
  \end{itemize}
  \item Solutions can be diluted or concentrated
  \begin{itemize}
    \item The number of moles of solute is the same, just the volume of solution changes (by adding or removing solvent)
    \item $C_1V_1=C_2V_2$
    \item Practice: Find the molar concentration when $36.43ml$ of $1.5M$ \ch{HCl} are diluted to $100.0ml$?
    \item Practice: How many $ml$ of $0.125M$ \ch{NaNO3} should be used to make $50.0ml$ of $2.4\times10^{-3}M$?
  \end{itemize}
\end{itemize}

\section{Other Units for Solution Concentration}
\begin{itemize}
  \item There are other, less common units of concentration (including some not covered until chapter 11)
  \item Mass \%, or \% by mass
  \begin{itemize}
    \item $\%m/m=\dfrac{m_A}{m_{total}}100\%$
  \end{itemize}
  \item Volume \%, or \% by volume
  \begin{itemize}
    \item $\%v/v=\dfrac{V_A}{V_{solution}}100\%$
  \end{itemize}
  \item m/V \%
  \begin{itemize}
    \item $\%m/v=\dfrac{g_A}{V_{solution}}100\%$
  \end{itemize}
  \item ppm and ppb
    \begin{itemize}
      \item $ppm=\dfrac{m_A}{m_{total}}10^6ppm$
      \item $ppb=\dfrac{m_A}{m_{total}}10^9ppb$
    \end{itemize}
\end{itemize}

\paragraph*{Quiz 3.3 - Concentration}
\paragraph*{Homework 3.3}
\begin{itemize}
  \item 49: Mass from molarity and volume
  \item 53: Calculating molarity
  \item 61: Dilution
  \item 73: ppm
  \item 77: \%m/m
\end{itemize}

\chapter{Stoichiometry of Chemical Reactions}

\section{Writing and Balancing Chemical Equations}
\begin{itemize}
  \item A chemical equation relates the amounts of reactants and products in a chemical reaction
  \item Consider the reaction: \ch{CH4 + 2 O2 -> CO2 + 2 H2O} (Figure 4.2)
  \item Reactants on the left, and products are on the right
  \item The large numbers are called \emph{stoichiometric coefficients}
  \item The subscript numbers are part of the chemical identity of the reactants and products
  \item The stoichiometric coefficients tell the right ratio to combine reactants, and the ratio of products produced (Figure 4.3)
  \item Balancing chemical reactions:
    \begin{itemize}
      \item To balance a chemical equation, we will add the proper coefficients to give the same numbers and types of atoms on both sides of the reaction arrow
      \item Start with any repeated atom moieties, such as polyatomic ions, and balance them as a unit rather than counting individual atoms
      \item Next balance elements that appear in only one compound on each side
      \item Finally, balance any remaining elements (\ch{O} is often best left to balance last)
      \item If necessary, multiply all coefficients by an integer
    \end{itemize}
  Practice: Balance the following chemical equation: \ch{Fe2O3(s) + Al(s) -> Al2O3(s) + Fe(s)}

  ~\hphantom{Practice:} \ch{Fe2O3(s) + 2 Al(s) -> Al2O3(s) + 2 Fe(s)}

  Practice: Balance the following chemical equation: \ch{C8H18(l) + O2(g) -> CO2(g) + H2O(g)}

  ~\hphantom{Practice:} \ch{2 C8H18(l) + 25 O2(g) -> 16 CO2(g) + 18 H2O(g)}

  Practice: Balance the following chemical equation: \ch{Ag2SO4(aq) + NaCl(aq) -> AgCl(s) + Na2SO4(aq)}

  ~\hphantom{Practice:} \ch{Ag2SO4(aq) + 2 NaCl(aq) -> 2 AgCl(s) + Na2SO4(aq)}
  \item Chemical reactions will sometimes include additional information
    \begin{itemize}
      \item Phases (s), (l), (g), (aq)
      \item Reaction conditions, such as temperature, or the presence of a catalyst (often written over the arrow)
      \item Enthalpy of reaction or equilibrium constant (covered in later chapters)
    \end{itemize}
  \item Reactions with ionic compounds can often be written differently and simplified
    \begin{itemize}
      \item The normal equation is called a \emph{molecular equation} (despite involving ionic compounds)

        \ch{CaCl2(aq) + 2 AgNO3(aq) -> Ca(NO3)2(aq) + 2 AgCl(s)}
      \item The \emph{complete ionic equation} will break \emph{soluble} ionic compounds into the separate cations and anions
      \item This is a more accurate representation of soluble ionic compounds, which are not actually \emph{together} in any chemically meaningful way

        \ch{Ca^{2+}(aq) + 2 Cl^-(aq) + 2 Ag^+(aq) + 2 NO3^-(aq) -> Ca^{2+}(aq) + 2 NO3^-(aq) + 2 AgCl(s)}
      \item The \emph{net ionic equation} removes \emph{spectator ions}, which didn't actually participate in any chemical change
      \item My silly analogy about party-goers and the social interactions in a party environment

        \ch{Ag^+(aq) + Cl^-(aq) -> AgCl(s)}
    \end{itemize}
\end{itemize}

\paragraph*{Quiz 4.1 - Balance Chemical Equations}
\paragraph*{Homework 4.1}
\begin{itemize}
  \item 5: Balancing equations 
  \item 11: Net ionic equations
\end{itemize}

\section{Classifying Chemical Reactions}
\begin{itemize}
  \item We can categorize chemical reactions in several ways, but for this class we will focus on the types of chemical changes caused by a reaction
  \item Precipitation reactions form one or more solid ionic product from all aqueous ionic reactants
    \begin{itemize}
      \item First, predict the products by switching cations and anions
      \item Next, predict the phase by using solubility rules (Table 4.1)
      \item Identify and eliminate the spectator ions to get a net ionic equation
      \item Some combinations of reactants will have no reaction, and some may have two solid products
      \item More about precipitation reaction in CHEM 1220
    \end{itemize}
\end{itemize}

\paragraph*{Quiz 4.2 - Precipitation Reactions}
\paragraph*{Homework 4.2}
\begin{itemize}
  \item 28: Predicting products of reactions with ionic compounds
\end{itemize}

\paragraph*{Resuming Section 4.2: Classifying Chemical Reactions}
\begin{itemize}
  \item Acid-Base reactions involve the transfer of a \ch{H^+} between reactants
    \begin{itemize}
      \item Acids will react with water to produce hydronium ion
      \item Bases will react with water to produce hydroxide ions
      \item \emph{Strong} acids/bases react with water to completion, while \emph{weak} acids/bases only react partially (equilibrium)
      \item Table 4.2 lists six common strong acids
      \item Alternative definition: Acids are proton donors and bases are proton acceptors
      \item A neutralization reaction produces water and a salt (ionic compound)
      \item More about acid/base reactions in CHEM 1220
    \end{itemize}
  \item Redox reactions involve the transfer of one or more electrons
    \begin{itemize}
      \item Consdier the reaction: \ch{2 Na(s) + Cl2(g) -> 2 NaCl(s)}
      \item OIL RIG (or LEO says GER) to remember which is oxidation, and which is reduction
      \item Oxidizing agents and reducint agents. ``Agent'' describes the effect on the reaction partner
      \item Sometimes the electron transfer is not as obvious: \ch{CH4 + 2 O2 -> CO2 + 2 H2O}
      \item We assign oxidation numbers to keep track of electrons
        \begin{itemize}
          \item Elements have ox. \# of $0$
          \item Monoatomic ions have ox. \# equal to their charge
          \item Within compounds, O has ox. \# $=-2$, and H has ox \# $=+1$
          \item Oxidation \#s add up to the total overall charge (this rule subsumes the first 2)
        \end{itemize}
      \item Balancing redox reactions needs additional steps to account for the transferred electrons
        \begin{enumerate}
          \item Split the reaction into half-reactions (need to find oxidation \#s to do this)
          \item Balance all elements except H and O
          \item Add the electrons, based on the changes to oxidation \#s
          \item Balance charge by adding \ch{H^+} in acid, or \ch{OH^-} in base
          \item Balance H and O together by adding water
          \item Multiply half-reactions to balance their electrons, then add them together
          \item Make any cancellations of water, \ch{H+}, or \ch{OH^-}
        \end{enumerate}
      \item You can balance redox reactions as a whole instead of as half-reactions if you prefer
      \item Electrochemical series and predicting if redox reactions will be spontaneous or not (Not in the book!)
      \item More about Redox reaction in \ldots CHEM 1220!
    \end{itemize}
\end{itemize}

\paragraph*{Quiz 4.3 - Redox Reactions}
\paragraph*{Homework 4.3}
\begin{itemize}
  \item 17: Assigning oxidation states
  \item 19: Classify acid/base and redox reactions
  \item 39: Balancing half-reactions
\end{itemize}

\section{Reaction Stoichiometry}
\begin{itemize}
  \item Measured amounts should always be converted to moles in order to make comparisons to other chemical species in an equation
    \begin{itemize}
      \item We often want to calculate the corresponding amount of different chemicals in a chemical reaction
      \item Making direct comparisons will not work due to different molar masses, etc.
      \item Figure 4.11 shows how to convert to moles from different measurements
      \item My version of this figure (``Chemistry Land'')
    \end{itemize}
  \item Practice: \ch{N2(g) + 3 H2(g) -> 2 NH3(g)} Find other amounts for $0.75g$ of \ch{H2}
\end{itemize}

\paragraph*{Quiz 4.4 - Stoichiometry}
\paragraph*{Homework 4.4}
\begin{itemize}
  \item 49: Stoichiometry with solid reactants
  \item 57: Stoichiometry with aqueous reactions
\end{itemize}

\section{Reaction Yields}
\begin{itemize}
  \item Often one or more reactants are provided in excess, with only one \emph{limiting reactant}
    \begin{itemize}
      \item Pick a product, and calculate how much product would be produced for each reactant
      \item The lowest amount is the theoretical yield. The other amounts can be discarded
      \item The reactant which lead to the lowest amount is the limiting reactant
      \item Calculate the rest of the corresponding amounts starting from the limiting reactant
      \item Double-check your work using the conservation of mass
    \end{itemize}
  \item Practice: \ch{2 C4H10 + 13 O2 -> 8 CO2 + 10 H2O} with $5.00g\ch{C4H10}$ and $9.00g\ch{O2}$
  \item Reactions rarely go perfectly, and the actual amount of product will be different from the theoretical yield
  \item $\%yield = \dfrac{actual~yield}{theoretical~yield}\times 100\%$
  \item Yields below $100\%$ can mean that the reaction didn't go to completion, or some product was lost during a purification stage
  \item Yields above $100\%$ usually indicate that some contaminants (including perhaps excess reactant) are mixed in with the product
\end{itemize}

\paragraph*{Quiz 4.5 - Limiting Reactants}
\paragraph*{Homework 4.5}
\begin{itemize}
  \item 61: Limiting Reactant Problem
  \item 63: Percent Yield
\end{itemize}

\section{Quantitative Chemical Analysis}
\begin{itemize}
  \item Quantitative analysis is the family of techniques which determine the amount of substance in a sample
  \item Titrations find the concentration of an aqueous \emph{analyte}
    \begin{itemize}
      \item React the unknown with a co-reactant of known concentration by adding titrant with a buret
      \item The chemical environment will change ($pH$, redox potential, etc.) once the reaction reaches the equivalence point
      \item Probes or color indicators can show when to stop the titration. This is called the end-point, and is ideally very close to the equivalence point
      \item $\dfrac{C_TV_T}{\nu_T}=\dfrac{C_AV_A}{\nu_A}$
    \end{itemize}
  \item Gravimetric analysis uses a chemical reaction to change the phase of the analyte so it can be separated and weighed
    \begin{itemize}
      \item Your \% copper lab found the amount of copper by reducing it to a solid
      \item Your hydrates lab found the amount of water by weighing it before and after the baking step
    \end{itemize}
  \item Combustion analysis - We actually covered this earlier with \% composition of compounds
\end{itemize}

\paragraph*{Quiz 4.6 - Titrations}
\paragraph*{Homework 4.6}
\begin{itemize}
  \item 79: Acid/base Titration
  \item 81: Precipitation Titration
\end{itemize}

\chapter{Thermochemistry}

\section{Energy Basics}
\begin{itemize}
	\item Thermochemistry is the study of heat and energy changes in chemical reactions
	\item It also includes topics like entropy and spontaneity
	\item Energy can come in two forms, kinetic and potential (Figure 5.3)
    \begin{itemize}
      \item Kinetic energy is the energy of motion: $KE=\dfrac{1}{2}mv^2$
      \item For chemists, the kinetic energy that matters is the movement of individual atoms, molecules, etc., manifested as temperature (Figure 5.4)
      \item Heat will always flow from colder objects to warmer objects (Figure 5.6)
      \item Increases in temperature almost always cause objects to increase in volume (Figure 5.5)
      \item Potential energy is stored energy: Gravitaional, electrostatic, chemical, etc.
      \item For chemists, the potential energy that matters is the energy of chemical bonds and intermolecular forces (Figure 5.2)
    \end{itemize}
	\item Energy has several common units
    \begin{itemize}
      \item The SI unit is the Joule: $1~J=1~kg\dfrac{m^2}{s^2}$
      \item The calorie: $1~cal=4.184~J$
      \item The Calorie (kcalorie): $1~Cal=1000~cal=4184~J$
    \end{itemize}
  \item When we talk about changes and transfers of energy, we need to carefully define our system
    \begin{itemize}
      \item The \emph{system} is the part of the universe where the reaction occurs, such as a beaker or chamber with reactants
      \item The \emph{surroundings} is the rest of the universe
      \item Open systems can exchange both heat and matter with the surroundings (an open beaker)
      \item Closed systems can exchange heat, but not matter with the surroundings (a closed chamber)
      \item Isolated systems cannot exchange either heat or matter with the surroundings (a closed, insulated flask)
    \end{itemize}
	\item Heat from chemical and physical changes is usually associated with temperature changes or phase changes (Figure 5.7)
    \begin{itemize}
      \item These examples all refer to heat from the \emph{system's} perspective
      \item Positive heat is observed by an “upward” phase change or a cold temperature
      \item Holding ice in your hand will melt the ice, and make your hand cold
      \item Negative heat is observed by a “downward” phase change or a hot temperature
      \item Burning wood in a fire feels warm because of the negative system heat
    \end{itemize}
  \item Heat and temperature change are related by several equations (Note that we are always measuring the temperature of the \emph{surroundings})
    \begin{itemize}
      \item When heat is added to a system, it will either undergo a phase change, or heat up
      \item The specific heat gives how much heat is required to warm a given substance
      \item For a complete system, $q=C\Delta T$ where $C$ is the \emph{heat capacity}
      \item For a pure substance, $q=mc_S\Delta T$ where $c_S$ is the \emph{specific heat} of the substance and $m$ is the mass
      \item Table 5.1 shows $c_S$ for many common substances

        Practice: Find how much heat is required to raise the temperature of $15.0g$ of iron by $32^\circ C$

        Practice: Find the temperature change when $42.5J$ of heat are added to $0.374g$ of aluminum
    \end{itemize}
\end{itemize}

\paragraph*{Quiz 5.1 - Energy}
\paragraph*{Homework 5.1}
\begin{itemize}
  \item 5: Heat capacity of samples
  \item 9: $q=mc_S\Delta T$
  \item 11: $q=mc_S\Delta T$, but solve for $c_S$
\end{itemize}

\section{Calorimetry}
\begin{itemize}
  \item Whenever heat is transferred, the total energy of the universe remains constant
	\item For a heat transfer in an isolated system, $q_1=-q_2$
	\item Consider a block of hot metal placed in a beaker of room temperature water. Heat will flow from the block into the water until the two temperature are equal to each other
	\item $q_1=-q_2$ becomes $m_1c_1\left(T_f-T_{i,1}\right)=-m_2c_2\left(T_f-T_{i,2}\right)$

	      Practice: A $10.0~g$ block of iron is heated to $93.5~^\circ C$ and placed in $25~ml$ of $23.0~^\circ C$ water.\\
	      ~\hphantom{Practice:} What is the final temperature? ($25.9~^\circ C$)
	\item We can also measure the heat transfer associated with a chemical reaction, called \emph{calorimetry}
  \item Exothermic processes release heat into the surroundings (warm to the touch)
  \item Endothermic processes absorb heat from the surroundings (cool to the touch)
  \item Figure 5.11 shows how a calorimeter would work for endothermic and exothermic processes
  \item Figures 5.12 and 5.13 show a constant pressure (coffee-cup) calorimeter
    \begin{itemize}
      \item The reaction is carried out in aqueous solution
      \item $q_{rxn}$ is the heat released or absorbed by the reaction
      \item The heat of reaction is exchanged with the solution: $q_{rxn} = -q_{soln} = -mc\Delta T$
      \item $m$ is the solution mass, which will include the water and any solutes
      \item $c$ is the solution specific heat, but this is simplified by assuming $c_{soln}=c_{water}=4.184\frac{J}{g~^\circ C}$
      \item $\Delta H = \dfrac{q_{rxn}}{n_{rxn}} = \dfrac{-mc\Delta T}{n_{rxn}}$ where $n_{rxn}$ is the moles of reaction: $n_{rxn}=\left(\dfrac{n_A}{\nu_A}\right)$
      \item Demo -- \ch{NaOH} enthalpy of solvation $\left(10~g, 100~ml, –44.2~\frac{kJ}{mol}\right)$
    \end{itemize}
  \item Figure 5.17 shows a constant volume (bomb) calorimeter
    \begin{itemize}
      \item The reaction is carried out in a chamber charged with high pressure \ch{O2}
      \item $q_{rxn}$ is exchanged with the whole bomb-calorimeter apparatus
      \item The calorimeter is calibrated to give a \emph{heat capacity} ($C_{cal}$) with units $\frac{J}{^\circ C}$
      \item $q_{rxn}=-q_{cal}=-C_{cal}\Delta T$
      \item For constant volume, we measure $U$ instead of $H$ because $w=0$
      \item $\Delta U = \dfrac{-C_{cal}\Delta T}{n_{rxn}}$
    \end{itemize}
\end{itemize}

\paragraph*{Quiz 5.2 - Calorimetry}
\paragraph*{Homework 5.2}
\begin{itemize}
  \item 19: Reaching thermal equilibrium 
  \item 25: Coffee cup calorimetry
  \item 31: Bomb calorimetry
\end{itemize}

\section{Enthalpy}
\begin{itemize}
  \item The sum of all types of energy in a system is the \emph{internal energy}, $U$
  \item Any change in the internal energy must come from heat or work: $\Delta U=q+w$
  \item Work is defined as $w=f\cdot d$ or $w=-P\cdot\Delta V$
    \begin{itemize}
      \item Positive work is when the system volume decreases
      \item Negative work is when the system volume increases, or a force moves part of the surroundings
      \item This $PV$ work is actually a bit problematic when trying to keep track of energy
        \begin{itemize}
		      \item $T$ is easy to measure with a thermometer, but both $P$ and (especially) $V$ are more difficult to measure
		      \item Most of our work as chemists is done at constant pressure (open flask or in a balloon)
		      \item Under constant pressure, we can use \emph{Enthalpy} ($H$) instead of internal energy ($U$)
          \item Technically, $H=U+PV$
		      \item While $U=q+w$, $H=q$ under constant pressure conditions (derivation in the book)
		      \item So, we only need to worry about heat when we deal with $H$
	      \end{itemize}
    \end{itemize}
	\item Heat is usually associated with temperature changes or phase changes, as discussed previously
	\item First Law of Thermodynamics: The energy of the universe is constant 
  \item Enthalpy in chemical reactions:
  \begin{itemize}
    \item A balanced chemical reaction may also include an enthalpy of reaction $\Delta H$
    \item This tells how much heat is produced or consumed with one mole of reaction
    \item $\Delta H$ can be a conversion factor between heat and amounts of reactants or products

          Practice: Consider the reaction \ch{N2(g) + 3 H2(g) -> 2 NH3(g)} $\Delta H = -92\frac{kJ}{mol}$

          ~\hphantom{Practice:} If $2.25~g$ of \ch{H2} are consumed in the above reaction, how much heat is released? ($34.2~kJ$)

          ~\hphantom{Practice:} If $54.6~kJ$ of heat are released, how many $g$ of \ch{NH3} will be produced? ($20.2~g$)
  \end{itemize}
\end{itemize}

\paragraph*{Quiz 5.3 - Enthalpy}
\paragraph*{Homework 5.3}
\begin{itemize}
  \item 41: Enthalpy of reaction from calorimetry
  \item 45: Heath from enthalpy of reaction
  \item 47: How much reactant from target heat output
\end{itemize}

\paragraph*{Resume Section 5.3 -- Enthalpy}
\begin{itemize}
  \item The ``Standard State'' is notated by $^\circ$ and is at $1M$ concentration and $1atm$ (or $1bar$), and technically doesn't include a temperature but values are often tabulated at $25^\circ C$
  \item Standard Enthalpies of Combusion ($\Delta H^\circ_C$):
    \begin{itemize}
      \item Standard enthalpy of combustion is the energy released when something reacts with oxygen
      \item Table 5.2 Lists the enthalpies of combustion for many combustible substances
      \item These values are not synonymous with energy density, due to molar mass and density complicating the values
    \end{itemize}
  \item Because enthalpy is a state function (Figure 5.20), we can calculate values of $\Delta H$ without measuring them
  \item Hess's Law: Any alternate path with the same starting and ending states will have the same\\overall $\Delta H$
  \begin{itemize}
    \item Drawing an energy level diagram can help to illustrate Hess's law (Figure 5.24 as an example)
    \item Find $\Delta H$ for this reaction: \ch{C_{diamond} + O2(g)->CO2(g)} \hspace{1em} $\left(\Delta H=-395.4~\nicefrac{kJ}{mol}\right)$

          \ch{C_{diamond} -> C_{graphite}} \hspace{2em} $\Delta H=-1.9~\nicefrac{kJ}{mol}$

          \ch{C_{graphite} + O2(g)->CO2(g)} \hspace{2em} $\Delta H=-393.5~\nicefrac{kJ}{mol}$
    \item The reverse of a reaction gives $-\Delta H$
    \item Find $\Delta H$ for this reaction: \ch{C(s) + 1/2 O2(g)->CO(g)} \hspace{1em} $\left(\Delta H=-110~\nicefrac{kJ}{mol}\right)$

          \ch{C(s) + O2(g) -> CO2(g)} \hspace{2em} $\Delta H=-393~\nicefrac{kJ}{mol}$

          \ch{CO(g) + 1/2 O2(g) -> CO2(g)} \hspace{2em} $\Delta H=-283~\nicefrac{kJ}{mol}$

    \item Double the reaction gives double the $\Delta H$
    \item Consider trying to find $\Delta H$ for the reaction below:

          \circled{$\star$} \ch{C2H5OH(l) + 2 O2(g) -> 2 CO(g) + 3 H2O(l)} \hspace{2em} $\Delta H = ?$
    \item Find an alternate path using these reactions with known $\Delta H$:

          \circled{A} \ch{C2H5OH(l) + 3 O2(g) -> 2 CO2(g) + 3 H2O(l)} \hspace{2em} $\Delta H=-1367~\nicefrac{kJ}{mol}$

          \circled{B} \ch{CO(g) + 1/2 O2(g) -> CO2(g)} \hspace{2em} $\Delta H=-283~\nicefrac{kJ}{mol}$
    \item The enthalpy of the first reaction can be found from the enthalpies of the other two
    \item $\Delta H_{\star} = \Delta H_A - 2 \Delta H_B = -801\dfrac{kJ}{mol}$ \hspace{1em} (Draw the energy level diagram)

          Practice: Find the enthalpy of reaction \circled{$\star$} using reactions \circled{A}, \circled{B}, and \circled{C}

          \circled{$\star$} \ch{CS_2(l) + 3 O2(g) -> CO2(g) + 2 SO2(g)} \hspace{2em} $\left(\Delta H = -1075.0\right)$

          \circled{A} \ch{C(s) + O2(g) -> CO2(g)} \hspace{2em} $\Delta H=-393.5~\nicefrac{kJ}{mol}$

          \circled{B} \ch{S(s) + O2(g) -> SO2(g)} \hspace{2em} $\Delta H=-296.8~\nicefrac{kJ}{mol}$

          \circled{C} \ch{C(s) + 2 S(s) -> CS2(l)} \hspace{2em} $\Delta H = 87.9~\nicefrac{kJ}{mol}$
  \end{itemize}
\end{itemize}

\paragraph*{Quiz 5.4 - Hess's Law}
\paragraph*{Homework 5.4}
\begin{itemize}
  \item 59: Hess's Law and enthalpy of reaction
  \item 63: Hess's Law and enthalpy of reaction
\end{itemize}

\paragraph*{Resume Section 5.3 -- Enthalpy}
\begin{itemize}
  \item Standard Enthalpies of Formation $\Delta H_{f}^\circ$
    \begin{itemize}
      \item To apply Hess's law to arbitrary reactions, you would need to devise an alternate path from an encyclopedia of known reactions -- this would be \emph{very} inconvenient
      \item Instead of using random reactions from one state to another, it is useful to devise a \emph{standard state} for each element
      \item The standard state is the most stable form of that element e.g. for O, it is \ch{O2(g)}, not \ch{O2(l)} or \ch{O3(g)}
      \item Each compound will have a \emph{standard formation reaction} which forms it from its elements in their standard state

            For water, that's \ch{H2(g) + 1/2 O2(g) -> H2O(l)} \hspace{1em} (This is one time a \ch{1/2} coefficient is acceptable)
      \item The enthalpy for this reaction is called the compound's \emph{Standard Enthalpy of Formation} $\left(\Delta H^\circ_f\right)$
      \item Elements in their standard state have $\Delta H^\circ_f = 0$
      \item Any reaction can be framed as a combination of standard formation reactions
        \begin{itemize}
          \item First, the reactants are broken down into their elements (the \emph{reverse} of formation reactions)
          \item Then, the elements are reassembled into the products (formation reactions)
          \item The energy level diagram for any reaction is the same: reactants $\rightarrow$ elements $\rightarrow$ products
          \item This pathway doesn't need to be \emph{practical}, it is enough to be theoretically \emph{possible}
          \item $\Delta H_{rxn}=\sum\limits_{products}\nu\cdot\Delta H^\circ_f - \sum\limits_{reactants}\nu\cdot\Delta H^\circ_f$
          \item This formula is general, for any reaction at all
          \item Instead of an encyclopedia of thousands of reactions, we only need a table of $\Delta H^\circ_f$ values
          \item Appendix G includes a large number of thermodynamic values like $\Delta H^\circ_f$
        \end{itemize}
        Practice: Find $\Delta H_{rxn}$ for~~ \ch{P4O10(s) + 6 H2O(l) -> 4 H3PO4(aq)} \hspace{1em} $\left(\Delta H_{rxn}=454.6~\dfrac{kJ}{mol}\right)$

        ~\hphantom{Practice:} Find $\Delta H_{rxn}$ for~~ \ch{C3H8(g) + 5 O2(g) -> 3 CO2(g) + 4 H2O(g)} \hspace{1em} $\left(\Delta H_{rxn}=-2043.9~\dfrac{kJ}{mol}\right)$
    \end{itemize}
\end{itemize}

\paragraph*{Quiz 5.5 - Enthalpies of Formation}
\paragraph*{Homework 5.5}
\begin{itemize}
  \item 67: Calculate $\Delta H_f^\circ$ from reaction enthalpies
  \item 69: Reaction enthalpies from enthalpies of formation
\end{itemize}

\chapter{Electronic Structure and Periodic Properties of Elements}

\section{Electromagnetic Energy}
\begin{itemize}
	\item Light is an electromagnetic wave, which can be thought of like a sound wave or a wave on a lake (Figure 6.2)
	\item Light has a wavelength, frequency, and speed according to the equation: $\nu\lambda=c$
	\item The speed of light is a constant, $2.998\times10^8~\dfrac{m}{s}$
	\item How far does light travel in $5.00~ms$? ($1.50~km$)
	\item The electromagnetic spectrum is more than just visible light (Figure 6.3)
	      \begin{itemize}
		      \item Higher frequencies (shorter wavelengths) are UV light, X-rays, and gamma rays
		      \item Lower frequencies (longer wavelengths) are infrared light, microwaves, and radio waves
		      \item TV-remotes are flashy lights, and radio towers are flashlights-on-a-stick
		      \item There are important technical differences in how we can use these different kinds of light, but they are fundamentally the same thing (an alternating electromagnetic wave)
		      \item Find the wavelength of your favorite radio station ($M\!H\!z$ is a frequency of $10^6~s^{-1}$)
	      \end{itemize}
	\item Light-matter interactions were central to the discovery of modern physics
	\item The photoelectric effect was an important matter/light interaction (Figure 6.11)
	      \begin{itemize}
		      \item Sometimes light falling on a metal will eject an electron -- this is the photoelectric effect
		      \item The kinetic energy of the ejected electron can be measured
		      \item The energy depended on the wavelength of light -- bluer light ejected electrons at faster velocities
		      \item There was a threshold where electron ejection stopped, and redder light would have no effect
		      \item This was surprising, because light intensity had \emph{no} effect on the photoelectron energy
		      \item Dim blue light would eject fast electrons, bright red light would have no effect
		      \item This was eventually explained by the idea that light carries energy in small discrete packets
		      \item These packets of energy are called \emph{photons} and the energy they carry depends on the wavelength
		      \item The photoelectric effect could be described by the equation: $KE = h\nu - \phi$
		      \item Here, $\nu$ was a new constant, called Planck's constant, and $\phi$ was the metal's work function
		      \item The equation for the energy of light was: $E=h\nu$ where $h$ is Planck's constant
	      \end{itemize}
	\item Light can also be \emph{absorbed} or \emph{emitted} by matter
	      \begin{itemize}
		      \item Light is emitted by gases like \ch{Ne}, \ch{Ar}, or \ch{Na} when high voltage passes through it
		      \item Light can also be absorbed by gases and other materials
		      \item Each substance showed a unique fingerprint of wavelengths of light emitted or absorbed (Figure 6.13)
		      \item \ch{He} was first identified by its absorbance spectrum in sunlight
		      \item The unique spectra arise from the particular energy levels of a substance
		      \item These spectra showed how matter can only store or release energy in certain, constrained amounts (or \emph{quanta})
		      \item i.e. in addition to the energy of light, the energy of matter was \emph{quantized} as well
	      \end{itemize}
  \item Ultraviolet Catastrophe (Figures 6.10 and 6.9)
	\item Wave/Particle duality (not in the textbook)
	      \begin{itemize}
		      \item These experiments show that light behaves like both a \emph{wave} (interference) and a \emph{particle} (quantization)
		      \item It is wrong to say light is either of those things -- rather, it is a new thing with similarities to both (rhinosceros vs dragon + unicorn)
		      \item On very small scales, matter behaves like both a wave and a particle as well!
		      \item Electrons, in particular, are strongly wave-like, with a characteristic wavelength
		      \item The IBM quantum corral image dramatically showed the real physicality of electron waves
		      \item This wave-like nature of electrons is important for understanding modern models of atomic structure
	      \end{itemize}
  \item All of these phenomena can be explained in terms of standing waves (Figures 6.7 and 6.8)
\end{itemize}

\section{The Bohr Model}
\begin{itemize}
	\item The Rydberg Equation: (Figure 6.14)
	      \begin{itemize}
		      \item School teacher Rydberg recognized a pattern in the wavelengths of light in the \ch{H} spectrum
          \item $\dfrac{1}{\lambda}=1.097\times10^{7}m^{-1}\left(\dfrac{1}{n_1^2}-\dfrac{1}{n_2^2}\right)$
		      \item His equation can be re-written in terms of energy
		      \item $E=2.179\times10^{-18}~J\left(\dfrac{1}{n_1^2}-\dfrac{1}{n_2^2}\right)$
	      \end{itemize}
	\item The Bohr model of the atom: (Figure 6.15)
	      \begin{itemize}
		      \item There was no real explanation for \emph{why} the absorption and emission spectra of different elements showed different discreet energies
		      \item Niels Bohr proposed that electrons orbit around the nucleus only at fixed distances
		      \item Absorption is when an electron shifts to a higher orbit, using a photon's energy
		      \item Emission is when an electron shifts to a lower orbit, releasing energy as a photon
		      \item The lowest energy state is the \emph{ground state}, all others are \emph{excited states}
		      \item The discrete orbits represent states where the circumference of the orbit is equal to a number of wavelengths for the electron (Figure 6.17)
          \item $E_n=-2.179\times10^{-18}~J\left(\dfrac{1}{n^2}\right)$
	      \end{itemize}
	      Practice: What is the wavelength of the $2\leftarrow4$ transition in the \ch{H} spectrum? ($486.1~nm$)
\end{itemize}

\paragraph*{Quiz 6.1 - Light and the Bohr Model}
\paragraph*{Homework 6.1}
\begin{itemize}
  \item 3: Frequency to wavelength
  \item 5: Wavelength to energy
  \item 23: Energy of a Bohr state
  \item 27: Energy of a Bohr transition
\end{itemize}

\section{Development of Quantum Theory}
\begin{itemize}
	\item Quantum mechanics continued to develop
	\item Heissenberg Uncertainty Principle: We cannot simultaneously measure the position and velocity of an electron (or any other quantum mechanical particle)
	\item This means that we generally speak of where an electron is \emph{most probable} to be found, rather than where it \emph{actually is}
	\item The Schr\"odinger wave equation describes matter starting from a wave-like perspective
	      \begin{itemize}
          \item DeBroglie began from the equation for the momentum of light: $p=\frac{h}{\lambda}$
          \item Light has no mass, but replacing $p$ with $mv$ give us: $mv = \frac{h}{\lambda}$ and rearranging gives: $\lambda=\frac{h}{mv}$
          \item This equation describes how anything with momentum has a characteristic wavelength
          \item Figure 6.18 illustrates one way electrons behave like waves
          \item Schr\"odinger went even further, adapting Maxwell's wave equations for matter with momentum (required some postulates)
		      \item The Schr\"odinger equation gives mathematical functions which describe the electron probability distribution
		      \item Each solution is called an \emph{orbital}, like the orbits of the Bohr model but 3-dimensional
	      \end{itemize}
	\item Orbitals are organized into \emph{shells} and \emph{subshells}
	      \begin{itemize}
		      \item Subshells are groups of orbitals with similar shapes and the same energy
		      \item Subshells are named $s$, $p$, $d$, and $f$
		      \item An $s$ subshell has only one orbital (2 $e$s), $p$ has 3 (6 $e$s), $d$ has 5 (10 $e$s), and $f$ has 7 (14 $e$s)
		      \item Subshells are grouped into shells, which are indicated by numbers ($1$, $2$, $3$, etc.)
		      \item These numbers are the numbers in the Rydberg equation, and are the principle energy levels
		      \item The first shell only has an $s$ subshell, and each shell beyond that adds one type
	      \end{itemize}
	      Practice: How many electrons can be placed in a $p$ subshell? ($6$)

	      ~\hphantom{Practice:} How many electron can be placed in the \nth{3} shell? ($18$)

	      ~\hphantom{Practice:} Which of the following subshells does \emph{not} exist? $2s$, $3f$, $3p$, $5d$ ($3f$)
	\item Subshells each have orbitals with different \emph{shapes} (Figure 6.21)
	      \begin{itemize}
		      \item Because of the Heissenberg uncertainty principle, we describe region where an electron is likely to be found
		      \item These regions have shapes based on the mathematical functions which form them
		      \item $s$ orbitals are spherical, $p$ orbitals are dumbells, $d$ orbitals are clover-leafed
	      \end{itemize}
  \item Quantum Numbers
    \begin{itemize}
      \item Remember that orbitals are actually mathematical functions
      \item Certain parts of those functions depend on integer numbers (like $n$ in the Rydberg equation)
      \item These integer numbers are called \emph{quantum numbers}
      \item Quantum numbers can be thought as an ``address'' for each electron (Street, Buildling, Unit, Name)
        \begin{itemize}
          \item $n$ -- Principal quantum number ($1$, $2$, \ldots) gives orbital shell, energy, and size
          \item $l$ -- Angular momentum quantum number ($0$, $1$, \ldots, $n-1$) gives orbital subshell and shape
          \item $l$ is why not all shells have all orbital types
          \item $m_l$ -- Magnetic quantum number ($-l$, \ldots, $l$) gives orbital within a subshell (different spatial orientations)
          \item $m_l$ is why subshells have different numbers of orbitals
          \item $m_s$ -- Spin quantum number $\left(\pm \frac{1}{2}\right)$ gives ``spin'' of the electron (up- or down- arrow, Figure 6.23)
        \end{itemize}
      \item Each electron in an atom/ion must have a \emph{unique} set of quantum numbers -- Pauli Exclusion Principle
      \item You should be able to point to the right electron given a set of quantum numbers, or give the 4 quantum numbers for an indicated electron in an energy level diagram

            Practice: Give numbers or indicate electrons on an energy level diagram
    \end{itemize}
\end{itemize}


\paragraph*{Quiz 6.2 - Orbitals and Quantum Numbers}
\paragraph*{Homework 6.2}
\begin{itemize}
  \item 33: Properties associated with quantum numbers
  \item 35: Identifying the subshell from quantum numbers
  \item 41: Orbital shapes and other properties
\end{itemize}

\section{Electronic Structure of Atoms (Electron Configurations)}
\begin{itemize}
  \item Electronic energy level diagrams 
  \begin{itemize}
    \item The orbitals within each subshell are precisely equal in energy (degenerate)
    \item The subshells themselves differ in energy (their order will be explained shortly)
    \item Draw the subshells with one line for each orbital
    \item Each orbital can hold two electrons, drawn as up- and down-arrows
    \item Find the total number of electrons for the element or ion
    \item Aufbau Principle -- Fill up the orbitals with electrons from the bottom-up
    \item Hund's Rule -- Fill a subshell with one electron in each orbital before pairing them up (like roommates in an apartment)
    \item This is the ground-state configuration of the element
  \end{itemize}
  \item We can use the periodic table as a cheat-sheet to the order of the subshells and electron configurations (Figure 6.27)
    \begin{itemize}
      \item The P. T. is actually quite long -- Lanthanides and Actinides have been cut and pasted
      \item Each region of the P. T. represents a different subshell
      \item The rows represent different shells
      \item The number of elements in each block is the number of electrons each shell can hold
      \item The $d$-block and $f$-block trail the row number by 1 and 2 ($d(-1)$ and $f(-2)$)
      \item The order of the subshells is found by simply following the elements and noting in which block they reside
    \end{itemize}
	\item The arrangements of the electrons can be written as an electron configuration
	\item Write the subshells, with their number of electrons as a superscript

	      Practice: Write the electronic configurations for \ch{O}, \ch{Zr}, and \ch{Bi}
	\item Especially for large elements like \ch{Pb}, these configurations are very unwieldy
	\item We can shorten them by referencing the configuration of the \emph{preceding} noble gas
	\item The electrons which make up this noble gas configuration are buried inside the atom, and called \emph{core} electrons
	\item For \ch{Bi}, we get $\left[\ch{Xe}\right]6s^24f^{14}5d^{10}6p^3$
	\item The outermost electrons (The ones we write) are called \emph{valence} electrons, and are the ones involved in bonding and ion formation
	\item A few transition metals have anomalous configurations (memorize only \ch{Cr} and \ch{Cu})
	\item The Lanthanum and Actinium boundary also shows some anomalies (don't memorize them) 
  \item Figure 6.29 gives the valence configuration of every element's ground state
	\item Technically, filled $d$ and $f$ subshells count as \emph{core}, and only the outermost $s$ and $p$ electrons will always count as \emph{valence}
	\item This means that only the main group elements have a reliable pattern in their number of valence electrons
	\item The number of valence electrons is the same as the ``A'' column names
	\item We can write the configurations of ions as well
    \begin{itemize}
      \item For most, simply add or remove electrons according to the normal pattern
      \item Transition metals will lose the outermost $s$ electrons before they lose any $d$ electrons (This is why so many transition metals have a stable $2+$ ion)
      \item Ions and atoms with identical configurations are called \emph{isoelectronic} to each other
    \end{itemize}
    Practice: List several stable ions which are isoelectronic with \ch{Ar}
\end{itemize}

\paragraph*{Quiz 6.3 - Electronic Configurations}
\paragraph*{Homework 6.3}
\begin{itemize}
  \item 49: Complete electron configurations
  \item 53: Energy level diagram of valence electrons
  \item 55: Identify an atom from its configuration
  \item 57: Identify an ion from its configuration
  \item 63: Electron configuration of an ion
  \item 79: Isoelectronic configurations
\end{itemize}

\section{Periodic Variations in Element Properties}
\begin{itemize}
  \item The sizes of atoms and ions is controlled by the attractive and repulsive forces between electrons and protons
    \begin{itemize}
      \item Electron and protons attract each other, shrinking the atomic size
      \item Electrons repel each other, increasing the atomic size
    \end{itemize}
	\item Effective nuclear charge $\left(Z_{eff}\right)$ attempts to summarize these interactions
    \begin{itemize}
      \item The actual nuclear charge ($Z$) is just the number of protons (quite high for larger elements)
      \item Core electrons will counteract much of the actual nuclear charge (called \emph{shielding})
      \item $Z_{eff} = Z-S$ \hspace{2em} $S$ stands for ``Shielding''
      \item $S$ can be closely approximated by the number of core electrons

            Practice: Find $Z_{eff}$ for \ch{Mg}, \ch{S}, and \ch{Br} ($2$, $6$, and $7$)
      \item Slater's rules gives a more sophisticated and accurate value for $S$
      \item $1$ for deep core electrons, $0.85$ for $V-1$ electrons, and $0.35$ for all but one $V$ electron
      \item Note that Slater's rules are not in your textbook

            Practice: Find $Z_{eff}$ for the same elements using Slater's rules ($2.85$, $5.45$, and $7.6$)
    \end{itemize}
	\item Atomic radius increases down a column because you are adding and entire new shell for each row
	\item Atomic radius decreases across a row because of the increasing $Z_{eff}$
	\item This makes \ch{He} the smallest element, and \ch{Fr} the largest (Figure 6.30 and 6.31)
	\item Anions are much larger and cations are much smaller than their neutral counterparts (Figure 6.32)
  \item Ionization energy (IE) is the energy required to remove an electron
    \begin{itemize}
      \item For example, it is the energy for this process: \ch{Li -> Li^+ + e^-}
      \item The two factors which control IE are radius, and $Z_{eff}$
      \item It is easier to remove electrons (smaller IE) from larger atoms
      \item It is easier to remove electrons (smaller IE) from atoms with lower $Z_{eff}$
      \item Opposite to radius, \ch{He} has the highest IE, and \ch{Fr} has the lowest IE
      \item There are breaks in this trend at the beginning and middle of the $p$ block (Figure 6.33 and 6.34)
    \end{itemize}
	\item Second- and third- ionization energy is the energy to remove a second and third electron (Table 6.3)
    \begin{itemize}
      \item Each successive electron is harder to remove
      \item After the valence electrons are gone, removing a core electron is \emph{much} harder to remove
    \end{itemize}
	\item Electron affinity (EA) is the energy released when an electron is added
    \begin{itemize}
      \item Electron affinity is usually exothermic, so these values are mostly negative
      \item The \emph{magnitude} of EA follows the same trend as IE
      \item There are lots of breaks in the trend (Figure 6.35 is a mess), so don't worry too much about EA
    \end{itemize}
\end{itemize}

\paragraph*{Quiz 6.4 - Periodic Trends}
\paragraph*{Homework 6.4}
\begin{itemize}
  \item 67: Atomic radius trend
  \item 71: Ionization energy trend
  \item 75: Ranking atomic radii
  \item 77: Ranking ionic radii
  \item 85: Second ionization energy
\end{itemize}

\chapter{Chemical Bonding and Molecular Geometry}

\section{Ionic Bonding}
\begin{itemize}
	\item Metal and nonmetal elemental atoms will react to form an ionic compound (Figure 7.2)
	\item The number of electrons gained/lost will result in noble gas configurations for both elements
	\item The cation and anion are now attracted to each other, and bind together in a lattice structure
	\item A \emph{formula unit} is the smallest unit which builds the extended lattice (Figure 7.3)
  \item Ions will combine in the correct ratio to balance the charge
    \begin{itemize}
      \item Find the least common multiple of the positive and negative charges
      \item Some high school classes teach a shortcut that doesn't always work: (\ch{TiO2})
    \end{itemize}
  \item Electronic configuration of many ions will be isoelectronic with a noble gas
\end{itemize}

\section{Covalent Bonding}
\begin{itemize}
	\item In both ionic and covalent compounds, bonds will form to complete the atoms' \emph{octets} (\emph{duets} in the case of \ch{H})
	\item Instead of transferring electrons to form ions, covalent compounds will share electrons
	\item The electrons in a covalent bond will count toward the octets of both bonding partners
	\item Drawing dot structures, we can see how many bonds an atom might need to form to fill its octet
	\item We'll usually represent two shared electrons in a bond by a dash -- this is the beginning of Lewis structures
	\item Electron pairs not involved in a bond are called \emph{lone pairs}
	\item Single, Double and Triple bonds share 2, 4, and 6 electrons, respectively
	\item Double and triple bonds are shorter and stronger than single bonds

	      Practice: Draw dot diagrams and Lewis diagrams for the diatomics \ch{F2}, \ch{O2}, and \ch{N2}\\
	      ~\hphantom{Practice: } Identify the bonds by their type, as well as any lone pairs
  \item Figure 7.4 shows the energetics of covalent bonds
  \item Bond polarity
    \begin{itemize}
      \item Figure 7.5 shows electrons being shared unequally between bonding partners
      \item The electrons will favor the element with greater electronegativity (Figure 7.6)
      \item The uneven distribution of charge is called a \emph{dipole} and the bond is called \emph{polar}
      \item Polar covalent bonds exist along a continuum from purely covalent to purely ionic (Figure 7.8)
      \item Greater electronegativity differences give greater ionic character to the bond
      \item Similar (or identical) electronegativities create non-polar bonds
      \item The dipole moment is $\mu=qr$, and can be measured
      \item Real dipole moments can be compared to the value for a complete transfer of electrons to give $\%~ionic~character$
      \item $\%~ionic~character=\dfrac{\mu_{measured}}{\mu_{ionic}}\times100\%$
    \end{itemize}
\end{itemize}

\section{Lewis Symbols and Structures}
\begin{itemize}
    \item For more complex molecules, dot diagrams are not robust enough to predict the proper structure
    \item Follow these 8 steps:
          \begin{itemize}
            \item Find the total number of valence electrons, considering any overall charge
            \item Write the peripheral atoms around the central atom
                  \begin{itemize}
                    \item Central atom will be the least electronegative (except \ch{H})
                    \item This is often the first atom written in a formula (except \ch{H})
                    \item \ch{H} will never be the central atom because it can only form 1 bond
                  \end{itemize}
            \item Connect each peripheral atom to the central atom with single bonds
            \item Place the remaining valence electrons around the peripheral atoms until their octets are complete
            \item If any electrons remain, place them on the central atom
            \item Convert peripheral lone pairs into double or triple bonds according to these two rules
                  \begin{itemize}
                    \item Make multiple bonds until the central atom has a complete octet
                    \item Make multiple bonds in a way that minimizes \emph{formal charges} (more on this later)
                  \end{itemize}
            \item Verify that the octet and duet rule are followed for all atoms in the structure
            \item If there is a charge, enclose the structure in square brackets and write the charge
          \end{itemize}

          Practice: \ch{CH4}, \ch{H2O}, \ch{NH3}, \ch{HCN}, \ch{CH2O}, \ch{NO3^-}, \ch{CO3^{2-}}, and \ch{NH4^+}
\end{itemize}

\paragraph*{Quiz 7.1 - Lewis Structures}
\paragraph*{Homework 7.1}
\begin{itemize}
  \item 15: Comparing electronegativities
  \item 21: Comparing bond polarities
  \item 29: Drawing Lewis structures
  \item 31: More challenging Lewis structures
\end{itemize}

\section{Formal Charges and Resonance}
\begin{itemize}
  \item Resonance and Formal Charge
    \begin{itemize}
      \item Often times we will have an arbitrary choice about which peripheral atom to form a double bond with
      \item Consider \ch{CO3^{2-}}, you could form the double bond with any of the three oxygens
      \item In these instances, the molecules will exhibit a phenomenon called \emph{resonance}
            \begin{itemize}
              \item Resonance is when electrons in a double (or triple) bond are shared between two or more bonding locations
              \item \ch{CO3^{2-}} forms a bond with \emph{all three} oxygen atoms
              \item It is not nearly as strong as a normal double bond, because the electrons are spread between three bonding centers
              \item This is called \emph{delocalization}, and sometimes the trio of bonding locations are collectively called a delocalized bond
              \item The bond lengths are all the same, somewhere between a single and double bond length
            \end{itemize}
      \item We represent resonance two ways: First, resonance structures
            \begin{itemize}
              \item Draw a different structure for each bonding location
              \item Enclose each structure in square brackets (even for neutral structures)
              \item Draw double-headed arrows between the structures
            \end{itemize}
      \item The other method is hybrid structures
            \begin{itemize}
              \item Draw a single structure, with dotted lines for each bonding location
              \item These structures are closer to an accurate picture of reality
              \item Counting electrons in these structures is impossible (bonds and lone-pairs)
            \end{itemize}

            Practice: Draw the resonance structures and hybrid structure for \ch{NO3^-} and \ch{O3}
      \item Formal charge compares how many electrons an atom has within an molecule to its number of valence electrons
            \begin{itemize}
              \item First draw a Lewis structure
              \item Count the electrons around an atom, dividing the bonding electrons between bonding partners
              \item Subtract this number from the number of valence atoms
            \end{itemize}
      \item Formal charge can be used to determine which structures are best when there are options
            \begin{itemize}
              \item Minimize the total formal charge
              \item Favor structures with negative formal charges on the more electronegative atoms
            \end{itemize}

            Practice: Find the best structures for \ch{OCN} and \ch{N2O}
    \end{itemize}
\end{itemize}

\paragraph*{Quiz 7.2 - Formal Charges and Resonance}
\paragraph*{Homework 7.2}
\begin{itemize}
  \item 45: Resonance in Lewis Structures (Note that some figures are omitting the lone pair and multiple-bond electrons)
  \item 51: Determining formal charges
  \item 55: Formal charges in resonance structures
  \item 57: Using formal charge to determine the best structure
\end{itemize}

\paragraph*{Resume Section 7.4}
\begin{itemize}
  \item Exceptions to the Octet Rule
    \begin{itemize}
      \item A few elements will have \emph{fewer} than 8 electrons
            \begin{itemize}
              \item \ch{Be} has only 2 electrons, so it can form only 2 bonds
              \item \ch{B} has only 3 electrons, so it can form only 3 bonds
            \end{itemize}
      \item Radicals are compounds with an unpaired electron
            \begin{itemize}
              \item Radicals tend to be very reactive
              \item Many radicals contain \ch{N} (\ch{NO} and \ch{NO2})
            \end{itemize}
      \item Some elements can have \emph{more} than 8 electrons
            \begin{itemize}
              \item Only elements in the 3rd row or below can exceed the octet rule
              \item This is because the extra electrons go into the $d$ subshell
              \item This can occur in order to minimize formal charges

                    Practice: Find proper Lewis structures for \ch{SO4^{2-}}, \ch{PO4^{3-}}, \ch{ClO3^{-}}, and \ch{ClO4^{-}}
              \item It can also occur because there are simply too many outer atoms

                    Practice: Find proper Lewis structures for \ch{SF6}, \ch{ClF5}, and \ch{PCl5}
            \end{itemize}
    \end{itemize}
\end{itemize}

\paragraph*{Quiz 7.3 - Violations of the Octet Rule}
\paragraph*{Homework 7.3}
\begin{itemize}
  \item 58: Sulfur dioxide, how formal charge motivates violations to the octet rule
  \item 63: Sulfuric acid structure
\end{itemize}

\section{Strengths of Ionic and Covalent Bonds}
\begin{itemize}
  \item Covanet Bond Enthalpy
    \begin{itemize}
      \item Bond enthalpy is the energy required to break a bond
      \item Really, not every \ch{C-H} bond is the same. The surrounding atoms affect bond enthalpy
      \item We can take the average enthalpy of a given bond type over many molecules (Table 7.2)
      \item This gives us a new way to calculate reaction enthalpies
            \begin{itemize}
              \item Draw a Lewis structure for each reactant and product
              \item identify the numbers and types of bonds
              \item Consider first breaking all the bonds of the reactants to produce individual atoms
              \item Then form new bonds between the atoms to make products
              \item $\Delta H_{rxn}=\sum\limits_{Bonds~Broken} n\cdot \Delta H_{Bond} - \sum\limits_{Bonds~Formed}n\cdot \Delta H_{Bond}$
              \item This will give an \emph{approximate} reaction enthalpy
            \end{itemize}
      \item Table 7.3 gives average values for bond lengths as well

            Practice: Using Table 7.2, estimate the reaction enthalpy for the combustion of methane $\left(-810~\dfrac{kJ}{mol}\right)$
    \end{itemize}
\end{itemize}

\paragraph*{Quiz 7.4 - Bond Enthalpies}
\paragraph*{Homework 7.4}
\begin{itemize}
  \item 65: $\Delta H_{rxn}$ based on bond enthalpies
  \item 67: Determining the more stable isomer
\end{itemize}

\paragraph*{Resume section 7.5}
\begin{itemize}
  \item Lattice Energy
    \begin{itemize}
      \item Consider the reaction: \ch{K(s) + 1/2 Cl2(g) -> KCl(s)} \hspace{1em}$\Delta H^\circ_f = -437~\dfrac{kJ}{mol}$
      \item The energy of formation is more than just the IE of the metal and EA of the non-metal
      \item The Born-Haber Cycle is an alternate route for formation reactions of ionic compounds:
        \begin{itemize}
          \item The steps of the Born-Haber cycle are shown in Figure 7.13
          \item Sublimating the metal, breaking the diatomic bond, IE, EA, and lattice energy
          \item The lattice energy is directly related to the attractive force between the ions
        \end{itemize}
      \item Lattice energy is controlled by two factors:
        \begin{itemize}
          \item Ions with larger charges have higher lattice energies
          \item Smaller ions have higher lattice energies
        \end{itemize}
    \end{itemize}
\end{itemize}

\paragraph*{Quiz 7.5 - Ionic Bonds}
\paragraph*{Homework 7.5}
\begin{itemize}
  \item 77: Identifying smallest lattice energy
  \item 79: Effect of charge on lattice energy
  \item 81: Comparing lattice energies
\end{itemize}

\section{Molecular Structure and Polarity}
\begin{itemize}
  \item VSEPR Theory and Geometry
    \begin{itemize}
      \item The three-dimensional shape of a molecule is important to its properties
      \item \emph{Isomers} are molecules with the same chemical formula, but a different 3-dimensional arrangement of its atoms
      \item Isomers might have quite different properties despite their identical composition
      \item Valence Shell Electron Pair Repulsion (VSEPR) model: (Figure 7.16)
            \begin{itemize}
              \item VSEPR is based on the idea that electron pairs will arrange themselves to be as far apart from each other as possible.
              \item This model gives accurate geometries for covalent molecules
              \item First, draw a good Lewis structure
              \item \emph{Electron Domains} are the regions around the central atom where electrons group -- A single bond, a double bond, a triple bond, and a lone pair are all electron domains
              \item The number of electron domains will give the \emph{electron geometry} -- This electron geometry is the template on which molecular geometry is based
              \item Demo: Balloons naturally adopt the electron geometries
              \item Next, count how many domains are bonding vs lone pairs
              \item The number of bonds within the electron geometry determines the molecular geometry (Figure 7.18)
              \item Figure 7.19 sums it all up nicely -- Just memorize this table
              \item Trigonal Bipyramidal electron geometry
                    \begin{itemize}
                      \item The 2 axial and 3 equatorial positions are different (Figure 7.20)
                      \item Lone pairs will occupy the equatorial positions first
                      \item Linear molecules are symmetrical (this will matter later)
                    \end{itemize}
              \item Octahedral electron geometry
                    \begin{itemize}
                      \item All the positions are equivalent
                      \item The second lone pair will be opposite the first one
                      \item Square planar molecules are symmetrical (this will matter later)
                    \end{itemize}
            \end{itemize}
      \item For larger molecules, you can apply VSEPR to each bonding center (consider \ch{CH3CO2H})
    \end{itemize}
  \item Molecular Polarity
    \begin{itemize}
      \item We already discussed electronegativity and its role in making polar \emph{bonds}
      \item For molecules with many polar bonds, those dipoles might cancel each other out or work together to make a polar molecule
      \item The \emph{molecular dipole} is the vector sum of all the bond dipoles (Figure 7.27)
      \item Factors that make non-polar molecules:
            \begin{itemize}
              \item No polar bonds like diatomic elements and \ch{O3}
              \item Symmetry in the polar bonds (no lone pairs, or linear and square planar molecules)
            \end{itemize}
      \item Factors that make polar-molecules:
            \begin{itemize}
              \item Lone pairs which break symmetry
              \item Bonds with different atoms (\ch{CH2Cl2})
            \end{itemize}
    \end{itemize}
\end{itemize}

\paragraph*{Quiz 7.6 - Molecular Geometry and Polarity}
\paragraph*{Homework 7.6}
\begin{itemize}
  \item 87: Electron geometry vs. molecular geometry
  \item 89: Non-polar molecules which contain polar bonds
  \item 93: Finding electron and molecular geometries
  \item 97: Identifying molecular dipoles 
\end{itemize}
\chapter{Advanced Theories of Covalent Bonding}

\section{Valence Bond Theory}
% \begin{itemize}
%   \item 
% \end{itemize}

\section{Hybrid Atomic Orbitals}
% \begin{itemize}
%   \item 
% \end{itemize}

\section{Multiple Bonds}
% \begin{itemize}
%   \item 
% \end{itemize}

\section{Molecular Orbital Theory}
% \begin{itemize}
%   \item 
% \end{itemize}

\chapter{Gases}

\section{Gas Pressure}
% \begin{itemize}
%   \item 
% \end{itemize}

\section{Relating Pressure, Volume, Amount, and Temperature: The Ideal Gas Law}
% \begin{itemize}
%   \item 
% \end{itemize}

\section{Stoichiometry of Gaseous Substances, Mixtures, and Reactions}
% \begin{itemize}
%   \item 
% \end{itemize}

\section{Effusion and Diffusion of Gases}
% \begin{itemize}
%   \item 
% \end{itemize}

\section{The Kinetic-Molecular Theory}
% \begin{itemize}
%   \item 
% \end{itemize}

\section{Non-Ideal Gas Behavior}
% \begin{itemize}
%   \item 
% \end{itemize}

\chapter{Liquids and Solids}

\section{Intermolecular Forces}
\begin{itemize}
  \item Many physical properties of solids, liquids, and gases can be explained by the strength of attractive forces between particles (Figure 10.5)
  \item Phase changes happen due to the interplay between kinetic energy and intermolecular forces (Figure 10.2)
  \item Pressure can also play a role in phase changes, as discussed later
  \item These \emph{intermolecular forces} come in different varieties
  \begin{itemize}
    \item Dispersion Forces Non-polar molecules, impacted by polarizability, molecular weight, and surface area
    \begin{itemize}
      \item Dominant in non-polar molecules
      \item Created by induced dipoles (Figure 10.6)
      \item Impacted by polarizability (Table 10.1)
      \item Impacted by molecular weight (hydrocarbons from methane to wax)
      \item Impacted by molecule shape (Figure 10.7 compares the boiling points of pentane isomers)
    \end{itemize}
    \item Dipole-Dipole Forces
    \begin{itemize}
      \item Dominant in polar molecules
      \item Results from attraction between permanent dipoles (Figure 10.9)
    \end{itemize}
  \item Hydrogen Bonding
    \begin{itemize}
      \item Dominant only in molecules capable of hydrogen bonding
      \item Must contain a hydrogen-donor atom (H attached to N, O, or F)
      \item Must contain a hydrogen-acceptor atom (lone pair of electrons attached to N, O, or F)
      \item Hydrogen bonds are more than just particularly strong dipole-dipole forces. They have strong directionality according to VSEPR
      \item Figures 10.10, 10.14, and other figures on the Internet show water, DNA, and proteins all organized by hydrogen bonds
      \item Figures 10.11 and 10.12 illustrate how much hydrogen bonds exceed dipole-dipole forces in strength
    \end{itemize}
  \end{itemize}
\end{itemize}
\backmatter
\chapter{Errata}
\end{document}
