\documentclass[12pt, letterpaper]{memoir}
\usepackage{ExamStyle}

\begin{document}
	\mainmatter
	
	\begin{center}
		{\Huge CHEM 1210}
		{\LARGE-- Fall 2016
		
		Midterm Exam 3 Study Guide (Ch. 6-8)}
	\end{center}
	
	This study guide is meant to provide only the barest direction as you study. Try to find practice problems from the textbook (both in the chapter text and in the end-of-chapter questions) rather than just relying on this guide. Note that tables and equations provided here will {\large\bfseries NOT} necessarily be given on the exam.

	\subsection*{Chapter 6 -- Atomic Structure of Atoms}
	\begin{itemize}
		\item Light and Quantum Mechanics (wave/particle duality)
		\item Equations for light: $E=h\nu$ and $\nu\lambda = c$
		\item Origins of atomic line spectra
		\item Rydberg Equation:
		\begin{itemize}
			\item $E=R_H\left(\dfrac{1}{n^2}\right)$
			\item $\Delta E_{n_f\leftarrow n_i} = R_H\left(\dfrac{1}{n_f^2}-\dfrac{1}{n_i^2}\right)$
			\item $R_H = -2.18\times10^{-18}J$
		\end{itemize}
		\item Matter Waves: $\lambda = \dfrac{h}{mv}$
		\item Heissenberg Uncertainty Principle: $\Delta x \Delta (mv) \geq \dfrac{h}{4\pi}$
		\item Quantum numbers for an electron:
		\begin{itemize}
			\item $n$ 
			\begin{itemize}
				\item Principle quantum number
				\item Which shell and what energy
				\item $ n = 1, 2, 3,\ldots$
			\end{itemize}
			\item $l$ 
			\begin{itemize}
				\item Angular momentum quantum number
				\item What type of orbital (s, p, d, f)
				\item $ l = 0, 1, 2, \ldots, n-1$
			\end{itemize}
			\item $m_l$ or $m$ 
			\begin{itemize}
				\item Magnetic quantum number
				\item Which orbital in a subshell
				\item $m_l = -l, -l+1,\ldots,l-1, l$
			\end{itemize}
			\item $m_s$ or $s$ 
			\begin{itemize}
				\item Spin quantum number
				\item Spin up $\uparrow$ or spin down $\downarrow$
				\item $m_s = \pm \dfrac{1}{2}$
			\end{itemize}
		\end{itemize}
		\item Shapes of orbitals
		\begin{itemize}
			\item s, p, d, and f shapes
			\item Radial nodes $= n-l-1$
			\item Angular nodes $= l$
		\end{itemize}
		\item Electronic structures:
		\begin{itemize}
			\item Aufbau principle
			\item Pauli exclusion principle
			\item Using the periodic table as a cheat-sheet
			\item Exceptions to the rule -- Cu, etc.
			\item Ions and isoelectronic series.
		\end{itemize}
	\end{itemize}
	
	\subsection*{Chapter 7 -- Periodic Properties of the Elements}
	\begin{itemize}
		\item Development of the periodic table (Mendeleev)
		\item $Z_{eff}$ -- increases to the right	
		\item Atomic radius -- increases down a column and decreases across a column.
		\item Radii of ions -- anions are bigger, cations are smaller
		\item Ionization energy -- increases toward the top and right
		\item Electron afinity -- increases toward the top and right
		\item Reactive trends
	\end{itemize}

	\subsection*{Chapter 8 --  Chemical Bonding Basics}
	\begin{itemize}
		\item Metallic vs. Ionic vs. Covalent bonds
		\item Octet rule
		\item Lattice Energy
		\item Born Haber Cycle
		\item Electron sharing in covalent bonds
		\item Single, double, and triple bonds
		\item Electronegativity
		\item Polar bonds
		\item Polar molecules
		\item Drawing Lewis dot structures
		\item Formal charges
		\item Oxidation states in covalent compounds
		\item Delocalization and resonance structures
		\item Bond enthalpies
	\end{itemize}
\end{document}