\documentclass[12pt, letterpaper]{memoir}
\usepackage{ExamStyle}

\begin{document}
	\mainmatter
	
	\begin{center}
		\vspace{12em}
		{\Huge CHEM 1210}
		{\LARGE-- Fall 2019
		
		Midterm Exam 2}

		\vspace{45em}
		Feel free to use this page as scratch paper, but work for questions in the free response\\ section must be shown \emph{in that section} to count for credit.

	\end{center}

	\newpage
	
	\includepdf[pages={1},angle=90, width=1.08\textwidth]{Table.pdf}
	
	\newpage
		
	{\Large Name \rule[-1mm]{4in}{.1pt}}
	\section*{Multiple Choice -- 3 points each -- {\large Only one answer may be correct!}}
	
	\subsection*{Problem 1}
	Which of the following is a \emph{strong} electrolyte:
	
	\begin{tabular}{llll}
		A) \ch{CH3OH} & B) \ch{C6H12O6} & C) \ch{H3PO4}\\
		D) \ch{CH3COOH} & E) \ch{NaCl} & F) \ch{CO2}
	\end{tabular}
	
	\subsection*{Problem 2}
	What is the charge of Yttrium (\ch{Y}) in \ch{Y_2(CO_3)_3}?
	
	\begin{tabular}{lllllll}
		A) $6-$ ~~~& B) $4-$ ~~~& C) $3-$ ~~~& D) $2-$\\
		E) $2+$ ~~~& F) $3+$ ~~~& G) $4+$ ~~~& H) $6+$
	\end{tabular}
	
	\subsection*{Problem 3}
	Which of the following ionic compounds is \emph{insoluble}:
	
	\begin{tabular}{llll}
		A) \ch{NH4NO3}& B) \ch{Li2CO3}& C) \ch{CrSO4}\\
		D) \ch{Na3PO4}& E) \ch{NiBr2}& F) \ch{CdS}
	\end{tabular}	
	
	\subsection*{Problem 4}	
	Calcium and hydroxide ion combine to form which ionic compound?
	
	\begin{tabular}{lllll}
		A) \ch{Ca_2O_2} ~~~& B) \ch{CaO} ~~~& C) \ch{Ca(H2O)2}\\
		D) \ch{Ca(OH)_2} ~~~& E) \ch{CaOH_2} ~~~& F) \ch{CaH2O}
	\end{tabular}	
	
	\subsection*{Problem 5}
	Through combustion analysis, a compound is found to contain $54.26\%$ by weight C, $5.69\%$ by weight H, and $40.04\%$ by weight Cl.
	
	\noindent What is the empirical formula?

	\begin{tabular}{llll}
		A) \ch{C10HCl7} & B) \ch{C4H5Cl} & C) \ch{C19H2Cl14} \\
		D) \ch{C2H5Cl} & E) \ch{C3H7Cl5} & F) \ch{C5H10Cl2}
	\end{tabular}
	%C4H5Cl

	\subsection*{Problem 6}
	$4.72~g$ of \ch{Na2CO3 · $10$ H2O} are dissolved to make $125.0~ml$ of solution
	
	\noindent What is the molar concentration? 
	
	\begin{tabular}{llll}
		A) $0.132~M$ & B) $0.0378~M$ & C) $0.356~M$\\
		D) $0.305~M$ & E) $0.0165~M$ & F) $0.0445~M$
	\end{tabular}
	
	\subsection*{Problem 7}
	Give the classification for the following reaction:
	
	\ch{Ba(NO3)2(aq) + K2SO4(aq) -> BaSO4(s) + 2 KNO3(aq)}
	
	\begin{tabular}{llll}
		A) Synthesis/Combination & B) Redox & C) Precipitation\\
		D) Combustion & E) Acid/Base & F) Decomposition
	\end{tabular}

	\subsection*{Problem 8}
	Give the classification for the following reaction:
		
	\ch{2 Na(s) + Sn(NO3)2(aq) -> 2 NaNO3(aq) + Sn(s)}
	
	\begin{tabular}{llll}
		A) Synthesis/Combination & B) Redox & C) Precipitation\\
		D) Combustion & E)Acid/Base & F) Decomposition
	\end{tabular}

	\subsection*{Problem 9}
	Benzene has the empirical formula of \ch{CH} and a molar mass of $78.112~\dfrac{g}{mol}$.
	
	\noindent
	What is the \emph{molecular formula} for Benzene?
	
	\begin{tabular}{llll}
		A) \ch{CH} & B) \ch{C8H8} & C) \ch{C3H3}\\
		D) \ch{C5H5} & E) \ch{C2H2} & F) \ch{C6H6}
	\end{tabular}	
	
	\subsection*{Problem 10}
	Which of the following is a \emph{weak} acid?
	
	\begin{tabular}{llll}
		A) \ch{NH3} & B) \ch{HBrO2} & C) \ch{Na3PO4}\\
		D) \ch{C3H7OH} & E) \ch{H2SO4} & F) \ch{HClO3}
	\end{tabular}
	
%	\subsection*{Problem 5}
%	95.6 mg of an organic compound are burned in oxygen gas to give 269 mg CO2 and 110 mg H2O. 
%	
%	\noindent What is the empirical formula?
%
%	\begin{tabular}{llll}
%		A) \ch{C10H20O}& B) \ch{C4H5}& C) \ch{C5H8NO2}& D) \ch{C5H4O}\\
%		E) \ch{CH2}& F) \ch{C8H10O4}& G) \ch{C10H20}& H) \ch{CH2O}
%	\end{tabular}

%	\begin{tabular}{llll}
%		A) \ch{} & B) \ch{} & C) \ch{}\\
%		D) \ch{} & E) \ch{} & F) \ch{}
%	\end{tabular}

%	\begin{tabular}{llll}
%		A) $$ & B) $$ & C) $$\\
%		D) $$ & E) $$ & F) $$
%	\end{tabular}

%	\begin{tabular}{llll}
%		A) & B) & C) \\
%		D) & E) & F)
%	\end{tabular}	

%	\begin{tabular}{llll}
%		A) & B) & C) & D) \\
%		E) & F) & G) & H)
%	\end{tabular}	

%	\begin{tabular}{lllll}
%		A) & B) & C) & D) & E) \\
%		F) & G) & H) & I) & J)
%	\end{tabular}	
	
	\newpage
	\section*{Short Answer}
	\subsection*{Problem 11 -- 10 points}
	Give the oxidation state for each atom.
	
	\begin{tabular}{rcl}
		\\
		\ch{Al} in \ch{Al2O3} && \rule[-1mm]{3cm}{0.1pt}\\ \\
		\ch{N} in \ch{NO3^-} && \rule[-1mm]{3cm}{0.1pt}\\ \\
		\ch{Cl} in \ch{ClO4^-} && \rule[-1mm]{3cm}{0.1pt}\\ \\
		\ch{P} in \ch{H3PO4} && \rule[-1mm]{3cm}{0.1pt}\\ \\
		\ch{O} in \ch{O2} && \rule[-1mm]{3cm}{0.1pt}\\ \\
		\ch{C} in \ch{CH4} && \rule[-1mm]{3cm}{0.1pt}\\ \\
		\ch{N} in \ch{N2O4} && \rule[-1mm]{3cm}{0.1pt}\\ \\
		\ch{Br} in \ch{BrO2^-} && \rule[-1mm]{3cm}{0.1pt}\\ \\
		\ch{N} in \ch{HNO3} && \rule[-1mm]{3cm}{0.1pt}\\ \\
		\ch{C} in \ch{C2H6O} && \rule[-1mm]{3cm}{0.1pt}\\ 
	\end{tabular}

	\newpage
	\subsection*{Problem 12 -- 10 Points}	
	
	\noindent
	\hspace{-2em}
	\begin{minipage}[T]{0.55\textwidth}
		To the right is an electrochemical activity series table
		
		\noindent
		State whether the reactions will proceed\\ \emph{spontaneously} or not
		
		\vspace{0.5em}
		\ch{Ca(NO3)2(aq) + Sn(s) -> Sn(NO3)2(aq) + Ca(s)}
		
		\vspace{3em}
		\ch{2 Cr(NO3)3(aq) + 3 Zn(s) -> 3 Zn(NO3)2(aq) + 2 Cr(s)}
		
		\vspace{3em}
		\ch{2 Ag(NO3)(aq) + Pb(s) -> Pb(NO3)2(aq) + 2 Ag(s)}
		
		\vspace{3em}
		\ch{Ni(NO3)2(aq) + Mg(s) -> Mg(NO3)2(aq) + Ni(s)}
		
		\vspace{3em}
		\ch{2 Na(NO3)(aq) + H2(g) -> 2 HNO3(aq) + 2 Na(s)}
	\end{minipage}
	\begin{minipage}[T]{0.45\textwidth}
		\begin{tabular}{c|c}
			\multicolumn{2}{c}{Activity Series} \\ \midrule
			Best Reducing Agents & Worst Oxidizing Agents \\
			\ch{Li} & \ch{Li^+} \\
			\ch{Ca} & \ch{Ca^{2+}} \\
			\ch{Na} & \ch{Na^+} \\
			\ch{Mg} & \ch{Mg^{2+}} \\
			\ch{Al} & \ch{Al^{3+}} \\
			\ch{Zn} & \ch{Zn^{2+}} \\
			\ch{Cr} & \ch{Cr^{3+}} \\
			\ch{Fe} & \ch{Fe^{2+}} \\
			\ch{Co} & \ch{Co^{2+}} \\
			\ch{Ni} & \ch{Ni^{2+}} \\
			\ch{Sn} & \ch{Sn^{2+}} \\
			\ch{Pb} & \ch{Pb^{2+}} \\
			\ch{H2} & \ch{H^+} \\
			\ch{Cu} & \ch{Cu^{2+}} \\
			\ch{Ag} & \ch{Ag^+} \\
			\ch{Au} & \ch{Au^{+}} \\
			Worst Reducing Agents & Best Oxidizing Agents
			
		\end{tabular}
	\end{minipage}
	
	\newpage	
	\subsection*{Problem 13 -- 10 Points}
	Give the formula or name for each compound:
	
	\vspace{1em}
	\begin{tabular}{rl}
		magnesium chloride & \rule[-1mm]{3in}{1pt}\\ \\
		\ch{N_2O} & \rule[-1mm]{3in}{1pt}\\ \\
		\ch{SO_2} & \rule[-1mm]{3in}{1pt}\\ \\
		iron(III) sulfide & \rule[-1mm]{3in}{1pt}\\ \\
		\ch{Hypoiodous Acid} & \rule[-1mm]{3in}{1pt}\\ \\
		\ch{Ba(BrO_3)_2} & \rule[-1mm]{3in}{1pt}\\ \\
		diboron hexafluoride & \rule[-1mm]{3in}{1pt}\\ \\
		\ch{HNO2} & \rule[-1mm]{3in}{1pt}\\ \\
		nickel(II) perchlorate & \rule[-1mm]{3in}{1pt}\\ \\
		\ch{Cu_2SO_4} & \rule[-1mm]{3in}{1pt}		
	\end{tabular}

	\newpage
	\section*{Free Response -- Work \emph{must} be shown for full credit}
	\subsection*{Problem 14 -- 10 Points}
	$50.0~ml$ of $0.75~M$ \ch{AgNO3} are mixed with $30.0~ml$ of $0.85~M$ \ch{Na2S}. 
	
	\noindent $\circ$ Write the balanced \emph{net ionic equation} for what occurs after mixing.
	
	\vspace{8em}
	\noindent $\circ$ Which reactant is the limiting reactant?

	\vspace{15em}	
	\noindent $\circ$ How many grams of solid product are produced?
%	
%	\vspace{12em}
%	\noindent What is the \emph{concentration} of the remaining excess reagent?
		
	

	\newpage
	\subsection*{Problem 15 -- 10 Points}
	A student has a \ch{Ca(OH)2} solution with unknown concentration. She takes a $25.00~ml$ sample of the unknown solution, and titrates it with $0.225~M$ \ch{HCl} solution. The end point is reached when $41.72~ml$ of the acid have been added. 
	
	\noindent $\circ$ What is the unknown \ch{Ca(OH)2} concentration?	

	\vspace{12em}
	\subsection*{Problem 16 -- 10 Points}
	A student needs to prepare $200.0~ml$ of a \ch{NaCl} solution with a concentration of $5.00\times10^{-4}~M$. Because the target concentration is so dilute, he plans to do this in two steps. He first weighs out $0.157~g$ of \ch{NaCl} and dissolves it in a volumetric flask to produce $100.0~ml$ of solution.
	
	\noindent $\circ$ What is the concentration after this first step?
	
	\vspace{12em}
	\noindent For the second step, the student plans to measure a small portion of this solution and dilute it to a final volume of $200.0~ml$ in another volumetric flask.
	
	\noindent $\circ$ In order to reach the right target concentration, how many $ml$ of the solution should he use?
	
	\newpage
	\subsection*{Problem 17 -- 10 Points}
	A $100.0~g$ sample of an unknown compound is found to be composed of:
	\begin{itemize}
	\item $38.70~g$ \ch{C}
	\item $51.56~g$ \ch{O}
	\item $9.74~g$ \ch{H}
	\end{itemize}
	
	\noindent $\circ$ What is the empirical formula?
	
	\vspace{15em}\noindent
	Another test shows that the molar mass of the unknown compound is $62.07~\dfrac{g}{mol}$
	
	\noindent $\circ$ What is the molecular formula?
\end{document}

