\documentclass[12pt, openany, letterpaper]{memoir}
\usepackage{NotesStyle}
%\renewcommand\thesection{\thechapter\Alph{section}}
%\renewcommand\thesubsection{\thesection.\Numeral{subsection}}

\begin{document}
\begin{center}
{\Large Combustion Analysis Notes}

Prepared by Dr. Matthew Rowley, 2017
\end{center}

You may encounter a moderately difficult combustion analysis problem like the one below:

\begin{mdframed}
\noindent $1.80~g$ of an unknown substance was combusted in excess oxygen to produce $2.64~g$ of \ch{CO2} gas and $1.08~g$ of \ch{H2O} vapor. A separate analysis gave the molar mass of the substance as $180~\nicefrac{g}{mol}$. Using these data find both the empirical and molecular formulas for the unknown substance.
\end{mdframed}

For this sort of problem, we should recognize that the substance (we can call it \ch{X}) contains at least \ch{C} and \ch{H} atoms, and may contain \ch{O} as well. The general form of the reaction is:

$\ch{X + O2(g) -> CO2(g) + H2O(g)}$

Without the chemical formula for \ch{X}, we cannot balance the equation, but it is still useful because we note three things:
\begin{itemize}
	\item The \ch{C} atoms which make up the \ch{CO2} can only come from \ch{X}
	\item The \ch{H} atoms which make up the \ch{H2O} can only come from \ch{X}
	\item Since \ch{O} is in potentially \emph{every} chemical species in this equation, we cannot directly measure the amount which came from \ch{X}
\end{itemize}

The first step is to find the number of \ch{C} and \ch{H} atoms, which we know must have come from \ch{X}:

$\dfrac{2.64~g~\ch{CO2}}{1} \left|\dfrac{1~mol~\ch{CO2}}{44.0098~g~\ch{CO2}}\right| \dfrac{1~mol~\ch{C~atoms}}{1~mol~\ch{CO2}} = 0.0600~mol~\ch{C~atoms}$

$\dfrac{1.08~g~\ch{H2O}}{1} \left|\dfrac{1~mol~\ch{H2O}}{18.0153~g~\ch{H20}}\right| \dfrac{2~mol~\ch{H~atoms}}{1~mol~\ch{H2O}} = 0.120~mol~\ch{H~atoms}$

Now that we know exactly how many \ch{C} and \ch{H} atoms \ch{X} contains, we can see how much they weigh together.

$\dfrac{0.0600~mol~\ch{C~atoms}}{1}\left|\dfrac{12.011~g}{1~mol~\ch{C~atoms}}\right. = 0.721~g$

\vspace{0.5em}
$\dfrac{0.120~mol~\ch{H~atoms}}{1}\left|\dfrac{1.00794~g}{1~mol~\ch{H~atoms}}\right.=0.121~g$

\vspace{0.5em}
$0.721~g + 0.121~g = 0.842~g$

This falls far short of the original mass of \ch{X}, which means that there must be some \ch{O} present in \ch{X}. In order to count how many \ch{O} atoms there are, we must find the mass that they make up.

$m_{\ch{(O~atoms)}} = m_{(total)} - m_{\ch{(C~and~H~atoms)}}$

\vspace{0.5em}
$1.80~g - 0.842~g = 0.96~g$

\vspace{0.5em}
$\dfrac{0.96~g}{1}\left|\dfrac{1~mol~\ch{O~atoms}}{15.9994~g}\right. = 0.0600~\ch{O~atoms}$

Now we have:

\begin{tabular}{c|c|c|c}
Element & C & H & O \\ \midrule
Moles & $0.0600$ & $.120$ & $0.0600$
\end{tabular}

To get a reduced molar ratio, divide all moles by the smallest number (that is, by $0.0600$) to get:

\begin{tabular}{c|c|c|c}
Element & C & H & O \\ \midrule
Moles & $1$ & $2$ & $1$
\end{tabular}

So the empirical formula is: \ch{CH2O}, and the mass for the empirical formula is $30.03\nicefrac{g}{mol}$. Molecules with this empirical formula are called ``hydrocarbons'' because they are carbon with a water.

To find the \emph{molecular formula}, we need to recognize that it will be some multiple of the empirical formula (i.e. \ch{CH2O}, \ch{C2H4O2}, \ch{C3H6O3}, \ch{C4H8O4}, etc.).

$M_{\mathrm{Molecular}} = n\times M_{\mathrm{Empirical}}$, so $n = \dfrac{M_{\mathrm{Molecular}}}{M_{\mathrm{Empirical}}} = \dfrac{180\nicefrac{g}{mol}}{30.03\nicefrac{g}{mol}}=5.994\approx 6$

This gives us the actual molecular formula: $\ch{C6H12O6}$

\end{document}